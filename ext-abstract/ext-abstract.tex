% This is LLNCS.DEM the demonstration file of
% the LaTeX macro package from Springer-Verlag
% for Lecture Notes in Computer Science,
% version 2.4 for LaTeX2e as of 16. April 2010
%
\documentclass{llncs}
%
\usepackage{amsfonts}
\usepackage{comment}
%\usepackage{natbib}
\usepackage{graphicx}
\usepackage{caption}
\usepackage{subcaption}

\newcommand{\SunroofAnalog}[1]{#1\ensuremath{_\downarrow}}
\newcommand{\HaskellAnalog}[1]{#1\ensuremath{_\uparrow}}

\newcommand{\NOTE}[1]{{\Large\textbf{NOTE:}\ #1}}
\newcommand{\TODO}[1]{{\textbf{TODO:}\ #1}}
\newcommand{\Src}[1]{{\tt{#1}}}

\newcommand{\IO}{\Src{IO}}
\newcommand{\JS}{\Src{JS}}
\newcommand{\JSA}{\ensuremath{\Src{JS}_\Src{A}}}
\newcommand{\JSB}{\ensuremath{\Src{JS}_\Src{B}}}

\newcommand{\Figure}[3]{%
\begin{figure}[h]%
\vspace{-0.5cm}%
\begin{center}%
\includegraphics[scale=0.55,clip=true,trim=0.45cm 0.45cm 0.45cm 0.45cm]{#2}%
\vspace{-0.5cm}%
\end{center}%
\caption{#3}%
\label{#1}%
\vspace{-0.5cm}%
\end{figure}%
}

\begin{document}
%
\title{Sunroof: A JavaScript Monad Compiler}
\subtitle{(Extended Abstract)}
%
\titlerunning{Sunroof}  % abbreviated title (for running head)
%                                     also used for the TOC unless
%                                     \toctitle is used
%
\author{Jan Bracker\inst{1,2}  \and Andy Gill\inst{1}}
%
\authorrunning{Jan Bracker \and Andy Gill} % abbreviated author list (for running head)
%
%%%% list of authors for the TOC (use if author list has to be modified)
\tocauthor{Jan Bracker, Andy Gill}
%
\institute{%
ITTC / EECS \\
The University of Kansas\\
Lawrence, KS 66045\\
~\\
\and
Institut f{\"u}r Informatik\\
Christian-Albrechts-Universit{\"a}t\\
Kiel, Germany}


\maketitle              % typeset the title of the contribution

\begin{abstract}
Sunroof is a Haskell-hosted Domain Specific Language (DSL) for generating Javascript.
Sunroof is build on top of the \JS-monad, which, like the Haskell \IO-monad, allows 
access to external resources, but specifically Javascript
resources. As such, Sunroof is primarily a feature-rich foreign
function API to the browser's Javascript engine, and all the browser-specific
functionality, including HTML-based rendering, event handling, and 
drawing to the HTML5 canvas. 

In this paper, we give the design and implementation of Sunroof.
Using monadic reification, we reify a deep embedding of the \JS-monad,
and from this embedding we generate JavaScript.
The Sunroof DSL has the feel of native Haskell, with a simple
Haskell-based type schema to guide the Sunroof programmer.
Furthermore, because we are generating code,
we can offer Haskell-style concurrency patterns, like MVars and Channels.
In combination with a web services package like Scotty,
the Sunroof compiler offers a robust platform to build interactive web applications,
giving the ability to interleave Haskell and Javascript computations
with each other as needed.
\keywords{DSLs, Javascript, Web Technologies, Cloud Computing}
\end{abstract}

\section{Introduction}

We start with an overview of the language, then discuss
how the language is implemented, and discuss interesting
techniques used inside Sunroof. We close with a more
complete web application.

\section{The JavaScript Monad}

JavaScript is an imperative language with access to a wide range
of established and useful services, like graphical canvases and event
handling. JavaScript as a language also provides features that are
traditionally associated with functional languages, like first-class 
functions. We want to express JavaScript in Haskell, adding use
of Haskell's static typing, and gaining access to JavaScript services.
And we do so using the transitional functional programming 
mechanism for being imperative, a monad~\cite{Moggi:91:ComputationMonads}.

\begin{figure}[t]
\vspace{-0.5cm}
\centering
\begin{subfigure}{0.45\textwidth}%
\begin{verbatim}
 jsCode :: JS t ()
 jsCode = do
   name <- prompt "Your name?"
   alert ("Your name: " <> name)
\end{verbatim}%
\end{subfigure}%
\hfill%
\begin{subfigure}{0.45\textwidth}
\vspace{0.25cm}%
\begin{verbatim}
  
  
var v0 = prompt("Your name?"); 
alert("Your name: " + v0);
\end{verbatim}%
\end{subfigure}% 
%\vspace{-0.2cm}%
\caption{Sunroof program and the generated Javascript on the left.}%
\label{fig:code-example}%
%\vspace{-0.5cm}
\end{figure}

The \JS-monad is the monad of JavaScript effects, as is an almost
exact analog for the Haskell \IO-monad, except there
is an extra phantom argument~\cite{Leijen:99:Phantom} that we will return
to shortly.  Figure~\ref{fig:code-example} gives a first example
of the \JS-monad in use with the generated JavaScript.

Inside this simple example is a challenging problem -- where does
\Src{v0} come from? The bind inside the monadic \Src{do} is
unconstrained:
\begin{verbatim}
(>>=) :: JS t a -> (a -> JS t b) -> JS t b
\end{verbatim}
\noindent  What we want is:
\begin{verbatim}
(>>=) :: (Sunroof a) => JS t a -> (a -> JS t b) -> JS t b
\end{verbatim}
\noindent where \Src{Sunroof} constrains the bind to
arguments for which we can generate a JavaScript variable.
Counterintuitively, 
it turns out that a specific form of normalization allows 
the ``\Src{a}'' type to be constrained and the bind to 
be an instance of the standard monad class~\cite{Sculthorpe:13:ConstrainedMonads}.
Through this keyhole of {\em monadic reification\/},
the entire Sunroof language is realized.

\section{JavaScript Threading Models}

Sunroof was first documented in our previous 
workshop paper~\cite{Farmer:12:WebDSLs},
where the possibility of monadic reification was observed.
In this paper, we raised an unresolved issue, do you
generated atomic JavaScript code, and keep the callback
centric model of computation, or generate JavaScript
using CPS, and allow for blocking primitives,
like Haskell \Src{MVar}s. The latter, though more powerful, 
precluded using the compiler to generate
code that can be cleanly called from native JavaScript.
Both choices had poor consequences.

So, rather than pick one, we decided to explicitly support both,
and make both first class threading strategies in our compiler.
In terms of user-interface, we parameter the \JS-monad
with a phantom type that represents the threading model
to compile with, with \Src{A} for \Src{A}tomic threads,
and \Src{B} for \Src{B}locking (cooperative concurrency) threads. 
Atomic threads are classical JavaScript threads, and
are never interrupted; while blocking threads can
support suspending operations. By using phantom
types, we can express the necessary
restrictions on specific combinators, as well
as provide combinators to allow both types of
threads to cooperate successfully.

\section{JavaScript Object Model}

JavaScript is object based. It provides various objects,
including numbers, booleans, maps, and others. We
provide in Sunroof about a dozen common object,
including \Src{JSObject} (the generic object type), \Src{JSNumber}
(floating point numbers), \Src{JSCanvas} (HTML5 canvas type) 
and others. A simple
casting function is provided when the type-system
needs to be overwritten. Along with each of these types,
we provide typed methods.

\begin{verbatim}
 jsDrawBox :: JSObject -> JS t ()
 jsDrawBox document = do
     c <- document # getElementById("foo");
     cxt <- c # getContext("2d")
     cxt # drawRect (0,0,100,100)
\end{verbatim}

Here, \Src{\#} is a reverse apply, so the types
of the function in the above example are
\begin{verbatim}
(#) :: o -> (o -> JS t a) -> JS t a
getElementById :: JSString -> JSObject -> JS t JSContext
getContext :: JSString -> JSContext -> JS t JSCanvas
drawRect :: (JSNumber,JSNumber,JSNumber,JSNumber) -> JSCanvas -> JS t ()
\end{verbatim}        

From experience, even though we are targeting
an untyped language, the type system gets in the
way less than we expected.

\section{JavaScript Functions and Continuations}

Given these threading models, we can realize both as 
JavaScript objects. 
\begin{verbatim}
function     :: (...) => (a -> JS A b)  -> JS t (JSFunction a b)
continuation :: (...) => (a -> JS B ()) -> JS t (JSContinuation a)
\end{verbatim}

\Src{JSFunction a b} and \Src{JSContinuation a},
like \Src{JSObject} and others, are realized as objects
in JavaScript. Thus they can be passed as arguments, returned
from functions and stored in mutable structures.

Functions and continuations can be called using \Src{apply}
and \Src{goto} respectively. \Src{apply} calls the function
and returns, \Src{goto} calls the continuation, but never
returns.
\begin{verbatim}
apply :: (...) => JSFunction args ret -> args -> JS t ret
goto  :: (...) => JSContinuation args -> args -> JS t a
\end{verbatim}

\section{The Sunroof Compiler}

\Figure{fig:structure}{../figures/sunroof-structure.pdf}{The structure of Sunroof.}

Given the language, and monadic-reification, how do we  compile this language? 
Figure \ref{fig:structure} shows how Sunroof is structured.
On the lowest level we provide an untyped expression language \Src{Expr}
that describes Javascript expressions. 
To provide type safety when using Sunroof we create
wrappers for each type we want to represent, e.g. \Src{JSNumber} or \Src{JSString}.
Through the \Src{Sunroof} type class we provide an 
interface to create wrapped and unwrapped
instances of our expressions. Based on the wrappers we can provide 
operations specific to a certain type, e.g. a \Src{Num} instance
for \Src{JSNumber} or a \Src{Monoid} instance for \Src{JSString}.

This technique enables us to utilize  Haskells type system when writing Javascript
and offers an easy way to add new types when needed~\cite{Svenningsson:12:CombiningEmbedding}.
By using phantom types we can also provide more advanced types,
like \Src{JSArray a}.

The next layer provides Javascript instructions through the type \Src{JSI}.
They represents abstract statements. While expressions and values
represented with type wrappers are assumed to be free of side-effect,
the instructions model side-effects in Javascript. Examples for Instructions
are assignment of attributes or the application of a function.

As Javascript is an imperative language it is important that the deep embedding
of Sunroof provides a way to express sequences of instructions. 
The \Src{JS} monad provides this ability.
All types involved in a computation need to be constrained by the 
\Src{Sunroof} type class to enable their translation into Javascript.
To constrain the monad like this we normalize it using the operational
package~\cite{Hackage:10:Operational,Apfelmus:10:Operational} and
we put constraints on the constructors of the instructions~\cite{Sculthorpe:13:ConstrainedMonads}.

The \Src{JS} monad comes in two threading flavors, parameterized using a phantom type.
Internally the \Src{JS} monad is implemented using a continuation. For atomic
computations we just produce a list of statements with the continuation. 
When translating possibly blocking code we directly translate that continuation
into Javascript functions. This gives us the power to handle 
computations as values in the Javascript and store them if needed.

Blocking operations just store the rest of their computation in a queue.
When the event to unblock occurs the pending computation is registered 
as a callback that will be executed as soon as the current computation
is done.

On top of our JS monad we provide ways of specifying (typed)
interfaces to JavaScript capabilities, a Foreign Function Interface.

\section{The Sunroof Server}

The Sunroof compiler can compile JavaScript than can be used
stand-alone inside a web application. But Sunroof really comes
into its own when used with the Sunroof server. There
are three major functions in our server.

\begin{verbatim}
sunroofServer :: ... -> (SunroofEngine -> IO ()) -> IO ()
sync          :: SunroofEngine -> JS t a  -> IO (ResultOf a)
async         :: SunroofEngine -> JS t () -> IO ()
\end{verbatim}        

\Src{sunroofServer} starts a small web server,
that calls the callback function for each request.
\Src{sync} and \Src{async} allow the Sunroof programmer
to remotely execute monadic JavaScript from the server.
\Src{ResultOf} is a type-function, that maps the 
Sunroof type to a corresponding Haskell type.

\section{Extended Example}


%
% ---- Bibliography ----
%

\bibliographystyle{splncs03}
\bibliography{ext-abstract}

\end{document}
