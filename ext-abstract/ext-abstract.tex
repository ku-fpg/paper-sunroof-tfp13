% This is LLNCS.DEM the demonstration file of
% the LaTeX macro package from Springer-Verlag
% for Lecture Notes in Computer Science,
% version 2.4 for LaTeX2e as of 16. April 2010
%
\documentclass{llncs}
%
\usepackage{amsfonts}
\usepackage{comment}
\usepackage{natbib}
\usepackage{graphicx}
\usepackage{caption}
\usepackage{subcaption}

\newcommand{\SunroofAnalog}[1]{#1\ensuremath{_\downarrow}}
\newcommand{\HaskellAnalog}[1]{#1\ensuremath{_\uparrow}}

\newcommand{\NOTE}[1]{{\Large\textbf{NOTE:}\ #1}}
\newcommand{\TODO}[1]{{\textbf{TODO:}\ #1}}
\newcommand{\Src}[1]{{\sf #1}}

\newcommand{\JSA}{\ensuremath{\Src{JS}_\Src{A}}}
\newcommand{\JSB}{\ensuremath{\Src{JS}_\Src{B}}}

\newcommand{\Figure}[3]{%
\begin{figure}[h]%
\vspace{-0.5cm}%
\begin{center}%
\includegraphics[scale=0.55,clip=true,trim=0.45cm 0.45cm 0.45cm 0.45cm]{#2}%
\vspace{-0.5cm}%
\end{center}%
\caption{#3}%
\label{#1}%
\vspace{-0.5cm}%
\end{figure}%
}

\begin{document}
%
\title{Sunroof: A JavaScript Monad Compiler}
\subtitle{(Extended Abstract)}
%
\titlerunning{Sunroof}  % abbreviated title (for running head)
%                                     also used for the TOC unless
%                                     \toctitle is used
%
\author{Jan Bracker\inst{1,2}  \and Andy Gill\inst{1}}
%
\authorrunning{Jan Bracker \and Andy Gill} % abbreviated author list (for running head)
%
%%%% list of authors for the TOC (use if author list has to be modified)
\tocauthor{Jan Bracker, Andy Gill}
%
\institute{%
ITTC / EECS \\
The University of Kansas\\
Lawrence, KS 66045\\
~\\
\and
Institut f{\"u}r Informatik\\
Christian-Albrechts-Universit{\"a}t\\
Kiel, Germany}


\maketitle              % typeset the title of the contribution

\begin{abstract}        

Sunroof is a Haskell-hosted Domain Specific Language (DSL) for generating Javascript.
Sunroof is build on top of the JS-monad, which, like the Haskell IO-monad, allows 
access to external resources, but specifically Javascript
resources. As such, Sunroof is primarily a feature-rich foreign
function API to the browser's Javascript engine, and all the browser-specific
functionality, including HTML-based rendering, event handling, and 
drawing to the HTML5 canvas. 

In this paper, we give the design and implementation of Sunroof.
Using monadic reification, we reify a deep embedding of the JS-monad,
and from this embedding we generate JavaScript.
The Sunroof DSL has the feel of native Haskell, with a simple
Haskell-based type schema to guide the Sunroof programmer.
Furthermore, because we are generating code,
we can offer Haskell-style concurrency patterns, like MVars and Channels.
In combination with a web services package like Scotty,
The Sunroof compiler offers a robust platform to build interactive web applications,
giving the ability to interleave Haskell and Javascript computations
with each other as needed.
\keywords{DSLs, Javascript, Web Technologies, Cloud Computing}
\end{abstract}

We start with an overview of the language, then discuss
how the language is implemented, and discuss interesting
techniques used inside Sunroof. We close with a more
complete web application.

\section{The JavaScript Monad}

JavaScript is an imperative language with access to a wide range
of established and useful services, like graphical canvases and event
handling. JavaScript as a language also provides features that are
traditionally associated with functional languages, like first-class 
functions. We want to express JavaScript in Haskell, adding use
of Haskell's static typing, and gaining access to JavaScript services.
And we do so using the transitional functional programming 
mechanism for being imperative, a monad~\cite{..}.

\begin{figure}[t]
\vspace{-0.5cm}
\centering
\begin{subfigure}{0.45\textwidth}%
\begin{verbatim}
 jsCode :: JS t ()
 jsCode = do
   name <- prompt "Your name?"
   alert ("Your name: " <> name)
\end{verbatim}%
\end{subfigure}%
\hfill%
\begin{subfigure}{0.45\textwidth}
\vspace{0.25cm}%
\begin{verbatim}
  
  
var v0 = prompt("Your name?"); 
alert("Your name: " + v0);
\end{verbatim}%
\end{subfigure}% 
%\vspace{-0.2cm}%
\caption{Sunroof program and the generated Javascript on the left.}%
\label{fig:code-example}%
%\vspace{-0.5cm}
\end{figure}

The JS-monad is the monad of JavaScript effect, as is an almost
exact analog for the Haskell IO-monad, except there
is an extra phantom argument~\cite{..} that we will return
to use shortly.  Figure~\ref{fig:code-example} give a first example
of the JS-monad in use, and the generated JavaScript.

Inside this simple example is a challenging problem -- where does
\verb|v0| come from? The bind inside the monadic \verb|do| is
unconstrained:
\begin{verbatim}
(>>=) :: forall a b . JS t a -> (a -> JS t b) -> JS t b
\end{verbatim}
\noindent  What we want is:
\begin{verbatim}
(>>=) :: (Sunroof a) => forall a b . JS t a -> (a -> JS t b) -> JS t b
\end{verbatim}
\noindent where \verb|Sunroof| constrains the bind to
arguments for which we can generate a JavaScript \verb|var|.
Counterintuitively, 
it turns out that a specific form of normalization allows 
the ``\verb|a|'' type to be constrained and the bind to 
be an instance of the standard monad class~\cite{..}.
Though this keyhole of {\em monadic reification\/},
the entire Sunroof language is realized.

\section{JavaScript Threading Models}

Sunroof was first documented in our previous 
workshop paper~\cite{Farmer:12:WebDSLs},
where the possibility was monadic reification was observed.
In this paper, we raised an unresolved issue, do you
generated atomic JavaScript code, and keep the callback
centric model of computation, or generate JavaScript
using CPS, and allow for blocking primitives,
like Haskel MVars. The latter, though more powerful, 
precluded using the compiler to generate
code that can be cleanly called from native JavaScript.
Both choices had poor consequences.

So, rather than pick one, we decided to explicitly support both,
and make both first class threading strategies in our compiler.
In terms of user-interface, we parameter the JS-monad
with a phantom type that represents threading model
to compile with, with \verb|A| for {\tt A}tomic threads,
and \verb|B| for {\tt B}locking (cooperative concurrency) threads. 
Atomic threads are classical JavaScript threads, and
are never interrupted; while blocking threads can
support blockable operations. And using phantom
types, we can express using types the necessary
restrictions on specific combinators, as well
as provide combinator to allow both types of
threads to cooperate successfully.

\section{JavaScript Object Model}

JavaScript is object based. It provides various objects,
including numbers, booleans, maps, and others. We
provide in Sunroof about a dozen common object,
including JSObject (the generic object) JSNumber
(floating point JavaScript numbers), JSCanvas,
the HTML5 canvas type, and others. A simple
casting function is provided when the type-system
needs overwritten.  Along with each of these types,
we provide typed methods.

\begin{verbatim}
 jsDrawBox :: JSObject -> JS t ()
 jsDrawBox window = do
     c <- window # getElementById("foo");
     cxt <- c # getContext("2d")
     cxt # drawRect (0,0,100,100)
\end{verbatim}

Here, \verb|#| is a reverse apply, so the types
of the function in the above example are
\begin{verbatim}
(#) :: o -> (o -> JS t a) -> JS t a
getElementById :: String -> JSObject -> JS t JSContext
getContext :: String -> JSContext -> JS t JSCanvas
drawRect :: (JSNumber,JSNumber,JSNumber,JSNumber) -> JSCanvas -> JS t ()
\end{verbatim}        

From experience, even though we are targeting
an untyped language, the type system gets in the
way less than we expected.

\section{JavaScript Functions and Continuations}

Given this threading models, we can realize both as 
JavaScript objects. 
\begin{verbatim}
function     :: (...) => (a -> JS A b)  -> JS a (JSFunction a b)
continuation :: (...) => (a -> JS B ()) -> JS a (JSContinuation a)
\end{verbatim}

\verb|JSFunction a b| and \verb|JSContinuation a|,
like \verb|JSObject| and others, are realized as objects
in JavaScript, can be passed to 

 As our closing example, we build one such application.

Furthermore,
because we are {\em generating JavaScript}, we can two different
threading models for JavaScript, atomic threaded code and blockable threads,
and separate these threads in a principled way using phantom types.
This enhanced threading model allows full access to Javascript APIs,
but also allows for analogues of the traditional
Haskell concurrency patterns, like MVars and Channels,
as well as first-class continuations.
We give examples of each of these features, explaining the
design space, how Sunroof operates internally and externally.


Using monadic reification, we reify a deep embedding of the JS-monad,
as well as an accompanying embedding of JavaScript expressions.
This small language gives a solid based on which to build application
in JavaScript.

This makes it easy to use Haskell abstractions for larger Javascript
applications without obscuring the produced Javascript on the Haskell
level. 


Figure \ref{fig:structure} shows how Sunroof is structured.
On the lowest level we provide an untyped expression language \Src{Expr}
that describes Javascript expressions. 
To provide type safety when using Sunroof we create wrappers for each 
type we want to represent. Examples for wrappers are \Src{JSNumber}
or \Src{JSString}. Through a type class
called \Src{Sunroof} we provide an interface to create wrapped and unwrapped
instances of our expressions. Based on the wrappers we can provide 
operations specific to a certain type, e.g. a \Src{Num} instance
for \Src{JSNumber} or a \Src{Monoid} instance for \Src{JSString}.

\Figure{fig:structure}{../figures/sunroof-structure.pdf}{The structure of Sunroof.}

This technique enables us to utilize 
Haskells type system when writing Javascript
and offers an easy way to add new types when needed.
It is known from 
\citet{Svenningsson:12:CombiningEmbedding}.
By using phantom types we can also provide more advanced types,
like $\Src{JSArray\ \alpha}$ or $\Src{JSFunction\ \alpha\ \beta}$.

The next layer provides Javascript instructions through the type \Src{JSI}.
They represents abstract statements. While expressions and values
represented with type wrappers are assumed to be free of side-effect,
the instructions model side-effects in Javascript. Examples for Instructions
are assignment of attributes or the application of a function.

As Javascript is a imperative language it is important that the deep embedding
of Sunroof provides a way to express sequences of instructions. 
The \Src{JS} monad provides this ability.
All types involved in a computation need to be constrained by the 
\Src{Sunroof} type class to enable their translation into Javascript.
To constrain the monad like this we normalize it using the operational package
\citep{Hackage:10:Operational,Apfelmus:10:Operational} and
we put constraints on the constructors of the instructions,
as described in detail by \citep{Sculthorpe:13:ConstrainedMonads}.
Using this approach we can use the do-notation to write Javascript 
in a intuitive way. The bind operator is directly translated to a variable
assignment and bound variables can be seen as references to previously 
bound values. An example for this can be seen in figure \ref{fig:code-example}.


The \Src{JS} monad comes in two flavors. One for atomic
($\JSA$) and one for possibly blocking 
computations ($\JSB$). The motivation for this distinction
lays in Javascripts threading model. It does not have multi-threading,
but it allows to register callbacks which a invoked as soon as the 
current computation is done. 
When using this threading model one can use the $\JSA$ monad.
But since we think that the multi-threading paradigms provided by
Haskell (e.g. \Src{forkIO}, \Src{MVar} or \Src{Chan}) are valuable 
to programmers and can ease the development of possible applications 
we provide them within the $\JSB$ monad.
Of course, every $\JSA$ instance can be lifted into a $\JSB$
instance, but since suspending or blocking the execution is not 
possible in $\JSA$ we can not allow lifting $\JSB$ into $\JSA$.
Inside the $\JSA$ monad we provide operations like \Src{forkJS} 
or the types \Src{JSMVar} and \Src{JSChan} to provide the same
abstractions as in Haskell on the Javascript level. This allows us
to write programs using Haskell threading idioms. Due to 
the non-threaded nature of Javascript, we can only provide 
cooperative multi-threading.

Internally the \Src{JS} monad is implemented using a continuation. For atomic
computations we just produce a list of statements with the continuation. 
When translating possibly blocking code we directly translate that continuation
into Javascript functions. This gives us the power to handle 
computations as values in the Javascript and store them if needed.
So blocking operations just store the rest of their computation in a queue.
When the event to unblock occurs the pending computation is registered 
as a callback that will be executed as soon as the current computation
is done.

\TODO{This stuff is not mentioned yet:}
\begin{itemize}
\item Conection: Function $\Leftrightarrow$ JSFunction, 
                 Continuation $\Leftrightarrow$ JSContinuation;
\item Direct mention of the contributions
\item Foreign Function Interface
\item Discussion: Why not just write JS by hand?
\item Deep Embedding of Numbers and Booleans through Data.Boolean (minor)
\item Design decision (minor)
\item Provided API (minor)
\end{itemize}


\begin{comment}
Table \ref{tab:SunroofTypes} shows how Sunroof provides 
analog versions of basic Haskell types in the Javascript domain.

\begin{table}
\caption{Major Sunroof Types}
\label{tab:SunroofTypes}
\begin{center}
\begin{tabular}{r@{\quad}l@{\quad}l@{\quad}c}
\hline\rule{0pt}{12pt}%
  Constraint
  & Sunroof Type $\tau$
  & Haskell Analog \HaskellAnalog{$\tau$} 
  \\ \hline\rule{0pt}{12pt}%
  
  & \Src{()}       & \Src{()}     \\
  & \Src{JSBool}   & \Src{Bool}   \\
  & \Src{JSNumber} & \Src{Double} \\
  & \Src{JSString} & \Src{String} \\
  
  \Src{Sunroof $\alpha$}
  & \Src{JSArray $\alpha$} 
  & \Src{[$\HaskellAnalog{\alpha}$]} \\
  
  \Src{SunroofArgument $\alpha$}
  & \Src{JSFunction $\alpha$ $\beta$ }
  & \Src{$\HaskellAnalog{\alpha}$ $\rightarrow$ JS$_\Src{A}$ $\HaskellAnalog{\beta}$} \\
  \Src{Sunroof $\beta$} \\ \hline
\end{tabular}
\end{center}
\end{table}
\end{comment}

%
% ---- Bibliography ----
%
\bibliographystyle{abbrvnat}
%\setlength{\bibsep}{0pt}
\bibliography{ext-abstract}

\end{document}
