% This is LLNCS.DEM the demonstration file of
% the LaTeX macro package from Springer-Verlag
% for Lecture Notes in Computer Science,
% version 2.4 for LaTeX2e as of 16. April 2010
%
\documentclass{llncs}
%
\usepackage{amsfonts}
\usepackage{comment}
\usepackage{natbib}
\usepackage{graphicx}
\usepackage{caption}
\usepackage{subcaption}

\newcommand{\SunroofAnalog}[1]{#1\ensuremath{_\downarrow}}
\newcommand{\HaskellAnalog}[1]{#1\ensuremath{_\uparrow}}

\newcommand{\NOTE}[1]{{\Large\textbf{NOTE:}\ #1}}
\newcommand{\TODO}[1]{{\textbf{TODO:}\ #1}}
\newcommand{\Src}[1]{{\sf #1}}

\newcommand{\JSA}{\ensuremath{\Src{JS}_\Src{A}}}
\newcommand{\JSB}{\ensuremath{\Src{JS}_\Src{B}}}

\newcommand{\Figure}[3]{%
\begin{figure}[h]%
\vspace{-0.5cm}%
\begin{center}%
\includegraphics[scale=0.55,clip=true,trim=0.45cm 0.45cm 0.45cm 0.45cm]{#2}%
\vspace{-0.5cm}%
\end{center}%
\caption{#3}%
\label{#1}%
\vspace{-0.5cm}%
\end{figure}%
}

\begin{document}
%
\title{Sunroof}
%
\titlerunning{Sunroof}  % abbreviated title (for running head)
%                                     also used for the TOC unless
%                                     \toctitle is used
%
\author{Jan Bracker \and Andy Gill}
%
\authorrunning{Jan Bracker \and Andy Gill} % abbreviated author list (for running head)
%
%%%% list of authors for the TOC (use if author list has to be modified)
\tocauthor{Jan Bracker, Andy Gill}
%
\institute{%
Information Technology and Telecommunication Center\\
Department of Electrical Engineering and Computer Science\\
The University of Kansas\\
2335 Irving Hill Road\\
Lawrence, KS 66045\\
   {\{???\}@ku.edu}%
}

\maketitle              % typeset the title of the contribution

Sunroof is a Domain Specific Language (DSL) for generating Javascript.
Sunroof is build on top of the JS-monad, which, like the Haskell IO-monad, allows 
read and write access to external resources, but specifically Javascript
resources. As such, Sunroof is primarily a feature-rich foreign
function API to the browser's Javascript engine, and all the browser-specific
functionality, like HTML-based rendering, event handling, and 
drawing to the HTML5 canvas. 

In this paper, we give the design and implementation of Sunroof, a 
deeply embedded Haskell-hosted DSL.
This makes it easy to use Haskell abstractions for larger Javascript
applications without obscuring the produced Javascript on the Haskell
level. 
Furthermore, Sunroof offers two threading models for 
building on top Javascript, atomic and blocking threads.
This allows full access to Javascript APIs, but
using Haskell concurrency patterns, like MVars and Channels.
In combination with a small web services package, like Scotty,
Sunroof offers a great platform to build interactive web applications,
giving the ability to interleave Haskell and Javascript computations
with each other as needed. 
We leverage the work from \cite{Farmer:12:WebDSLs}.

Figure \ref{fig:structure} shows how Sunroof is structured.
On the lowest level we provide an untyped expression language \Src{Expr}
that describes Javascript expressions. 
To provide type safety when using Sunroof we create wrappers for each 
type we want to represent. Examples for wrappers are \Src{JSNumber}
or \Src{JSString}. Through a type class
called \Src{Sunroof} we provide an interface to create wrapped and unwrapped
instances of our expressions. Based on the wrappers we can provide 
operations specific to a certain type, e.g. a \Src{Num} instance
for \Src{JSNumber} or a \Src{Monoid} instance for \Src{JSString}.

\Figure{fig:structure}{../figures/sunroof-structure.pdf}{The structure of Sunroof.}

This technique enables us to utilize 
Haskells type system when writing Javascript
and offers an easy way to add new types when needed.
It is known from 
\citet{Svenningsson:12:CombiningEmbedding}.
By using phantom types we can also provide more advanced types,
like $\Src{JSArray\ \alpha}$ or $\Src{JSFunction\ \alpha\ \beta}$.

The next layer provides Javascript instructions through the type \Src{JSI}.
They represents abstract statements. While expressions and values
represented with type wrappers are assumed to be free of side-effect,
the instructions model side-effects in Javascript. Examples for Instructions
are assignment of attributes or the application of a function.

As Javascript is a imperative language it is important that the deep embedding
of Sunroof provides a way to express sequences of instructions. 
The \Src{JS} monad provides this ability.
All types involved in a computation need to be constrained by the 
\Src{Sunroof} type class to enable their translation into Javascript.
To constrain the monad like this we normalize it using the operational package
\citep{Hackage:10:Operational,Apfelmus:10:Operational} and
we put constraints on the constructors of the instructions,
as described in detail by \citep{Sculthorpe:13:ConstrainedMonads}.
Using this approach we can use the do-notation to write Javascript 
in a intuitive way. The bind operator is directly translated to a variable
assignment and bound variables can be seen as references to previously 
bound values. An example for this can be seen in figure \ref{fig:code-example}.

\begin{figure}
\vspace{-0.5cm}
\centering
\begin{subfigure}{0.45\textwidth}%
\begin{verbatim}
 jsCode :: JS A ()
 jsCode = do
   name <- prompt "Your name?"
   alert ("Your name: " <> name)
\end{verbatim}%
\end{subfigure}%
\hfill%
\begin{subfigure}{0.45\textwidth}
\vspace{0.25cm}%
\begin{verbatim}
  
  
var v0 = prompt("Your name?"); 
alert("Your name: " + v0);
\end{verbatim}%
\end{subfigure}% 
\vspace{-0.2cm}%
\caption{Sunroof program and the generated Javascript on the left.}%
\label{fig:code-example}%
\vspace{-0.5cm}
\end{figure}

The \Src{JS} monad comes in two flavors. One for atomic
($\JSA$) and one for possibly blocking 
computations ($\JSB$). The motivation for this distinction
lays in Javascripts threading model. It does not have multi-threading,
but it allows to register callbacks which a invoked as soon as the 
current computation is done. 
When using this threading model one can use the $\JSA$ monad.
But since we think that the multi-threading paradigms provided by
Haskell (e.g. \Src{forkIO}, \Src{MVar} or \Src{Chan}) are valuable 
to programmers and can ease the development of possible applications 
we provide them within the $\JSB$ monad.
Of course, every $\JSA$ instance can be lifted into a $\JSB$
instance, but since suspending or blocking the execution is not 
possible in $\JSA$ we can not allow lifting $\JSB$ into $\JSA$.
Inside the $\JSA$ monad we provide operations like \Src{forkJS} 
or the types \Src{JSMVar} and \Src{JSChan} to provide the same
abstractions as in Haskell on the Javascript level. This allows us
to write programs using Haskell threading idioms. Due to 
the non-threaded nature of Javascript, we can only provide 
cooperative multi-threading.

Internally the \Src{JS} monad is implemented using a continuation. For atomic
computations we just produce a list of statements with the continuation. 
When translating possibly blocking code we directly translate that continuation
into Javascript functions. This gives us the power to handle 
computations as values in the Javascript and store them if needed.
So blocking operations just store the rest of their computation in a queue.
When the event to unblock occurs the pending computation is registered 
as a callback that will be executed as soon as the current computation
is done.

\TODO{This stuff is not mentioned yet:}
\begin{itemize}
\item Conection: Function $\Leftrightarrow$ JSFunction, 
                 Continuation $\Leftrightarrow$ JSContinuation;
\item Direct mention of the contributions
\item Foreign Function Interface
\item Discussion: Why not just write JS by hand?
\item Deep Embedding of Numbers and Booleans through Data.Boolean (minor)
\item Design decision (minor)
\item Provided API (minor)
\end{itemize}


\begin{comment}
Table \ref{tab:SunroofTypes} shows how Sunroof provides 
analog versions of basic Haskell types in the Javascript domain.

\begin{table}
\caption{Major Sunroof Types}
\label{tab:SunroofTypes}
\begin{center}
\begin{tabular}{r@{\quad}l@{\quad}l@{\quad}c}
\hline\rule{0pt}{12pt}%
  Constraint
  & Sunroof Type $\tau$
  & Haskell Analog \HaskellAnalog{$\tau$} 
  \\ \hline\rule{0pt}{12pt}%
  
  & \Src{()}       & \Src{()}     \\
  & \Src{JSBool}   & \Src{Bool}   \\
  & \Src{JSNumber} & \Src{Double} \\
  & \Src{JSString} & \Src{String} \\
  
  \Src{Sunroof $\alpha$}
  & \Src{JSArray $\alpha$} 
  & \Src{[$\HaskellAnalog{\alpha}$]} \\
  
  \Src{SunroofArgument $\alpha$}
  & \Src{JSFunction $\alpha$ $\beta$ }
  & \Src{$\HaskellAnalog{\alpha}$ $\rightarrow$ JS$_\Src{A}$ $\HaskellAnalog{\beta}$} \\
  \Src{Sunroof $\beta$} \\ \hline
\end{tabular}
\end{center}
\end{table}
\end{comment}

%
% ---- Bibliography ----
%
\bibliographystyle{abbrvnat}
%\setlength{\bibsep}{0pt}
\bibliography{ext-abstract}

\end{document}
