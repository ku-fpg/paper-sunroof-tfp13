 
\section{Conclusion}

Sunroof took the key idea of monad reification and
successfully created the \JS-monad to describe computations
in JavaScript. This work was started by Farmer and
Gill \cite{Farmer:12:WebDSLs}, with the observation
of the possibility to reify a JS monad.
This paper documents the work since this initial implementation.
By adding the concept 
of \Src{JSFunction} and \Src{JSContinuation}, there now is a 
stronger connection between
functions in the JavaScript and the Sunroof language space 
(\FigRef{fig:func-cont}). It is possible to go back and forth between 
both worlds. Combining both concepts, functions and the \JS-monad,
we were able to create a second implementation of the monad, this
time based on the direct translation of continuations from Haskell
to JavaScript. It enabled us to build a blocking threading model
on top of JavaScript that resembles the model already known from Haskell.
Based on this model and the provided abstraction over continuations,
we can use primitives like \Src{forkJS} or \Src{yield}.
Higher-level abstractions like \Src{JSMVar} and \Src{JSChan} are also
available. 
