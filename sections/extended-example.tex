 
\section{Case Study - A small Calculator}
\label{sec:extended-example}

To see how Sunroof works in practice we will look into the 
experience we gathered when writing a small calculator
for arithmetic expressions (Figure \ref{fig:example-application}). 
We use Sunroof to display our interface
and the results of our computation. Haskell will be used to parse the 
arithmetic expressions and calculate the result. The Sunroof server 
will be used to implement this JavaScript/Haskell hybrid.

\FigureS%
{fig:example-application}%
{figures/example-application.png}%
{The example application running on the Sunroof server.}%
{scale=0.76}

The classical approach to develop an application like this would have 
been to write a server that provides a RESTful interface and replies 
through a JSON data structure. This can be seen in Figure 
\ref{fig:example-structure}.

\Figure%
{fig:example-structure}%
{figures/example-structure.pdf}%
{Classical structure and Sunroof structure of a web application.}

The client side of that application would have been written in JavaScript
directly.

How does Sunroof improve or change this classical structure?
First of all, in Sunroof you write the client side code together with
your server application within Haskell. In our example all code 
for the server and client is in Haskell. The control logic 
for the client side is provided through the server.
This leads to a tight coupling between both sides. 
It enforces both sides to work together
correctly because they share types and interfaces. 
This also shows how Sunroof blurs the border between the server 
and client side. You are not restricted by an interface or language 
barrier. If you need the client to do something you can just 
send arbitrary Sunroof code to execute in the client.
Setting up the client-server response loop, which shuffles 
new input to the server and executes the response in the client
took about 25 lines of Haskell.

A problem that still arises is data conversion. Pure Haskell data types
can not be handled in Sunroof and vice versa, because of the language
boundary between JavaScript and Haskell. So code to convert between two 
essentially equal data structures on each side has to be written,
as wall as representation of Haskell structures in Sunroof. An
essential part of our example deals with conversion and encoding of
data in JavaScript. About 85 lines are devoted to this task.
But there is a great potential in automatically generating this
kind of Sunroof infrastructure using techniques like template Haskell,
since most of the written code is systematic.

The client code for displaying the results is basically a 
transliteration of the JavaScript that one would write for this 
purpose. It makes up main part of the application, about 190 lines.
The transliteration going on here is not very appealing, but can 
be automated through higher level libraries and interfaces, which
Sunroof is intended to deliver a foundation for.

The rest of our code to parse the arithmetic expression and calculate 
result takes up about 85 lines. 

What did we gain? First of all, in a sense we blurred the line between
both languages. Code for client and server side can be written in one place
instead of being separated. Like this we do not have to worry about 
either of both sides getting out of sync with the other as much, since
the type system helps us. We also noticed that we have to write
an considerable amount of boilerplate code to communicate data that goes beyond 
the basic data types Sunroof already support. But this process can automatized.
The transliteration of actual JavaScript into Sunroof has a overhead and
tends to be verbose. This problem can be hidden away through libraries building
on Sunroof. So the main gain is a solid and robust foundation to build
on more advanced systems that want to utilize the browser. Also basic
communication between server and browser was simplified.

\TODO{This section turned out in weird way. I am not satisfied.}

