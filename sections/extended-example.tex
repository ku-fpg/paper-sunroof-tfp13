 
\section{Case Study - A small Calculator}
\label{sec:extended-example}

To see how Sunroof works in practice we will look into the 
experience we gathered when writing a small calculator
for arithmetic expressions (Figure \ref{fig:example-application}). 
We use Sunroof to display our interface
and the results of our computation. Haskell will be used to parse the 
arithmetic expressions and calculate the result. The Sunroof server 
will be used to implement this JavaScript/Haskell hybrid.

\FigureS%
{fig:example-application}%
{figures/example-application.png}%
{The example application running on the Sunroof server.}%
{scale=0.6}

The classical approach to develop an application like this would have 
been to write a server that provides a RESTful interface and replies 
through a JSON data structure. 
The client side of that application would have been written in JavaScript
directly.
This can be seen in Figure \ref{fig:example-structure}.

\Figure%
{fig:example-structure}%
{figures/example-structure.pdf}%
{Classical structure and Sunroof structure of a web application.}

How does Sunroof improve or change this classical structure?
First of all, in Sunroof you write the client side code together with
your server application within Haskell. In our example all code 
for the server and client is in Haskell. The control logic 
for the client side is provided through the server.
This leads to a tight coupling between both sides. 
It enforces both sides to work together
correctly because they share types and interfaces. 
This also shows how Sunroof blurs the border between the server 
and client side. You are not restricted by an interface or language 
barrier. If you need the client to do something you can just 
send arbitrary Sunroof code to execute in the client.

Lets look into a few statistics (Table \ref{tab:example-statistics}) 
about how much code it took to write which parts of the client.

\begin{table}
\begin{center}
\begin{tabular}{l@{\quad}r@{\quad}r}
\hline\rule{0pt}{12pt}%
Part of Application & Lines of Code & Percentage \\[2pt]
\hline\rule{0pt}{12pt}%
Response loop & 25 & 6.5\% \\[2pt]
Data conversion & 85 & 22.0\% \\[2pt]
Rendering & 190 & 49.5\% \\[2pt]
Parsing and Interpretation & 85 & 22.0\% \\[2pt]
\hline
\end{tabular}
\end{center}
\caption{Lines of code needed for the example.}
\label{tab:example-statistics}
\end{table} 

The client-server response loop shuffles new input to the server 
and executes the response in the client.

Data conversion is needed, because pure Haskell data types
can not be handled in Sunroof and vice versa. There still
is a language boundary between JavaScript and Haskell. 
Code to convert between two essentially equal data structures on 
each side has to be written, as wall as representation of Haskell 
structures in Sunroof. But there is great potential in automatically 
generating this code using techniques like template Haskell.

The code for displaying the results is basically a 
transliteration of the JavaScript that you would write for this 
purpose.
The transliteration going on here is not very appealing. 
In future this code can be generated through higher level 
libraries. Sunroof is intended to deliver a foundation for
this purpose.

The rest of our code to parse the arithmetic expression and calculate 
result is classical Haskell code. 

So right now we have a lot of boilerplate code for data conversion
that has potential to be generated.
The transliteration of actual JavaScript into Sunroof has a overhead and
tends to be verbose. This problem can be hidden away through 
higher level libraries using Sunroof as a backend. 
We can see that Sunroof has the potential to be a low level interface
to the browsers capabilities. It offers a solid and robust foundation to build
on more advanced systems that want to utilize the browser.

