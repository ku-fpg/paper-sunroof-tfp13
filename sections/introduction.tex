 
\section{Introduction}

% Simon: Describe the problem
JavaScript is an imperative language with access to a wide range
of established and useful services, like graphical canvases and event
handling. JavaScript also provides features that are associated with 
functional languages, like first-class functions. 
We want to express JavaScript in Haskell, adding use
of Haskell's static typing, and gaining access to JavaScript services
in the Browser from Haskell.

% Simon: State your contributions
Sunroofs was developed to tackle this goal.
The small example in Figure \ref{fig:code-example} 
shows how Sunroof feels.
% Simon: An illustrative example
\begin{figure}[h]
\vspace{-0.5cm}
\centering
\begin{subfigure}{0.45\textwidth}%
\begin{Code}
 jsCode :: JS t ()
 jsCode = do
   name <- prompt "Your name?"
   alert ("Your name: " <> name)
\end{Code}%
\end{subfigure}%
\hfill%
\begin{subfigure}{0.45\textwidth}
\vspace{0.25cm}%
\begin{Code}
  
  
var v0 = prompt("Your name?"); 
alert("Your name: " + v0);
\end{Code}%
\end{subfigure}% 
%\vspace{-0.2cm}%
\caption{Sunroof program and the expected JavaScript on the left.}%
\label{fig:code-example}%
\vspace{-0.5cm}
\end{figure}
% Simon: Why is this useful?
The expected JavaScript output is shown on the right. But what makes 
this useful? We produced code that is shorter 
then the Sunroof code on the left and can easily be written by hand.

% Simon: List of contributions
Sunroof's approach has several advantages. 
It introduces a threading model and abstraction similar
to Haskell's. Like this programmers can reuse well known 
techniques from Haskell when writing JavaScript.
We also utilize the static type system to support us when 
writing code. The functions
in Figure \ref{fig:code-example} have the following signatures:
\begin{Code}
prompt :: JSString -> JSString -> JS t JSObject
alert  :: JSString -> JS t ()
\end{Code}
This improves the quality of written code.
Using a deep embedding gives opportunity 
for optimizations when producing the actual JavaScript.
At the same time Sunroof offers an interface that is 
close to actual JavaScript making it easy to use.
The interface is easily extendable through
a foreign function interface.
We model the imperative nature of JavaScript
using the transitional functional programming 
mechanism to be imperative, a monad~\cite{Moggi:91:ComputationMonads}.
% The scope of this paper is to show how Sunroof achieves these goals.
% It will cover the most important parts of Sunroof and how they are 
% implemented.
\Figure%
{fig:structure}%
{figures/sunroof-structure.pdf}%
{The structure of Sunroof.}

We will go cover each of the layers in Figure \ref{fig:structure}
throughout the paper:
\begin{itemize}
\item
The \JS-monad together with the underlying \JSI~primitives 
will be explained in Section \ref{sec:js-monad}. 
We will show how it is implemented and how we solved
the problem of constraining types involved in the monadic 
computations.
\item
Section \ref{sec:object-model} will discuss how we annotate 
JavaScript objects with types using wrappers 
and offer the possibility to add custom types later on.
\item
The special role of functions and continuations and
how we model them as first-class values in Haskell and JavaScript
will be covered in Section \ref{sec:functions-continuations}. .
\item
The two threading models offered by Sunroof are explained 
in Section \ref{sec:threading-models}.
\item
Section \ref{sec:ffi} introduces Sunroof's foreign function interface.
\item
Translation of Sunroof to JavaScript is handled in 
Section \ref{sec:compiler}. We will explain the 
compilation of selected language constructs. This is 
especially interesting in the light of our use of continuations
and their translation to JavaScript.
\item
The ability to interleave Haskell and JavaScript computations as needed
through the Sunroof server will be highlighted in Section \ref{sec:server}.
\item
Section \ref{sec:extended-example} will cover a small application 
written in Sunroof to surveys how usable Sunroof is in the 
context of application development. 
%It shows how Sunroof can be used to utilize the 
%display capabilities of a browser, while still using Haskell when needed
%and where it is strong.
\end{itemize}

\begin{comment}
\subsubsection{RESOURCES - REMOVE WHEN FINISHED}

\begin{itemize}
\item Give an example of Sunroof 
\item Explain the example
\item Talk about the general structure of Sunroof
\item Relate each part of the structure to a specific section:
  \begin{itemize}
  \item \JSA\ /\ \JSB: \JS-monad in section \ref{sec:js-monad};
  together with \JSI\ and contiuations used by the \JS-monad.
  \item Type Wrappers / Expr: Discussed in section \ref{sec:object-model}
  \end{itemize}
\item Details about functions and continuations are given in section \ref{sec:functions-continuations}
\item 
Sunroof was first documented in our previous 
workshop paper~\cite{Farmer:12:WebDSLs},
where the possibility of monadic reification was observed.
In this paper, we raised an unresolved issue:

Threading Model: Built on top of \JS-monad (section \ref{sec:threading-models})
\item Explain how the compiler works and what it does \ref{sec:compiler}
\item In section \ref{sec:server} we will look at the sunroof server 
\item Extended example in section \ref{sec:extended-example}

\end{itemize}

We want to use the Javascript API. Fast interpreters,
running in a browser advantage, 

We do not want to program in Javascript.

\end{comment}


