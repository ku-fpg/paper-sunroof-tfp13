 
\section{Introduction}

% Simon: Describe the problem
There are many reasons to want to program in a functional language:
efficiency of development cost, informally reasoning, high-level 
control- and concurrency-structures, like monads~\cite{...}.
However, mainstream languages often have better environmental support
than what is provided by functional languages,
for example the Objective C and the iOS eco-structure, or JavaScript and HTML5
web browsers.
This paper examines the challenges of providing
an intentionally blurred interface between Haskell
and JavaScript, to support the development of web-based applications.

% Simon: State your contributions
JavaScript is an imperative language with access to a wide range
of established and useful services like graphical canvases and event
handling of browser events. 
We want to express JavaScript in Haskell, adding use
of Haskell's static typing, and gaining access to JavaScript services
in the browser directly in Haskell.

In general, there are a number of ways of providing access to non-native services,
such as the JavaScript canvas.
\begin{enumerate}
\item A first approach is to provide, in Haskell, 
foreign function ``hooks'' to key JavaScript functionality.
Haskell already provides many similar hooks into the RTS,
so why not into the JavaScript engine?
For this to work efficiently, the compiler
needs to target JavaScript. There are already a
number is systems doing this~\cite{...}.
If executed well, this would be ideal,
there are shortcomings: many standard libraries
are not supported directly, the generated
code is not as efficiently executed as native Haskell,
and the compilers are still immature and incomplete.

\item An second approach is to keep the 
foreign function ``hooks'' to key JavaScript functionality,
but instead run Haskell as a server that JavaScript
and the browser interacts with.
Unfortunately, every JavaScript call becomes an expensive proposition: an RPC call
to a browser.
Though some straight-line calls can be batched together --
our own blank-canvas hackage package~\cite{..} was built on this idea --
the granularity of interaction through JavaScript call is just too fine for
this idea to scale well.

\item A first, alternative approach is to use Haskell to generate JavaScript from
an abstract syntax tree of the JavaScript code. There have
been a number of attempt to have DSLs that do the in other spaces,
and functional languages have good support for generating trees.
However, this approach works well for data-flow, for example
as use in Lava. When describing control-flow,
as would be typically in a JavaScript program,  writing such an AST always
feels forced, primarily because there is a mismatch between
bindings in the native language (Haskell) and the non-native
generated code (JavaScript variable names).

\end{enumerate}

This paper investigate the expansion of scope of the second option,
specifically adding binding to the Haskell code
that can be batch together and send to the JavaScript engine,
and then scaling the language to support larger examples.
Binding is done using regular Haskell monadic binding, and fits naturally
into what we expect from a monadic API. Until recently,
it was thought that it was impossible to use a regular
Haskell monad for this purpose. In a previous paper,
we show that such a construction is possible~\cite{..}
In this paper, we expand on this observation,
and show that this form of reification is useful in practice.

Further building on this capability, we also investigate providing
JavaScript control flow and function abstraction mechanisms
to the Haskell programmer interested in using the browser API.
Though for technical reasons we can not compile the transitional
pattern matching and let-binding to JavaScript without committing
to a full compiler Haskell to JavaScript compiler, both
control flow and function abstraction can be provided
with a small syntactical overhead.

With these three capabilities, a programmer can start programming
using the provided JavaScript API directly, and refine
their program to migrate more and more computation
from the server into the browser. In a real sense,
this blurs the distention between an RPC approach (2),
and a full Haskell to JavaScript compiler (1),
built using existing infrastructure, and not needing
the full compiler.


