 
\section{Introduction}

Sunroofs was developed to support Haskell programmers 
when generating JavaScript. Lets look at the small example
in figure \ref{fig:code-example} to see how it achieves this goal.

\begin{figure}[h]
\vspace{-0.5cm}
\centering
\begin{subfigure}{0.45\textwidth}%
\begin{verbatim}
 jsCode :: JS t ()
 jsCode = do
   name <- prompt "Your name?"
   alert ("Your name: " <> name)
\end{verbatim}%
\end{subfigure}%
\hfill%
\begin{subfigure}{0.45\textwidth}
\vspace{0.25cm}%
\begin{verbatim}
  
  
var v0 = prompt("Your name?"); 
alert("Your name: " + v0);
\end{verbatim}%
\end{subfigure}% 
%\vspace{-0.2cm}%
\caption{Sunroof program and the expected JavaScript on the left.}%
\label{fig:code-example}%
\vspace{-0.5cm}
\end{figure}

We can see what code we expect the program to produce when 
translated to JavaScript on the right. But what makes 
this useful? The JavaScript we want to produce with the given
Haskell is shorter and can easily be written by hand.
So lets look at the advantages of this approach:
\begin{itemize}

\item
Sunroof introduces a threading model (Section \ref{sec:threading-models}) similar
to Haskell's. Like this programmers can reuse known abstractions
and have a known set of powerful tools.

\item 
It also introduces types that support us when 
writing code and reveal common mistakes. The functions
in figure \ref{fig:code-example} have toe following signature:
\begin{verbatim}
prompt :: JSString -> JSString -> JS t JSObject
alert  :: JSString -> JS t ()
\end{verbatim}
This can improve the quality of our code.
Section \ref{sec:object-model} will look into this in more detail.

\item 
Using a deep embedding like this gives opportunity 
for optimizations when producing the actual JavaScript.

\item
At the same time Sunroof offers an interface that is 
close to actual JavaScript making it easy to learn.
It utilizes the \Src{do}-notation (Section \ref{sec:js-monad}) and creates a 
simple foreign function interface to achieve this goal (TODO: SECTION?).
\end{itemize}

\Figure%
{fig:structure}%
{figures/sunroof-structure.pdf}%
{The structure of Sunroof.}
The scope of this paper is to show how Sunroof achieves these goals.
It will cover the most important parts of Sunroof and how they are 
implemented. Sunroofs structure can be seen in Figure \ref{fig:structure}.
In this paper we will:
\begin{itemize}
\item 
Go over the \JS-monad in Section \ref{sec:js-monad}. There
we will show how it is implemented and how we solved
the problem of restricting the types involved in the monadic 
computations.
\item
We will also discuss how we annotate the JavaScript
objects in Sunroof with types and also give the possibility 
to add custom types later on (Section \ref{sec:object-model}).
\item
Section \ref{sec:functions-continuations} will cover how
we model function and continuations in Sunroof and handle
them as first-class values in Sunroof and JavaScript.
\item
How we implemented the two threading models offered by Sunroof is explained 
in Section \ref{sec:threading-models}.
\item
The actual translation of Sunroof to JavaScript is handled in 
Section \ref{sec:compiler}. We will take a close look to the
different layers of abstractions and in detail explain the 
translation of selected language constructs. This is 
especially interesting in the light of continuations, which
are directly translated into JavaScript.
\item
As Sunroof is quite useful in combination with a server,
Section \ref{sec:server} will explain how we can utilize it 
to interleave Haskell and JavaScript computations as needed.
\item
TODO: Extended Example
\end{itemize}



\begin{comment}
Figure \ref{fig:structure} shows how Sunroof is structured.



An essential part of most modern web applications is Javascript.
It opens various possibilities to create browser based applications.
Examples for this development are Googlemail or Facebook.
These applications rely on Javascript heavily. What makes this 
development possible are the fast interpreters available in all
major browsers. Developing applications inside the browser has
the advantage of being independent from the underlying hard- and
software platform. The problem is shifted from the actual machine
to the browser. Also modern web technologies like HTML5 and CSS
give easy to use tools for developing user interfaces for the
mentioned application. They are standardized and improved over years
up to now. 

Yet, writing actual code in Javascript can be tedious and error prone 
due to its not-statically typed nature. Also the object-oriented
paradigm and the callback mechanism used instead of proper multi-threading 
can be a stumbling block for veteran Haskell programmers. Especially when
facing the lack of type safety.

To make the power of Javascript available to Haskell programmers 
we developed Sunroof. It enables Haskell programmers to write
Javascript using Haskell through a deep embedding. This 
approach has several advantages:

\begin{description}
\item[Type Safety:] Sunroof provides custom types for all major Javascript
types. Thus it gives type safety similar to what we are used to from 
normal Haskell programs when writing Javascript.
\item[Paradigms \& Patterns:] As a Haskell EDSL Sunroof follows functional programming 
paradigms and principals. It is possible to reuse common combinators and 
patterns in Sunroof.
\item[Similarity:] At the same time Sunroof offers a API for Javascript 
that allows to translate Javascript into Sunroof code easily.
\item[Optimization:] Due to Sunroofs deep embedded nature it is able
to apply optimizations like expression sharing when compiling to Javascript.
\end{description}

\NOTE{This might be a bit to enthusiastic.}

\end{comment}

\subsubsection{RESOURCES - REMOVE WHEN FINISHED}

\begin{itemize}
\item Give an example of Sunroof 
\item Explain the example
\item Talk about the general structure of Sunroof
\item Relate each part of the structure to a specific section:
  \begin{itemize}
  \item \JSA\ /\ \JSB: \JS-monad in section \ref{sec:js-monad};
  together with \JSI\ and contiuations used by the \JS-monad.
  \item Type Wrappers / Expr: Discussed in section \ref{sec:object-model}
  \end{itemize}
\item Details about functions and continuations are given in section \ref{sec:functions-continuations}
\item 
Sunroof was first documented in our previous 
workshop paper~\cite{Farmer:12:WebDSLs},
where the possibility of monadic reification was observed.
In this paper, we raised an unresolved issue:

Threading Model: Built on top of \JS-monad (section \ref{sec:threading-models})
\item Explain how the compiler works and what it does \ref{sec:compiler}
\item In section \ref{sec:server} we will look at the sunroof server 
\item Extended example in section \ref{sec:extended-example}
\end{itemize}

JavaScript is an imperative language with access to a wide range
of established and useful services, like graphical canvases and event
handling. JavaScript as a language also provides features that are
traditionally associated with functional languages, like first-class 
functions. We want to express JavaScript in Haskell, adding use
of Haskell's static typing, and gaining access to JavaScript services.
And we do so using the transitional functional programming 
mechanism for being imperative, a monad~\cite{Moggi:91:ComputationMonads}.


We want to use the Javascript API. Fast interpreters,
running in a browser advantage, 

We do not want to program in Javascript.

We start with an overview of the language, then discuss
how the language is implemented, and discuss interesting
techniques used inside Sunroof. We close with a more
complete web application.




