 
\section{The JavaScript Monad}
\label{sec:js-monad}

The \JS-monad is
used to model sequences of JavaScript statements and
their side-effects. In that sense it is almost
exact analog for the Haskell \IO-monad, except there
is an extra phantom argument~\cite{Leijen:99:Phantom} 
that decides which threading model to use, as
discussed in Section \ref{sec:threading-models}.
For now we can ignore this extra argument.

The basic idea is that each binding becomes an
assignment to a fresh variable in JavaScript. Like
this the results of previous computations are passed on to 
later ones.
An example for this can be seen in Figure~\ref{fig:code-example}.
The binding \Src{name} is translated to the freshly generated
variable \Src{v0}.

Inside this simple example is a challenging problem -- where does
\Src{v0} come from? The bind inside the monadic \Src{do} is
unconstrained:
\begin{verbatim}
(>>=) :: JS t a -> (a -> JS t b) -> JS t b
\end{verbatim}
What we want is:
\begin{verbatim}
(>>=) :: (Sunroof a) => JS t a -> (a -> JS t b) -> JS t b
\end{verbatim}
Where \Src{Sunroof} constrains the bind to
arguments for which we can generate a JavaScript variable.
Counterintuitively, 
it turns out that a specific form of normalization allows 
the ``\Src{a}'' type to be constrained and the bind to 
be an instance of the standard monad class~\cite{Sculthorpe:13:ConstrainedMonads}.
Through this keyhole of {\em monadic reification},
the entire Sunroof language is realized. The 
normalization is done through the Operational package 
\cite{Apfelmus:10:Operational,Hackage:10:Operational}.

It provides the \Src{Program} data type as a monad that
can be equipped with custom primitives or effects.
We represent these primitives with the \JSI~instruction 
(Figure \ref{fig:jsi-definition}) type. It represents the 
abstract JavaScript instructions that are sequenced by the \Src{Program}
monad.
\begin{figure}
\begin{verbatim}
data JSI :: T -> * -> * where
  JS_Invoke   :: (SunroofArgument a, Sunroof r) 
              => a -> JSFunction a r -> JSI t r
  JS_Function :: (SunroofArgument a, Sunroof b) 
              => (a -> JS A b) -> JSI t (JSFunction a b)
  JS_Branch   :: (SunroofThread t, SunroofArgument a, Sunroof bool) 
              => bool -> JS t a -> JS t a  -> JSI t a
  JS_Assign   :: (Sunroof a) 
              => JSSelector a -> a -> JSObject -> JSI t ()
  ...
\end{verbatim}
\caption{Parts of the \JSI nstruction data type.}
\label{fig:jsi-definition}
\end{figure}
The type parameter \Src{t} in \Src{JSI t a} again represents 
the threading model and can be ignored up to Section \ref{sec:threading-models}. 
\Src{a} represents the type of the primitive's return value. 
Figure \ref{fig:jsi-definition} shows some of the instructions
in \JSI.
\Src{JS\_Invoke} calls a function, that has been created with \Src{JS\_Function}.
Branches are represented with \Src{JS\_Branch}. Assignments to variable
are represented by \Src{JS\_Assign\_}.

If it was not for our promise to provide an alternative 
threading model, the normalization through the Operational package
would be enough to offer a monad suitable for Sunroof.
But, because of our threading plans, there is more then just 
normalization going on behind the scenes. 
Actually the \JS-monad is a continuation monad over the 
\Src{Program}-monad.
\begin{verbatim}
data JS :: T -> * -> * where
  JS :: ((a -> Program (JSI t) ()) -> Program (JSI t) ()) -> JS t a
  ...
\end{verbatim}
The monad instance used is the standard implementation of 
a continuation monad (TODO: cite).


TODO: Maybe put this section after the section about the object model.

\begin{comment}
\subsubsection{RESOURCES - REMOVE WHEN FINISHED}

Basic Idea
\begin{itemize}
\item Put the current stuff into the introduction does not belong here
\item \JS-monad is analog of \IO-monad
\item Monad used to model sequences of statements with side-effect in 
JavaScript \cite{Moggi:91:ComputationMonads}
\item binding in Haskell becomes binding in JavaScript (more about this in section \ref{sec:compiler})
\item There also is a expression level (discussed in section \ref{sec:object-model})
\item Feels like a native monad, cf STM.
\item Can build abstractions on top of this.
\end{itemize}
Implementation
\begin{itemize}
\item Problems when using monad: Types need to be constrained to work
\item We use monad reification \cite{Apfelmus:10:Operational,Hackage:10:Operational}
\item But \JS-monad is a little bit more complicated
\item Show \JS-constructor: \Src{JS :: ((a -> Program (JSI t) ()) -> Program (JSI t) ()) -> JS t a}
\item Whats going on here?
\item \JS-monad is a continuation monad on the \Src{Program} type from operational
\item Monad instance is the standard implementation for a continuation 
monad (TODO: cite for continuation monad)
\item This is necessary, because we want to support different threading styles 
(discussed in chapter \ref{sec:threading-models})
\item For now think of it as producing a sequence/list of abstract instructions: \JSI
\item Look at the \JSI\ type and introduce it
\item Explain how it is a high-level representation of JavaScript
\item TODO: Look at all constructors? Or only the interesting ones (marked with "!")?
  \begin{itemize}
  \item ! Function and continuation are technically the same, but different instructions
  \item ! Reason for this is TODO?
  \item ! Why are there two different assignments?
  \item ! One is assignment to variable, other is assignment to the field of an object
  \item ! Eval, this looks weird. Forces evaluation and binding to a variable.
  \item ! Otherwise, we would use macro semantics in many places.
  \item ! Fix is provided to be able to write recursive functions
  \item ! Why isn't that possible without fix? problem boils down to 
        not observable sharing between statements and bindings. This
        leads compiler into infinite loops.
  \item Select selects a field
  \item Delete remove a field
  \item Invoke call a function/method
  \item Return for return of a function
  \item Comment writes some comment, helps for debugging
  \end{itemize}

\end{itemize}
\end{comment}



