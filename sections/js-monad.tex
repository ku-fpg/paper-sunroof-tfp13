 
\section{The JavaScript Monad}
\label{sec:js-monad}

The imperative nature and side-effects of JavaScript are 
modeled through the \JS-monad. It resembles the \IO-monad, 
but has an extra phantom argument~\cite{Leijen:99:Phantom} 
to decide which threading model is used. 
For now we can ignore this extra argument. It will be
discussed in \SecRef{sec:threading-models}.

The basic idea is that a binding in the \JS-monad becomes an
assignment to a fresh variable in JavaScript. 
This allows the results of previous computations to be passed on to 
later ones.
\FigRef{fig:code-example} gives an example.
The binding \Src{name} is translated to the freshly generated
variable \Src{v0}.

This simple example displays a challenging problem. Where does
\Src{v0} come from? The bind inside the monadic \Src{do} is
fully polymorphic over \Src{a}.
\begin{Code}
(>>=) :: JS t a -> (a -> JS t b) -> JS t b
\end{Code}
Thus we do not know how to create a value of type \Src{a}. What we want is:
\begin{Code}
(>>=) :: (Sunroof a) => JS t a -> (a -> JS t b) -> JS t b
\end{Code}
where \Src{Sunroof} constrains the bind to
arguments for which we can generate a variable.
It turns out that a specific form of normalization allows 
the type \Src{a} to be constrained and \JS~to 
be an instance of the standard monad class~\cite{Sculthorpe:13:ConstrainedMonads}.
Through this keyhole of {\em monadic reification},
the entire Sunroof language is realized. 

The normalization is done through Operational 
\cite{Apfelmus:10:Operational,Hackage:10:Operational}.
It provides the \Src{Program}-monad that
can be equipped with custom primitives.
We represent these primitives with the JavaScript instruction
(\JSI)
type shown in \FigRef{fig:jsi-definition}. It represents the 
abstract instructions that are sequenced inside 
the \Src{Program}-monad.
\begin{figure}[t]
\begin{Code}
data JSI :: T -> * -> * where
  JS_Invoke   :: (SunroofArgument a, Sunroof r) 
              => a -> JSFunction a r -> JSI t r
  JS_Function :: (SunroofArgument a, Sunroof b) 
              => (a -> JS A b) -> JSI t (JSFunction a b)
  JS_Branch   :: (SunroofThread t, SunroofArgument a, Sunroof bool) 
              => bool -> JS t a -> JS t a  -> JSI t a
  JS_Assign_  :: (Sunroof a) => Id -> a -> JSI t ()
  ...
\end{Code}
\caption{Parts of the JavaScript instruction data type (\JSI).}
\label{fig:jsi-definition}
\end{figure}
The parameter \Src{t} in \Src{JSI t a} again represents 
the threading model and can be ignored up to \SecRef{sec:threading-models}. 
The type \Src{a} represents the primitive's return value. 
\Src{JS\_Invoke} calls a function that has been created with \Src{JS\_Function}.
Branches are represented with \Src{JS\_Branch}. Assignments to a variable
are represented by \Src{JS\_Assign\_}.

Without the alternative threading model, 
the normalization through Operational
would be enough to offer a monad suitable for Sunroof.
Due to our threading plans, there is more than just 
normalization going on behind the scenes. 
The \JS-monad is a continuation monad over the 
\Src{Program}-monad.
\begin{Code}
data JS :: T -> * -> * where
  JS :: ((a -> Program (JSI t) ()) -> Program (JSI t) ()) -> JS t a
  ...
\end{Code}
The monad instance used is the standard implementation of 
a continuation monad.
%\cite{Wadler:90:ComprehendingMonads,Wadler:94:MonadsComposableContinuations}.
%\TODO{Both of these cites do not seem right, but they were the best I could find.
%I can't find the paper from Oleg and Simon you talked about}

\begin{comment}
\subsubsection{RESOURCES - REMOVE WHEN FINISHED}

Basic Idea
\begin{itemize}
\item Put the current stuff into the introduction does not belong here
\item \JS-monad is analog of \IO-monad
\item Monad used to model sequences of statements with side-effect in 
JavaScript \cite{Moggi:91:ComputationMonads}
\item binding in Haskell becomes binding in JavaScript (more about this in \SecRef{sec:compiler})
\item There also is a expression level (discussed in \SecRef{sec:object-model})
\item Feels like a native monad, cf STM.
\item Can build abstractions on top of this.
\end{itemize}
Implementation
\begin{itemize}
\item Problems when using monad: Types need to be constrained to work
\item We use monad reification \cite{Apfelmus:10:Operational,Hackage:10:Operational}
\item But \JS-monad is a little bit more complicated
\item Show \JS-constructor: \Src{JS :: ((a -> Program (JSI t) ()) -> Program (JSI t) ()) -> JS t a}
\item Whats going on here?
\item \JS-monad is a continuation monad on the \Src{Program} type from operational
\item Monad instance is the standard implementation for a continuation 
monad
\item This is necessary, because we want to support different threading styles 
(discussed in \SecRef{sec:threading-models})
\item For now think of it as producing a sequence/list of abstract instructions: \JSI
\item Look at the \JSI\ type and introduce it
\item Explain how it is a high-level representation of JavaScript
\item \TODO{Look at all constructors? Or only the interesting ones (marked with "!")?}
  \begin{itemize}
  \item ! Function and continuation are technically the same, but different instructions
  \item ! Reason for this is TODO?
  \item ! Why are there two different assignments?
  \item ! One is assignment to variable, other is assignment to the field of an object
  \item ! Eval, this looks weird. Forces evaluation and binding to a variable.
  \item ! Otherwise, we would use macro semantics in many places.
  \item ! Fix is provided to be able to write recursive functions
  \item ! Why isn't that possible without fix? problem boils down to 
        not observable sharing between statements and bindings. This
        leads compiler into infinite loops.
  \item Select selects a field
  \item Delete remove a field
  \item Invoke call a function/method
  \item Return for return of a function
  \item Comment writes some comment, helps for debugging
  \end{itemize}

\end{itemize}
\end{comment}



