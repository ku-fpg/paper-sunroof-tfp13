 
\section{The JavaScript Monad}
\label{sec:js-monad}

JavaScript is an imperative language with access to a wide range
of established and useful services, like graphical canvases and event
handling. JavaScript as a language also provides features that are
traditionally associated with functional languages, like first-class 
functions. We want to express JavaScript in Haskell, adding use
of Haskell's static typing, and gaining access to JavaScript services.
And we do so using the transitional functional programming 
mechanism for being imperative, a monad~\cite{Moggi:91:ComputationMonads}.

The \JS-monad is the monad of JavaScript effects, as is an almost
exact analog for the Haskell \IO-monad, except there
is an extra phantom argument~\cite{Leijen:99:Phantom} that we will return
to shortly.  Figure~\ref{fig:code-example} gives a first example
of the \JS-monad in use with the generated JavaScript.

Inside this simple example is a challenging problem -- where does
\Src{v0} come from? The bind inside the monadic \Src{do} is
unconstrained:
\begin{verbatim}
(>>=) :: JS t a -> (a -> JS t b) -> JS t b
\end{verbatim}
What we want is:
\begin{verbatim}
(>>=) :: (Sunroof a) => JS t a -> (a -> JS t b) -> JS t b
\end{verbatim}
Where \Src{Sunroof} constrains the bind to
arguments for which we can generate a JavaScript variable.
Counterintuitively, 
it turns out that a specific form of normalization allows 
the ``\Src{a}'' type to be constrained and the bind to 
be an instance of the standard monad class~\cite{Sculthorpe:13:ConstrainedMonads}.
Through this keyhole of {\em monadic reification\/},
the entire Sunroof language is realized.

\subsubsection{RESOURCES - REMOVE WHEN FINISHED}

We use monad reification.
\begin{itemize}
\item binding in Haskell becomes binding in reified language.
\item Feels like a native monad, cf STM.
\item Can build abstractions on top of this.
\end{itemize}




