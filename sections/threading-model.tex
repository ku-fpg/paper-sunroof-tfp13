 
\section{Threading Models}
\label{sec:threading-models}

Sunroof was first documented in our previous 
workshop paper~\cite{Farmer:12:WebDSLs},
where the possibility of monadic reification was observed.
In this paper, we raised an unresolved issue: do you
generate atomic JavaScript code, and keep the callback
centric model of computation, or generate JavaScript
using CPS, and allow for blocking primitives,
like Haskell \Src{MVar}s. The latter, though more powerful, 
precluded using the compiler to generate
code that can be cleanly called from native JavaScript.
Both choices had poor consequences.

So, rather than pick one, we decided to explicitly support both,
and make both first class threading strategies in our compiler.
In terms of user-interface, we parameter the \JS-monad
with a phantom type that represents the threading model
to compile with, with \Src{A} for \Src{A}tomic threads,
and \Src{B} for \Src{B}locking (cooperative concurrency) threads. 
Atomic threads are classical JavaScript threads, and
are never interrupted; while blocking threads can
support suspending operations. By using phantom
types, we can express the necessary
restrictions on specific combinators, as well
as provide combinators to allow both types of
threads to cooperate successfully.

\subsubsection{RESOURCES - REMOVE WHEN FINISHED}


