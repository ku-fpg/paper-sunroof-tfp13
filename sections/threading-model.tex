 
\section{Threading Models}
\label{sec:threading-models}

JavaScript uses a callback centric model of computation. There
is no concurrency, only a central loop that executes callbacks
when events occur.

In contrast, Haskell has real concurrency and wide-spread 
abstractions for synchronization, e.g. \Src{MVar}s and \Src{Chan}s
\cite{Jones:96:ConcurrentHaskell}.
So the question arises: do we generate atomic JavaScript code, 
and keep the callback centric model, or generate JavaScript
using CPS \cite{Claessen:99:PoorMansConcurrencyMonad}, 
and allow for blocking primitives and
cooperative concurrency. The latter, though more powerful, 
precluded using the compiler to generate
code that can be cleanly called from native JavaScript.
Both choices had poor consequences.

We decided to explicitly support both,
and make both first-class threading {\em strategies\/} in Sunroof.
In terms of user-interface, we parameterize the \JS-monad
with a phantom type that represents the threading model used, 
with \Src{A} for \Src{A}tomic,
and \Src{B} for \Src{B}locking threads. 
Atomic threads are classical JavaScript computations that
cannot be interrupted and actively use the callback
mechanism. Blocking threads can
support suspending operations and cooperative concurrency
abstractions as known from Haskell. By using phantom
types, we can express the necessary
restrictions on specific combinators, as well
as provide combinators to allow both types of
threads to cooperate.

The blocking model hides the callback mechanism behind abstractions.
This implies that every atomic computation can be converted into 
a blocking computation. \Src{liftJS} achieves this.
\begin{Code}
liftJS :: Sunroof a => JS A a -> JS t a
\end{Code}

When suspending, we register our current
continuation as a callback to resume later. This gives other 
threads (registered continuations) a chance to run.
Of course, this model depends on cooperation between the threads,
because a not terminating or suspending thread will keep others from running.

There are three main primitives for the blocking model:
\begin{Code}
forkJS      :: SunroofThread t1 => JS t1 () -> JS t2 ()
threadDelay :: JSNumber -> JS B ()
yield       :: JS B ()
\end{Code}
They can all be seen as analogues of their \IO~counterparts.
\Src{forkJS} resembles \Src{forkIO}.
It registers the given computation as a callback. 
\Src{yield} suspends the current thread by 
registering the current continuation as a callback,
giving other threads time to run.
\Src{threadDelay} is a form of \Src{yield} that sets 
the callback to be called after a certain amount of time.
We rely on the JavaScript function \Src{window.setTimeout} 
\cite{whatwg:timers} to register our callbacks.

The class \Src{SunroofThread} offers functions to retrieve the 
current threading model (\Src{evalStyle}) and to create a possible
blocking computation (\Src{blockableJS}).
\begin{Code}
class SunroofThread (t :: T) where
  evalStyle   :: ThreadProxy t -> T
  blockableJS :: (Sunroof a) => JS t a -> JS B a
\end{Code}
Based on these primitive combinators, we also offer a Sunroof 
version of \Src{MVar} and \Src{Chan}: \Src{JSMVar} and \Src{JSChan}.
\begin{Code}
newMVar      :: (SunroofArgument a) => a -> JS t (JSMVar a)
newEmptyMVar :: (SunroofArgument a) => JS t (JSMVar a)
putMVar      :: (SunroofArgument a) => a -> JSMVar a -> JS B ()
takeMVar     :: (SunroofArgument a) => JSMVar a -> JS B a

newChan   :: (SunroofArgument a) => JS t (JSChan a)
writeChan :: (SunroofArgument a) => a -> JSChan a -> JS t ()
readChan  :: (SunroofArgument a) => JSChan a -> JS B a
\end{Code}
Note that the types reflect if a specific operation can block.
For example, \Src{writeChan} can never block, so can use either threading model,
but \Src{readChan} may block, so uses the \Src{B} threading model.

%Both implementations use JavaScript arrays to store the waiting readers and
%writers in the form of continuation objects.
%Because of this, all functions
%are able to handle \Src{SunroofArgument}s, not just \Src{Sunroof}
%types. 
%When arguments are written, either the waiting
%continuation is called with those arguments or a new continuation 
%that applies an incoming continuation with those arguments is created.

\begin{comment}
\subsubsection{RESOURCES - REMOVE WHEN FINISHED}

\begin{itemize}
\item JavaScript does not have a threading model.
\item There is only one thread and callbacks can be registered to be
called when events happen once thread of execution has ended.
\item Using our CPS in the \JS-monad and translating it directly 
to JavaScript we can simulate/emulate a threading model similar to
Haskell's.
\item Basic idea is to use callbacks as mechanism to continue
suspended computations (in form of continuations).
\item \TODO{Is there a paper about this technique?}
\item We support both styles.
\item Decision which one to use is made by the first 
type parameter of the \JS-monad. A - atomic / B - blocking
\item Compiler comes in these two flavors: \Src{sunroofCompileJSA} and \Src{sunroofCompileJSB}
\end{itemize}
Concurrency primitives
\begin{itemize}
\item A primitives for the blocking threading model we provide:
\Src{forkJS}, \Src{threadDelay} and \Src{yield}
\item Show type signatures for the three methods
\item \Src{forkJS} can be thought of an equivalent to \Src{forkIO},
it registers a callback for the given computation, so 
it is done once the current computation is done.
\item \Src{threadDelay} registers a callback to execute 
the rest of this computation (the rest of this continuation)
as soon as the given amount of time has passed. This thread 
of computation ends here.
\item \Src{yield} sets the timeout of \Src{threadDelay} to zero.
\end{itemize}
Concurrency abstractions
\begin{itemize}
\item As a further abstraction we also provide \Src{JSMVar} and \Src{JSChan}.
\item These are equivalents of \Src{MVar} and \Src{Chan} in Sunroof.
\item Both are expressed purely in terms of the above primitives 
and continuations.
\item Both utilize two lists to store the processes that 
are waiting for data and those that are trying to write data.
and register them to be run through \Src{forkIO} and \Src{goto}
as needed.
\item Show interface for both types
\item Remark how they enforce \JSB\ to ensure that 
CPS is translated down to JavaScript
\item Spare the implementation as it is not interesting in the 
scope of this paper.
\item \TODO{Reference MVar and Chan papers / 
are there parallels to their implementation?}
\item \TODO{Cite: Continuations to store computations?}
\end{itemize}
\end{comment}





