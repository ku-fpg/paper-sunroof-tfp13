 
\section{Related Work}

There have been several attempts to translate Haskell to JavaScript.
Prominent ones are the compiler backends for 
UHC \cite{Stutterheim:12:ImprovingUHCJavaScriptBackend} and 
GHCJS \cite{project:ghcjs}. There are also projects like Fay \cite{project:fay} 
that compile subsets of Haskell to JavaScript or JMacro \cite{project:jmacro}
which use quasiquotation \cite{Mainland:07:QuasiquotingHaskell} to embed 
a custom-tailored language into Haskell code.

At the same time there are also projects like 
CoffeeScript \cite{project:coffeescript} or LiveScript \cite{project:livescript}
to build custom languages 
that are very similar to JavaScript but add convenient syntax and
support for missing features.

Our approach to cooperative concurrency through continuations in JavaScript has
has been used before 
\cite{Cooper:07:LinksWebProgrammingTiers,Predescu:02:CocoonContinuationBasedControlFlow}.
To our knowledge, creating a direct connection
between Haskell and JavaScript continuations has not been 
attempted before.

Deep embeddings of monads based on data structures have been used before
in Unimo \cite{Lin:06:Unimo} and Operational \cite{Apfelmus:10:Operational,Hackage:10:Operational}. 
The specific approach Sunroof takes 
by using GADTs has been discussed by 
Sculthorpe et al. \cite{Sculthorpe:13:ConstrainedMonads} 
in detail.

The Sunroof server does not have the aim to provide a full-featured 
web framework, as HAppS, Snap or Yesod do. It only provides 
the infrastructure to communicate with the currently calling website
through the Kansas comet \cite{project:kansas-comet} 
push mechanism \cite{pattern:push}. Although all of the
frameworks mentioned above would be able to implement this technique,
to our knowledge, none of them has yet.

To our knowledge, Sunroof is the only library that supports 
generation of JavaScript inside of Haskell using pure Haskell
in a type-safe manner. All other approaches discussed above
either require a separate compilation step or introduce new 
syntax inside of Haskell.

There is an effort to generalize Active \cite{project:active}, a library for animations, and
implement a backend based on Sunroof \cite{project:sunroof-active}.



