 
\section{Foreign Function Interface}
\label{sec:ffi}

Sunroof also offers a 
foreign function interface, which enables us to easily 
access predefined JavaScript. There are three core functions.
\begin{Code}
fun    :: (SunroofArgument a, Sunroof r) 
       => String -> JSFunction a r
object :: String -> JSObject
new    :: (SunroofArgument a) 
       => String a -> JS t JSObject
invoke :: (SunroofArgument a, Sunroof o, Sunroof r) 
       => String -> a -> o -> JS t r
\end{Code}
\Src{fun} is used to create Sunroof functions from their names in JavaScript.
This can happen in two ways. Either to call a function in line or to 
create a real binding for that function. As an example 
the \Src{alert} function can be called in line through \Src{fun "alert" \$\$ "text"}
or you can provide a binding in form of a Haskell function for it.
\begin{Code}
alert :: JSString -> JS t ()
alert s = fun "alert" $$ s
\end{Code}

Objects can be bound through the \Src{object} function, e.g.
the \Src{document} object is bound through \Src{object "document"}.
Constructors can be called using \Src{new}. To create a new
object you would call \Src{new "Object" ()}.

We can call methods of objects through \Src{invoke}. Again, this 
can be used in line and to create a real binding. A inline 
use of this to produce \Src{document{\linebreak}.getElementById("id")} would look like this: 
\begin{Code}
object "document" # invoke "getElementById" "id"
\end{Code}
Where \Src{\#} is just a flipped function application. To provide a binding 
to the \Src{getElementById} method one can write:
\begin{Code}
getElementById :: JSString -> JSObject -> JS t JSObject
getElementById s = invoke "getElementById" s
\end{Code}

Providing actual bindings ensures that
everything is typed correctly and prevents resolving ambiguities 
through large type annotations inside of code.

The current release of Sunroof already provides bindings for most of the 
core browser API, the HTML5 canvas element and some of the JQuery API.

\begin{comment}
Table \ref{tab:ffi} gives an overview of
how Sunroof's FFI can be used. 
\begin{table}
\begin{center}
\begin{tabular}{l@{\quad}p{7cm}}
  \hline\rule{0pt}{12pt}%
  JavaScript & 
  Sunroof \\ \hline\rule{0pt}{12pt}%
%
  \Src{alert("Test");} & 
  \Src{fun "alert" \$\$ "Test"} \\[2pt]
%
  \Src{alert} as a Sunroof function & 
  \Src{alert :: JSFunction JSString ()\newline alert = fun "alert"} \\[2pt]
%
  \Src{alert} as a Haskell function & 
  \Src{alert :: JSString -> JS t ()\newline alert s = fun "alert" \$\$ s} \\[2pt]
%
  \Src{document.getElementById("id");} & 
  \Src{object "document" \# invoke "getElementById" "id"} \\[2pt]
%
  \Src{getElementById} as method & 
  \Src{getElementById :: JSString -> JSObject -> JS t JSObject\newline
       getElementById s = invoke "getElementById" s} \\[2pt]
\hline
\end{tabular}
\end{center}
\caption{JavaScript expressed through the Sunroof FFI.}
\label{tab:ffi}
\end{table} 
\end{comment}