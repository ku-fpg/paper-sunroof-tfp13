\section{Calling JavaScript from Haskell}
\label{sec:js-rpc}

From a programmers' point of view, calling JavaScript functions
appears straightforward. We, as a community know how to reflect an
API into Haskell, using the IO monad. Furthermore, objects in
the target API become handles in Haskell. 

As a first example, consider this simplified example of Sunroof code, and corresponding JavaScript.
\noindent
\begin{Code}
-- Haskell                          // JavaScript
ioCode :: IO ()
ioCode = send jsCode

jsCode :: JS ()
jsCode = do                        function jsCode() {
   name <- prompt "Your name?"       var v0 = prompt("Your name?"); 
   alert ("Your name: " <> name)     alert("Your name: " + v0); 
                                   }
\end{Code}%     
%\noindent 

Here, we use a new monad, the \Src{JS} monad, our JavaScript
analog to the \Src{IO} monad,
and an explicit \Src{send} command that sends the JavaScript to the browser.
This reversal of control, where the
server sends the clients commands is called the Ajax Comet~\cite{..},
or simply long polling.
This interface also bundles the \Src{prompt} and \Src{alert} commands
into one interaction transaction.
It is this flavor of interface we want to support in our Sunroof compiler
and web server.  

To make Sunroof a viable interface to JavaScript, we need to
resolve the following issues:
\begin{itemize}
\item JavaScript is an object-based, imperative, dynamically typed language.
Haskell is a pure, function-based, statically typed language.
Specifically how do these two worlds map onto each other.
In section~\ref{sec:object-model}, we discuss the chosen object model
and our JS monad.
%
\item We need to provide an imperative and therefor effectful interface,
including control flow, into our target language of JavaScript.
We do this via the \Src{JS} monad, which we discuss
in section~\ref{sec:js-monad}.
\item We choice to provide a way of defining functions
in Sunroof in a way that they are first-class functions
in JavaScript. This uses our \Src{JS} monad, and
is discussed in section~\ref{sec:functions-continuations}.
%
\item We need to provide a foreign function interface,
to allow us to call specific JavaScript-native functions,
like \Src{prompt} and \Src{alert}.
In section~\ref{sec:ffi} we present this interface.
%
\item Critically, we need to be able to compile our Sunroof DSL
into JavaScript. We look at this in section~\ref{sec:compiler}.
\item To enable the compiled code to dynamically interact with
a web browser, we provide an expansion of the \Src{send} idea above,
which we discuss in section~\ref{sec:server}.
\end{itemize}

