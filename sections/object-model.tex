 
\section{JavaScript Object Model}
\label{sec:object-model}

In our JavaScript object model, there are a family of core types, all prefixed with JS.
Each one corresponds to a specific JavaScript type.
There is support for booleans, strings, numbers, functions,
object and arrays, and other common programming structures.
The use of \Src{JSNumber}, rather than (say) \Src{Double},
explicitly reminds us of the enforced diminished capabilities of
being within an embedded language.
All of the \Src{JS}-types are representable in our target language.

Using the \Src{JS}-types, we can give types to \Src{alert} and \Src{prompt}.
\begin{Code}
alert :: JSString -> JS ()
prompt :: JSString -> JS String
\end{Code}
In reality, JavaScript overloads \Src{alert} and \Src{prompt},
so this is providing type-safe access to a sub-set of what
alert can do.

There are two primary Haskell classes represent things that can
be passed to and returned from JavaScript functions.
Instances of the \Src{Sunroof} class 
are values that are returned from JavaScript functions,
and represent all basic JavaScript values.
Instances of the \Src{SunroofArgument} class
are values that be passed to JavaScript functions,
including tuples of values which are used for representing multi-argument 
function calls.
The asymmetry here is a reflection of the JavaScript
asymmetry inherited from C: you can pass multiple arguments
to a function, but only get a single thing back.
Finally, there is a third class, \Src{SunroofKey}, which is
is a JavaScript version of the Haskell \Src{Show} class,
but specifically for generating JavaScript object keys.



\begin{table}[t]
\begin{center}
\begin{tabular}{r@{\quad}l@{\quad}l@{\quad}c}
\hline\rule{0pt}{12pt}%
  Constraint
  & Sunroof Type $\tau$
  & Haskell Analog \HaskellAnalog{$\tau$}
  & \Src{js} \\ \hline\rule{0pt}{12pt}%
  
  & \Src{()}       & \Src{()}     & $\checkmark$ \\
  & \Src{JSBool}   & \Src{Bool}   & $\checkmark$ \\
  & \Src{JSNumber} & \Src{Double} & $\checkmark$ \\
  & \Src{JSString} & \Src{String} & $\checkmark$ \\
  
  \Src{Sunroof $\alpha$}
  & \Src{JSArray $\alpha$} 
  & \Src{[$\HaskellAnalog{\alpha}$]}
  & \\
  
  \Src{SunroofKey $\alpha$}
  & \Src{JSMap $\alpha$ $\beta$}
  & \Src{Map $\HaskellAnalog{\alpha}$ $\HaskellAnalog{\beta}$}
  & \\
  \Src{Sunroof $\beta$} \\
  
  \Src{SunroofArgument $\alpha$}
  & \Src{JSFunction $\alpha$ $\beta$ }
  & \Src{$\HaskellAnalog{\alpha}$ $\rightarrow$ JS$_\Src{A}$ $\HaskellAnalog{\beta}$} 
  & \\
  \Src{Sunroof $\beta$} \\
  
  \Src{SunroofArgument $\alpha$}
  & \Src{JSMVar $\alpha$}
  & \Src{MVar $\HaskellAnalog{\alpha}$}
  & \\
  
  \Src{SunroofArgument $\alpha$}
  & \Src{JSChan $\alpha$}
  & \Src{Chan $\HaskellAnalog{\alpha}$}
  & \\[2pt]
\hline
\end{tabular}
\end{center}
\caption{Sunroof types and their Haskell pendant.}
\label{tab:sunroof-types}
\end{table} 


Table~\ref{tab:sunroof-types} enumerates major \Src{JS} types,
and any restrictions on the type arguments enforced by containers,
like \Src{JSArray}. What can be seen from this is that we
enforce a Hindley–Milner style thinking to our containers,
which is distinct from JavaScript dynamic typing.
Some types involve 
phantom types to give more type safety \cite{Cheney:03:FirstClassPhantomTypes}.
The smooth embedding of booleans and numbers is done through
the Boolean package \cite{project:boolean}.

One design decision in Sunroof is that we enforce a stronger typing
than JavaScript itself would, but also provide an explicit \Src{cast},
for use where the type-systems differ.
\begin{Code}
cast :: (Sunroof a, Sunroof b) => a -> b
\end{Code}
From experience with using Sunroof,
the mis-match in typing between Haskell and Sunroof/JavaScript
is not large problem in practice,
many programs translate from Sunroof to JavaScript
maintaining the stronger typing.
Casts are typically used in the same way \Src{show} is used
to map numbers to strings in the context of building
string values that contain numbers. We also provide
some variants of \Src{cast} with more specific types
to help alleviate some of the dynamic typing.

\TabRef{tab:sunroof-types} shows that most basic 
Haskell types have counterparts in
Sunroof. To convert Haskell values into their 
counterparts, we provide the \Src{SunroofValue} class.
\begin{Code}
class SunroofValue a where
  type ValueOf a :: *
  js :: (Sunroof (ValueOf a)) => a -> ValueOf a
\end{Code}
The type function
\Src{ValueOf} \cite{Chakravarty:05:AssociatedTypeSynonyms} 
provides the corresponding Sunroof type.
\Src{js} converts a value from Haskell to Sunroof. 
By design \Src{SunroofValue} 
only provides instances for values that can be converted in a pure
manner. 


%This approach ensures to bind a new value to a variable,
%instead of creating copies of that value everywhere
%it is used. This resolves some unwanted macro behavior 
%of Sunroof.

% JSTuple is not a major selling point nor is it essential to sunroof.
\begin{comment}
\subsubsection{RESOURCES - REMOVE WHEN FINISHED}

General Objects Types and Expressions
\begin{itemize}
\item We want to use Haskell's type system to give use type safety in JavaScript/Sunroof.
\item JavaScript is untyped / dynamically typed \TODO{what is it actually? cite/reference?}
\item We can not represent all types in JavaScript
\item We have to give a way to add types later on
\item Use approach from \cite{Svenningsson:12:CombiningEmbedding}
\item There is a core expression language that is used to describe how and expression is built
\item We wrap that core type into wrappers
\item Introduce \Src{Sunroof} to mark wrappers
\item Wrappers have specific functionality for certain type
\item Wrappers can have phantom type to give more type safety
\item Example \Src{JSString} and \Src{JSArray a}
\item Maybe go through example type like \Src{JSString} or \Src{()}.
\item Unavoidable need to cast types some times: \Src{cast}
\item Adding a new type can be done mechanically: Template Haskell
\end{itemize}
Haskell to Sunroof conversion
\begin{itemize}
\item Many haskell types have equivalent or similar types in JavaScript
\item So we offer \Src{SunroofValue}; It connects a Haskell value with 
its JavaScript companion through a type function (cite type functions)
and gives a function to convert: \Src{js}
\item Introduce \Src{SunroofValue}
\item Conversion is pure for most types since atomic values 
are created when converting.
\item Exception \Src{JSFunction}, same reason as for \Src{JSTuple} and \Src{tuple}
\item Show table with types as an overview. Small comment paragraph about table.
\item \Src{JSArray} not possible because of conflicts with \Src{String}
instance (minor point); also issue of mutability of arrays; observable 
allocation issue like with \Src{JSTuple}.
\item Forward reference to \SecRef{sec:threading-models} for \Src{JSMVar}
and \Src{JSChan}
\end{itemize}
JSTuple and records in JS
\begin{itemize}
\item Design decision: We do not want to introduce internal
structure for types. (Tuples aren't JS types)
\item \Src{JSTuple} exists to introduce types with custom structure
\item Introduce \Src{JSTuple}
\item manages composition and decomposition of custom types.
\item Meant to codify immutable records in JavaScript
\item They can only be changed
by decomposing them into Haskell and recreating a new structure
with other values.
\item Useful to manage more complex data structures in JavaScript.
\item Decomposition is pure operation: \Src{match}
\item Justification lies in the fact that they are meant to be 
immutable. Although there are possibilities to break this.
\item Composition is monadic effect: \Src{tuple}
\item It captures observable allocation and ensures that a reference to the
allocated value is created and used afterwards.
\item If it was not monadic the let binding would be like a macro
that reallocated the same object at each point. This 
is not desired behavior in most use-cases
\end{itemize}
\end{comment}






