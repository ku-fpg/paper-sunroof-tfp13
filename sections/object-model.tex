 
\section{JavaScript Object Model}
\label{sec:object-model}

\begin{comment}
JavaScript is object based. It provides various objects,
including numbers, booleans, maps, and others. We
provide in Sunroof about a dozen common object,
including \Src{JSObject} (the generic object type), \Src{JSNumber}
(floating point numbers), \Src{JSCanvas} (HTML5 canvas type) 
and others. A simple
casting function is provided when the type-system
needs to be overwritten. Along with each of these types,
we provide typed methods.
\begin{verbatim}
 jsDrawBox :: JSObject -> JS t ()
 jsDrawBox document = do
     foo <- document # getElementById("foo")
     cxt <- foo # getContext("2d")     cxt # drawRect (0,0,100,100)
\end{verbatim}
Here, \Src{\#} is a reverse apply, so the types
of the function in the above example are
\begin{verbatim}
(#) :: o -> (o -> JS t a) -> JS t a
getElementById :: JSString -> JSObject -> JS t JSContext
getContext :: JSString -> JSContext -> JS t JSCanvas
drawRect :: (JSNumber,JSNumber,JSNumber,JSNumber) -> JSCanvas -> JS t ()
\end{verbatim}        
From experience, even though we are targeting
an untyped language, the type system gets in the
way less than we expected.
\end{comment}

As JavaScript is not statically typed, one goal
of Sunroof is to use Haskell's type system to
increase the correctness of the written JavaScript.
At the same time we have the problem that we can not characterize 
all different types of objects in JavaScript, since 
users can write their own objects. This means our 
system to type JavaScript needs to be extensible.

To solve this problem we provide a basic \Src{Expr}ession 
language that has no associated type information to construct 
JavaScript expressions.
% We use a white lie instead of the actual fixpoint Expr type.
% We only need that type for expression sharing, which 
% is not essential for the scope of this paper.
\begin{verbatim}
data Expr 
  = Lit String -- Precompiled (atomic) JavaScript literal
  | Var Id     -- Variable
  | Apply expr [expr]    -- Function application
  ...
\end{verbatim}
This core expression type is then wrapped to represent a more specific 
type. Each of these wrappers implements the \Src{Sunroof} type 
class.
\begin{verbatim}
class SunroofArgument a => Sunroof a where
  box :: Expr -> a
  unbox :: a -> Expr
  ...
\end{verbatim}
\TODO{Maybe hide SunroofArgument at this point.}
It marks that these types represent possible values in JavaScript.
An example for this is the \Src{JSString} data type.
\begin{verbatim}
data JSString = JSString Expr
instance Sunroof JSString where
  box = JSString
  unbox (JSString e) = e
\end{verbatim}
But what do we gain by adding a wrapper? We can
provide specific functionality for each distinct type.
Our example type \Src{JSString} has a \Src{Monoid} and a 
\Src{IsString} instance that are not provided for other 
wrappers, e.g. \Src{JSBool} or \Src{JSNumber}.
This approach was first introduced by 
Svenningsson \cite{Svenningsson:12:CombiningEmbedding}.

Table \ref{tab:sunroof-types} gives a summary of the 
most prominent types in Sunroof. Some types like \Src{JSFunction} also involve 
phantom types to give more type safety \TODO{cite phantom types}.
The requisite \Src{SunroofArgument} is needed to permit values
to be function arguments. A detailed explanation is given in 
Section \ref{sec:functions-continuations}.





\begin{table}
\begin{center}
\begin{tabular}{r@{\quad}l@{\quad}l@{\quad}c}
\hline\rule{0pt}{12pt}%
  Constraint
  & Sunroof Type $\tau$
  & Haskell Analog \HaskellAnalog{$\tau$}
  & \Src{js} \\ \hline\rule{0pt}{12pt}%
  
  & \Src{()}       & \Src{()}     & $\checkmark$ \\
  & \Src{JSBool}   & \Src{Bool}   & $\checkmark$ \\
  & \Src{JSNumber} & \Src{Double} & $\checkmark$ \\
  & \Src{JSString} & \Src{String} & $\checkmark$ \\
  
  \Src{Sunroof $\alpha$}
  & \Src{JSArray $\alpha$} 
  & \Src{[$\HaskellAnalog{\alpha}$]}
  & \\
  
  \Src{SunroofKey $\alpha$}
  & \Src{JSMap $\alpha$ $\beta$}
  & \Src{Map $\HaskellAnalog{\alpha}$ $\HaskellAnalog{\beta}$}
  & \\
  \Src{Sunroof $\beta$} \\
  
  \Src{SunroofArgument $\alpha$}
  & \Src{JSFunction $\alpha$ $\beta$ }
  & \Src{$\HaskellAnalog{\alpha}$ $\rightarrow$ JS$_\Src{A}$ $\HaskellAnalog{\beta}$} 
  & $\checkmark$ \\
  \Src{Sunroof $\beta$} \\
  
  \Src{SunroofArgument $\alpha$}
  & \Src{JSMVar $\alpha$}
  & \Src{MVar $\HaskellAnalog{\alpha}$}
  & \\
  
  \Src{SunroofArgument $\alpha$}
  & \Src{JSChan $\alpha$}
  & \Src{Chan $\HaskellAnalog{\alpha}$}
  & \\[2pt]
\hline
\end{tabular}
\end{center}
\caption{Sunroof types and their Haskell pendant.}
\label{tab:sunroof-types}
\end{table} 
The table also shows that most of basic Haskell types have counterparts in
Sunroof. To convert these core Haskell values into their Sunroof 
counterparts we provide the \Src{SunroofValue} type class.
\begin{verbatim}
class SunroofValue a where
  type ValueOf a :: *
  js :: (Sunroof (ValueOf a)) => a -> ValueOf a
\end{verbatim}
It provides the corresponding Sunroof value through the type function
\Src{ValueOf} \TODO{cite type functions} and offers the function 
\Src{js} to convert to that type. By design \Src{SunroofValue} does
only provide instances for values that can be converted in a pure
manner. Certain primitive values in JavaScript are referential 
transparent according to \Src{==} while others, like general objects,
are not. As an example, if you call \Src{new Object()} twice you get the 
same empty object, but when compared by \Src{==} they are different. They
are not identical, because in this case reference equality is checked
instead of value equality. We call this observable allocation and handle 
it as a side-effect which should only occur in the \JS-monad. 

An advantage of this approach is that we ensure to bind a variable to the 
new value. Instead of creating copies of that value everywhere
it is used, we refer to one value. This resolves 
unwanted macro behavior of Sunroof in these cases.

Sunroof also offers the ability to work with record like data structures to 
keep values that belong together in one place.
For this purpose Sunroof offers the \Src{JSTuple} type class.
\begin{verbatim}
class Sunroof o => JSTuple o where
  type Internals o
  match :: (Sunroof o) => o -> Internals o
  tuple :: Internals o -> JS t o
\end{verbatim}
If you have a record of Sunroof data that you want to
encode as a JavaScript object you can provide this ability 
by implementing \Src{JSTuple}. The \Src{Internals} type function
delivers your record type. Encoding that record as a \Src{Sunroof}
value is done through \Src{tuple}. 
Because of the observable allocation issue, we have to be 
inside the \JS-monad to do this.
Decomposing the encoding is done with \Src{match}. Because the
\Src{JSTuple} idiom is meant to represent immutable data structures, this 
can be done in a pure manner although there are ways to mutate 
values referenced by the decomposed record, since they are only 
references to the actual data in JavaScript.

\begin{comment}
\subsubsection{RESOURCES - REMOVE WHEN FINISHED}

General Objects Types and Expressions
\begin{itemize}
\item We want to use Haskell's type system to give use type safety in JavaScript/Sunroof.
\item JavaScript is untyped / dynamically typed \TODO{what is it actually? cite/reference?}
\item We can not represent all types in JavaScript
\item We have to give a way to add types later on
\item Use approach from \cite{Svenningsson:12:CombiningEmbedding}
\item There is a core expression language that is used to describe how and expression is built
\item We wrap that core type into wrappers
\item Introduce \Src{Sunroof} to mark wrappers
\item Wrappers have specific functionality for certain type
\item Wrappers can have phantom type to give more type safety
\item Example \Src{JSString} and \Src{JSArray a}
\item Maybe go through example type like \Src{JSString} or \Src{()}.
\item Unavoidable need to cast types some times: \Src{cast}
\item Adding a new type can be done mechanically: Template Haskell
\end{itemize}
Haskell to Sunroof conversion
\begin{itemize}
\item Many haskell types have equivalent or similar types in JavaScript
\item So we offer \Src{SunroofValue}; It connects a Haskell value with 
its JavaScript companion through a type function (cite type functions)
and gives a function to convert: \Src{js}
\item Introduce \Src{SunroofValue}
\item Conversion is pure for most types since atomic values 
are created when converting.
\item Exception \Src{JSFunction}, same reason as for \Src{JSTuple} and \Src{tuple}
\item Show table with types as an overview. Small comment paragraph about table.
\item \Src{JSArray} not possible because of conflicts with \Src{String}
instance (minor point); also issue of mutability of arrays; observable 
allocation issue like with \Src{JSTuple}.
\item Forward reference to section \ref{sec:threading-models} for \Src{JSMVar}
and \Src{JSChan}
\end{itemize}
JSTuple and records in JS
\begin{itemize}
\item Design decision: We do not want to introduce internal
structure for types. (Tuples aren't JS types)
\item \Src{JSTuple} exists to introduce types with custom structure
\item Introduce \Src{JSTuple}
\item manages composition and decomposition of custom types.
\item Meant to codify immutable records in JavaScript
\item They can only be changed
by decomposing them into Haskell and recreating a new structure
with other values.
\item Useful to manage more complex data structures in JavaScript.
\item Decomposition is pure operation: \Src{match}
\item Justification lies in the fact that they are meant to be 
immutable. Although there are possibilities to break this.
\item Composition is monadic effect: \Src{tuple}
\item It captures observable allocation and ensures that a reference to the
allocated value is created and used afterwards.
\item If it was not monadic the let binding would be like a macro
that reallocated the same object at each point. This 
is not desired behavior in most use-cases
\end{itemize}
\end{comment}






