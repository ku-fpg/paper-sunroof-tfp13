 
\section{JavaScript Object Model}
\label{sec:object-model}

JavaScript is object based. It provides various objects,
including numbers, booleans, maps, and others. We
provide in Sunroof about a dozen common object,
including \Src{JSObject} (the generic object type), \Src{JSNumber}
(floating point numbers), \Src{JSCanvas} (HTML5 canvas type) 
and others. A simple
casting function is provided when the type-system
needs to be overwritten. Along with each of these types,
we provide typed methods.
\begin{verbatim}
 jsDrawBox :: JSObject -> JS t ()
 jsDrawBox document = do
     foo <- document # getElementById("foo")
     cxt <- foo # getContext("2d")     cxt # drawRect (0,0,100,100)
\end{verbatim}
Here, \Src{\#} is a reverse apply, so the types
of the function in the above example are
\begin{verbatim}
(#) :: o -> (o -> JS t a) -> JS t a
getElementById :: JSString -> JSObject -> JS t JSContext
getContext :: JSString -> JSContext -> JS t JSCanvas
drawRect :: (JSNumber,JSNumber,JSNumber,JSNumber) -> JSCanvas -> JS t ()
\end{verbatim}        
From experience, even though we are targeting
an untyped language, the type system gets in the
way less than we expected.

\begin{table}
\caption{Major instances of the Sunroof class}
\begin{center}
\begin{tabular}{r@{\quad}l@{\quad}l@{\quad}c}
\hline\rule{0pt}{12pt}%
  Constraint
  & Sunroof Type $\tau$
  & Haskell Analog \HaskellAnalog{$\tau$}
  & \Src{js} \\ \hline\rule{0pt}{12pt}%
  
  & \Src{()}       & \Src{()}     & $\checkmark$ \\
  & \Src{JSBool}   & \Src{Bool}   & $\checkmark$ \\
  & \Src{JSNumber} & \Src{Double} & $\checkmark$ \\
  & \Src{JSString} & \Src{String} & $\checkmark$ \\
  
  \Src{Sunroof $\alpha$}
  & \Src{JSArray $\alpha$} 
  & \Src{[$\HaskellAnalog{\alpha}$]}
  & \\
  
  \Src{SunroofKey $\alpha$}
  & \Src{JSMap $\alpha$ $\beta$}
  & \Src{Map $\HaskellAnalog{\alpha}$ $\HaskellAnalog{\beta}$}
  & \\
  \Src{Sunroof $\beta$} \\
  
  \Src{SunroofArgument $\alpha$}
  & \Src{JSFunction $\alpha$ $\beta$ }
  & \Src{$\HaskellAnalog{\alpha}$ $\rightarrow$ JS$_\Src{A}$ $\HaskellAnalog{\beta}$} 
  & $\checkmark$ \\
  \Src{Sunroof $\beta$} \\
  
  \Src{SunroofArgument $\alpha$}
  & \Src{JSMVar $\alpha$}
  & \Src{MVar $\HaskellAnalog{\alpha}$}
  & \\
  
  \Src{SunroofArgument $\alpha$}
  & \Src{JSChan $\alpha$}
  & \Src{Chan $\HaskellAnalog{\alpha}$}
  & \\[2pt]
\hline
\end{tabular}
\end{center}
\end{table} 

\subsubsection{RESOURCES - REMOVE WHEN FINISHED}

General Objects Types and Expressions
\begin{itemize}
\item We want to use Haskell's type system to give use type safety in JavaScript/Sunroof.
\item JavaScript is untyped / dynamically typed (TODO: what is it actually? cite/reference?)
\item We can not represent all types in JavaScript
\item We have to give a way to add types later on
\item Use approach from \cite{Svenningsson:12:CombiningEmbedding}
\item There is a core expression language that is used to describe how and expression is built
\item We wrap that core type into wrappers
\item Introduce \Src{Sunroof} to mark wrappers
\item Wrappers have specific functionality for certain type
\item Wrappers can have phantom type to give more type safety
\item Example \Src{JSString} and \Src{JSArray a}
\item Maybe go through example type like \Src{JSString} or \Src{()}.
\item Unavoidable need to cast types some times: \Src{cast}
\item Adding a new type can be done mechanically: Template Haskell
\end{itemize}
JSTuple and records in JS
\begin{itemize}
\item Design decision: We do not want to introduce internal
structure for types. (Tuples aren't JS types)
\item \Src{JSTuple} exists to introduce types with custom structure
\item Introduce \Src{JSTuple}
\item manages composition and decomposition of custom types.
\item Meant to codify immutable records in JavaScript
\item They can only be changed
by decomposing them into Haskell and recreating a new structure
with other values.
\item Useful to manage more complex data structures in JavaScript.
\item Decomposition is pure operation: \Src{match}
\item Justification lies in the fact that they are meant to be 
immutable. Although there are possibilities to break this.
\item Composition is monadic effect: \Src{tuple}
\item It captures observable allocation and ensures that a reference to the
allocated value is created and used afterwards.
\item If it was not monadic the let binding would be like a macro
that reallocated the same object at each point. This 
is not desired behavior in most use-cases
\end{itemize}
Haskell to Sunroof conversion
\begin{itemize}
\item Many haskell types have equivalent or similar types in JavaScript
\item So we offer \Src{SunroofValue}; It connects a Haskell value with 
its JavaScript companion through a type function (cite type functions)
and gives a function to convert: \Src{js}
\item Introduce \Src{SunroofValue}
\item Conversion is pure for most types since atomic values 
are created when converting.
\item Exception \Src{JSFunction}, same reason as for \Src{JSTuple} and \Src{tuple}
\item Show table with types as an overview. Small comment paragraph about table.
\item \Src{JSArray} not possible because of conflicts with \Src{String}
instance (minor point); also issue of mutability of arrays; observable 
allocation issue like with \Src{JSTuple}.
\item Forward reference to section \ref{sec:threading-models} for \Src{JSMVar}
and \Src{JSChan}
\end{itemize}







