 
\section{The Sunroof Server}
\label{sec:server}

The Sunroof compiler can compile JavaScript that can be used
stand-alone. But Sunroof really comes
into its own when used with the Sunroof server.

It provides infrastructure to send arbitrary pieces 
of JavaScript to a calling website for execution. 
Like this it is possible to interleave Haskell and JavaScript 
computations with each other as needed. The three major functions
provided are \Src{sunroofServer}, \Src{syncJS} and \Src{asyncJS}.
\begin{Code}
type SunroofApp = SunroofEngine -> IO ()
sunroofServer :: SunroofServerOptions -> SunroofApp -> IO ()

syncJS  :: SunroofResult a 
        => SunroofEngine -> JS t a -> IO (ResultOf a)
asyncJS :: SunroofEngine -> JS t () -> IO ()
\end{Code}
\Src{sunroofServer} starts the server.
It will call the given callback function
for each request to the server.
\Src{syncJS} and \Src{asyncJS} allow the server
to remotely execute monadic JavaScript from inside the 
requesting website.
\Src{asyncJS} executes JavaScript asynchronously without 
waiting for a return value. In contrast to that 
\Src{syncJS} waits until the execution is complete and
then sends the result back to the server and converts it 
into a Haskell value that can be processed further. 
Values that can be returned from a synchronous execution 
have to implement the \Src{SunroofResult} class. It contains a type-function
\Src{ResultOf} that maps the 
Sunroof type to a corresponding Haskell type and also 
implements a function to convert incoming data to that 
type.

Besides these basic functions it also provides the abstract 
concepts of \Src{Downlink}s and \Src{Uplink}s. They are 
channels for sending data either from the server or to
the server.

\begin{comment}
\subsubsection{RESOURCES - REMOVE WHEN FINISHED}

\begin{itemize}
\item Introduce major idea of a server that can communicate
with the calling website and send arbitrary pieces
of JavaScript to execute on demand.
\item \TODO{Go into detail about the kansas-comet stuff }
or just give a reference to the original paper?
\item Introduce the major function for running and communicating:
\Src{sunroofServer}, \Src{syncJS} and \Src{asyncJS}
\item Describe what they do.
\item Specifically look at \Src{syncJS} and \Src{SunroofResult}
used to return actual Haskell values for the sent JavaScript return value.
\item Introduce abstractions \Src{Downlink} and \Src{Uplink} that
can be used to communicate in either direction.
\item \TODO{This is only a brief introduction?}
\end{itemize}
\end{comment}





