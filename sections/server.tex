 
\section{The Sunroof Server}
\label{sec:server}

The Sunroof compiler can compile JavaScript than can be used
stand-alone inside a web application. But Sunroof really comes
into its own when used with the Sunroof server. There
are three major functions in our server.

\begin{verbatim}
sunroofServer :: SunroofServerOptions -> SunroofApp -> IO ()
syncJS  :: SunroofResult a => SunroofEngine 
        -> JS t a -> IO (ResultOf a)
asyncJS :: SunroofEngine -> JS t () -> IO ()
\end{verbatim}        

\Src{sunroofServer} starts a small web server,
that calls the callback function for each request.
\Src{syncJS} and \Src{asyncJS} allow the Sunroof programmer
to remotely execute monadic JavaScript from the server.
\Src{ResultOf} is a type-function, that maps the 
Sunroof type to a corresponding Haskell type.


\subsubsection{RESOURCES - REMOVE WHEN FINISHED}

\begin{itemize}
\item Introduce major idea of a server that can communicate
with the calling website and send arbitrary pieces
of JavaScript to execute on demand.
\item TODO: Go into detail about the kansas-comet stuff 
or just give a reference to the original paper?
\item Introduce the major function for running and communicating:
\Src{sunroofServer}, \Src{syncJS} and \Src{asyncJS}
\item Describe what they do.
\item Specifically look at \Src{syncJS} and \Src{SunroofResult}
used to return actual Haskell values for the sent JavaScript return value.
\item Introduce abstractions \Src{Downlink} and \Src{Uplink} that
can be used to communicate in either direction.
\item TODO: This is only a brief introduction?
\end{itemize}






