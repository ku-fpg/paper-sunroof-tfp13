% This is LLNCS.DEM the demonstration file of
% the LaTeX macro package from Springer-Verlag
% for Lecture Notes in Computer Science,
% version 2.4 for LaTeX2e as of 16. April 2010
%
\documentclass{llncs}
%
\usepackage{amsfonts}

\newcommand{\SunroofAnalog}[1]{#1\ensuremath{_\downarrow}}
\newcommand{\HaskellAnalog}[1]{#1\ensuremath{_\uparrow}}

\newcommand{\NOTE}[1]{{\Large\textbf{NOTE:}\ #1}}
\newcommand{\TODO}[1]{{\textbf{TODO:}\ #1}}
\newcommand{\Src}[1]{{\tt{#1}}}

\newcommand{\IO}{\Src{IO}}
\newcommand{\JS}{\Src{JS}}
\newcommand{\JSA}{\ensuremath{\Src{JS}_\Src{A}}}
\newcommand{\JSB}{\ensuremath{\Src{JS}_\Src{B}}}

\newcommand{\Figure}[3]{%
\FigureS{#1}{#2}{#3}{scale=0.55,clip=true,trim=0.45cm 0.45cm 0.45cm 0.45cm}
}

\newcommand{\FigureS}[4]{%
\begin{figure}[h]%
\vspace{-0.5cm}%
\begin{center}%
\includegraphics[#4]{#2}%
\vspace{-0.5cm}%
\end{center}%
\caption{#3}%
\label{#1}%
\vspace{-0.5cm}%
\end{figure}%
}


\begin{document}
%
\title{Sunroof: A JavaScript Monad Compiler}
%\subtitle{}
%
\titlerunning{Sunroof}  % abbreviated title (for running head)
%                                     also used for the TOC unless
%                                     \toctitle is used
%
\author{Jan Bracker\inst{1,2}  \and Andy Gill\inst{1}}
%
\authorrunning{Jan Bracker \and Andy Gill} % abbreviated author list (for running head)
%
%%%% list of authors for the TOC (use if author list has to be modified)
\tocauthor{Jan Bracker, Andy Gill}
%
\institute{%
ITTC / EECS \\
The University of Kansas\\
Lawrence, KS 66045\\
~\\
\and
Institut f{\"u}r Informatik\\
Christian-Albrechts-Universit{\"a}t\\
Kiel, Germany}


\maketitle

\begin{abstract}        
        
Sunroof is a Domain Specific Language (DSL) for generating Javascript.
Sunroof is build on top of the JS-monad, which, like the Haskell IO-monad, allows 
read and write access to external resources, but specifically Javascript
resources. As such, Sunroof is primarily a feature-rich foreign
function API to the browser's Javascript engine, and all the browser-specific
functionality, like HTML-based rendering, event handling, and 
drawing to the HTML5 canvas. 

In this paper, we give the design and implementation of Sunroof, a 
deeply embedded Haskell-hosted DSL.
This makes it easy to use Haskell abstractions for larger Javascript
applications without obscuring the produced Javascript on the Haskell
level. 
Furthermore, Sunroof offers two threading models for 
building on top Javascript, atomic and blocking threads.
This allows full access to Javascript APIs, but
using Haskell concurrency patterns, like MVars and Channels.
In combination with a small web services package, like Scotty,
Sunroof offers a great platform to build interactive web applications,
giving the ability to interleave Haskell and Javascript computations
with each other as needed.
\keywords{Haskell DSL, Javascript, Web Technologies, Cloud Computing}
\end{abstract}
%
\section{Introduction}

An essential part of most modern web applications is Javascript.
It opens various possibilities to create browser based applications.
Examples for this development are Googlemail or Facebook.
These applications rely on Javascript heavily. What makes this 
development possible are the fast interpreters available in all
major browsers. Developing applications inside the browser has
the advantage of being independent from the underlying hard- and
software platform. The problem is shifted from the actual machine
to the browser. Also modern web technologies like HTML5 and CSS
give easy to use tools for developing user interfaces for the
mentioned application. They are standardized and improved over years
up to now. 

Yet, writing actual code in Javascript can be tedious and error prone 
due to its not-statically typed nature. Also the object-oriented
paradigm and the callback mechanism used instead of proper multi-threading 
can be a stumbling block for veteran Haskell programmers. Especially when
facing the lack of type safety.

To make the power of Javascript available to Haskell programmers 
we developed Sunroof. It enables Haskell programmers to write
Javascript using Haskell through a deep embedding. This 
approach has several advantages:

\begin{description}
\item[Type Safety:] Sunroof provides custom types for all major Javascript
types. Thus it gives type safety similar to what we are used to from 
normal Haskell programs when writing Javascript.
\item[Paradigms \& Patterns:] As a Haskell EDSL Sunroof follows functional programming 
paradigms and principals. It is possible to reuse common combinators and 
patterns in Sunroof.
\item[Similarity:] At the same time Sunroof offers a API for Javascript 
that allows to translate Javascript into Sunroof code easily.
\item[Optimization:] Due to Sunroofs deep embedded nature it is able
to apply optimizations like expression sharing when compiling to Javascript.
\end{description}

\NOTE{This might be a bit to enthusiastic.}

We want to use the Javascript API. Fast interpreters,
running in a browser advantage, 

We do not want to program in Javascript.

\section{Example Sunroof Program}

A simple drawing program, that we build up step by step.
Perhaps bouncing ball.
Perhaps drawing line.

\section{Background}

We chose to implement Sunroof as deep embedded DSL for different
reasons. Porting a compiler, like GHC, to Javascript would have
cost a lot of effort. There is no clear way to port the existing
API and there are many language features that would be difficult to
translate. Providing a foreign function interface would 
introduce the cost of translating back- and forward. 
\TODO{Not even sure how this would work or where cost comes in.}
Another possibility would be to develop a custom language. 
Besides that people would need to learn a third language for this 
approach, it would involve effort similar to porting a compiler.
As a deep embedded DSL people only have to know Haskell to learn
the new language. The cost of developing the DSL remains light, 
because we can build upon a mature language and its features.

Choices when compiling to Javascript
\begin{itemize}
\item Port a compiler to Javascript - issues, mismatching API, other efforts.
\item Provide an FFI - back and forward cost.
\item Compile a custom, cut down FP language. - example
\item DSL, restrictive, syntactically, how do we handle binding?
\end{itemize}

\subsection{Basic Idea}

As we have seen in the previous example application Sunroof uses a 
monad to model computations that can be translated to Javascript.
A binding operation is thought to be an assignment to 
a fresh variable in the reified language.

We use monad reification.
\begin{itemize}
\item binding in Haskell becomes binding in reified language.
\item Feels like a native monad, cf STM.
\item Can build abstractions on top of this.
\end{itemize}

Design objectives and contributions.
\begin{itemize}
\item 
Specifically, we want to explore how similar to a native Haskell monad,
like the IO or STM monads, can we make the JS monad feel to Sunroof users.
\item We want to reflect objects. (easy)

\item We want also investigate the relationship between Haskell functions
and Javascript functions. In particular, there is an isomorphism
between monadic functions in Haskell, with the type \verb|a -> JS b|,
and Javascript functions, which we will notate using \verb|JSFunction|.
This relationship is more interesting than a Haskell synonym;
it reflects the reification options for capturing functions.
\end{itemize}

 
Compiling
\begin{itemize}
\item If we transliterate, we have straight line code, can not pause.
    (Wait for Mvar, for example)
\item If we CPS translate, we can use continuations to capture the
   notion of a paused thread. Works well.
   Problems:
  \begin{itemize}
   \item Can not translate functions, how do they get there return value
   \item (Assumes straight line code.)
   \item Also, the code becomes unreadable to anyone except a die-hard 
       compiler freak.
  \end{itemize}
\end{itemize}
 
Choice:
\begin{itemize}
\item We support both!
\item Phantom argument to JS
\item A = Atomic, B = Blockable.
\end{itemize}

Highlights
\begin{itemize}
\item function / continuation
\item implementing wait \& fork
\end{itemize}

\begin{table}
\caption{Major instances of the Sunroof class}
\begin{center}
\begin{tabular}{r@{\quad}l@{\quad}l@{\quad}c}
\hline\rule{0pt}{12pt}%
  Constraint
  & Sunroof Type $\tau$
  & Haskell Analog \HaskellAnalog{$\tau$}
  & \Src{js} \\ \hline\rule{0pt}{12pt}%
  
  & \Src{()}       & \Src{()}     & $\checkmark$ \\
  & \Src{JSBool}   & \Src{Bool}   & $\checkmark$ \\
  & \Src{JSNumber} & \Src{Double} & $\checkmark$ \\
  & \Src{JSString} & \Src{String} & $\checkmark$ \\
  
  \Src{Sunroof $\alpha$}
  & \Src{JSArray $\alpha$} 
  & \Src{[$\HaskellAnalog{\alpha}$]}
  & $\checkmark$ \\
  
  & \Src{JSMap JSString $\beta$}
  & \Src{Map String $\HaskellAnalog{\beta}$}
  & \\
  
  \Src{SunroofArgument $\alpha$}
  & \Src{JSFunction $\alpha$ $\beta$ }
  & \Src{$\HaskellAnalog{\alpha}$ $\rightarrow$ JS$_\Src{A}$ $\HaskellAnalog{\beta}$} 
  & $\checkmark$ \\
  \Src{Sunroof $\beta$} \\
  
  \Src{SunroofArgument $\alpha$}
  & \Src{JSMVar $\alpha$}
  & \Src{MVar $\HaskellAnalog{\alpha}$}
  & \\
  
  \Src{SunroofArgument $\alpha$}
  & \Src{JSChan $\alpha$}
  & \Src{Chan $\HaskellAnalog{\alpha}$}
  & \\[2pt]
\hline
\end{tabular}
\end{center}
\end{table} 

\begin{table}
\caption{Reifying and calling Javascript functions}
\begin{center}
\begin{tabular}{r@{\quad}c@{\quad}c@{\quad}c@{\quad}c}
\hline\rule{0pt}{12pt}%

                & Monadic Function      & Reification   & Object in     & Invocation\\
                & in Haskell            & Function      & Javascript    & Function\\
\hline\rule{0pt}{12pt}%
  Functions
  & $\alpha\rightarrow\ $\Src{JS}$_\Src{A}~\beta$
  & \Src{function}
  & \Src{JSFunction}~$\alpha~\beta$
  & \Src{apply} \\
  Continuations
  & $\alpha\rightarrow\ $\Src{JS}$_\Src{B}~\Src{()}$
  & \Src{continuation}
  & \Src{JSContinuation}~$\alpha$
  & \Src{goto}\\
\hline
\end{tabular}
\end{center}
\end{table} 

Up to this point, Sunroof is academically interesting, but in a real sense
we are writing Javascript using Haskell syntax, so why not just write
Javascript? There are three things we have bought by using our DSL.
\begin{itemize}
\item Typing -- We have a simple monomorphic type system that helps development of Sunroof code.
\item Haskell Abstractions --
\item Dynamic Generation -- 
\end{itemize}

\section{Haskell Abstraction}

About MVar and Chan.

\section{Sunroof Server}



\section{Case Study - hp2ps}

hp2ps is a tools for graphically displaying GHC heap profiles.
The tools history is that is was written by Malcolm Wallace while working
for the York ??? group, in 1992. The tool works as a simple filter,
reading hp files, and outputting postscript, with the postscript
doing the heavy lifting of the graphical rendering.

-----------

Talk about Active, but not in any details.


 It was written
in C by 

\section{Related Work}

To Javascript in Javascript (CoffeeScript).

Haskell to Javascript

Other Deep embeddings.

%
% ---- Bibliography ----
%


\bibliographystyle{splncs03}
\bibliography{sunroof}

\end{document}
