% This is LLNCS.DEM the demonstration file of
% the LaTeX macro package from Springer-Verlag
% for Lecture Notes in Computer Science,
% version 2.4 for LaTeX2e as of 16. April 2010
%
\documentclass{llncs}
%
\usepackage{amsfonts}
\usepackage{comment}
\usepackage{graphicx}
\usepackage{caption}
\usepackage{subcaption}
\usepackage{verbatim}
\usepackage{url}

\newcommand{\SunroofAnalog}[1]{#1\ensuremath{_\downarrow}}
\newcommand{\HaskellAnalog}[1]{#1\ensuremath{_\uparrow}}

%\newcommand{\NOTE}[1]{{\Large\textbf{NOTE:}\ #1}}
\newcommand{\TODO}[1]{{(\textbf{TODO:}\ #1)}}
\newcommand{\Src}[1]{{\tt{#1}}}

\newcommand{\IO}{\Src{IO}}
\newcommand{\JS}{\Src{JS}}
\newcommand{\JSI}{\Src{JSI}}
\newcommand{\JSA}{\ensuremath{\Src{JS}_\Src{A}}}
\newcommand{\JSB}{\ensuremath{\Src{JS}_\Src{B}}}

\newcommand{\Figure}[3]{%
\FigureS{#1}{#2}{#3}{scale=0.55,clip=true,trim=0.45cm 0.45cm 0.45cm 0.45cm}
}

\newcommand{\FigureS}[4]{%
\begin{figure}[t]%
%\vspace{-0.5cm}%
\begin{center}%
\includegraphics[#4]{#2}%
\vspace{-0.5cm}%
\end{center}%
\caption{#3}%
\label{#1}%
\vspace{-0.5cm}%
\end{figure}%
}

\newenvironment{Code}{\verbatim}{\endverbatim}

\newcommand{\FigRef}[1]{Fig.~\ref{#1}}
\newcommand{\SecRef}[1]{Section~\ref{#1}}
\newcommand{\TabRef}[1]{Table~\ref{#1}}

% intentional for the referee's copy
\pagestyle{plain}
\newcommand{\CURSOR}{\noindent\rule{\textwidth}{4pt}}

\begin{document}
%
\title{Sunroof: A Monadic DSL for Generating JavaScript}
%\subtitle{}
%
\titlerunning{Sunroof}  % abbreviated title (for running head)
%                                     also used for the TOC unless
%                                     \toctitle is used
%
\author{Andy Gill\inst{1} and Jan Bracker\inst{1,2}}
%
\authorrunning{Jan Bracker \and Andy Gill} % abbreviated author list (for running head)
%
%%%% list of authors for the TOC (use if author list has to be modified)
\tocauthor{Andy Gill, Jan Bracker}
%
\institute{%
ITTC / EECS \\
The University of Kansas, Lawrence, KS 66045\\
~\\
\and
Institut f{\"u}r Informatik\\
Christian-Albrechts-Universit{\"a}t, Kiel, Germany}

\maketitle

\begin{abstract}        
Sunroof is a Haskell-hosted Domain Specific Language (DSL) for generating JavaScript.
Sunroof is built on top of the JavaScript monad, which, like the Haskell \IO-monad, allows 
access to external resources, but specifically JavaScript
resources. As such, Sunroof is primarily a feature-rich 
foreign-function API to the browser's JavaScript engine, and all the browser-specific
functionality, including HTML-based rendering, event handling, and 
drawing to the HTML5 canvas. 

In this paper, we give the design and implementation of Sunroof.
Using monadic reification, we generate JavaScript from
a deep embedding of the JavaScript monad.
The Sunroof DSL has the feel of native Haskell, with a simple
Haskell-based type schema to guide the Sunroof programmer.
Furthermore, because we are generating code,
we can offer Haskell-style concurrency patterns, such as MVars and Channels.
In combination with a web-services package,
the Sunroof compiler offers a robust platform to build interactive web applications.

\keywords{DSLs, JavaScript, Web Technologies, Cloud Computing}
\end{abstract}
%
%%%%%%%%%%%%%%%%%%%%%%%%%%%%%%%%%%%%%%%%%%%%%%%%%%%%%%%%%%%%%%%%%%%%%%%%%%%%%%%%%%%%%%%%%%%%%%%%
%%%%%%%%%%%%%%%%%%%%%%%%%%%%%%%%%%%%%%%%%%%%%%%%%%%%%%%%%%%%%%%%%%%%%%%%%%%%%%%%%%%%%%%%%%%%%%%%
%%%%%%%%%%%%%%%%%%%%%%%%%%%%%%%%%%%%%%%%%%%%%%%%%%%%%%%%%%%%%%%%%%%%%%%%%%%%%%%%%%%%%%%%%%%%%%%%

\section{Introduction}\label{sec:intro}

% Simon: Describe the problem
There are many reasons to want to program in a functional language:
efficiency of development cost, informally reasoning, high-level 
control- and concurrency-structures, like monads~\cite{...}.
However, mainstream languages often have better environmental support
than what is provided by functional languages,
for example the Objective C and the iOS eco-structure, or JavaScript and HTML5
web browsers.
This paper examines the challenges of providing
an intentionally blurred interface between Haskell
and JavaScript, to support the development of web-based applications.

% Simon: State your contributions
JavaScript is an imperative language with access to a wide range
of established and useful services like graphical canvases and event
handling of browser events. 
We want to express JavaScript in Haskell, adding use
of Haskell's static typing, and gaining access to JavaScript services
in the browser directly in Haskell.

In general, there are a number of ways of providing access to non-native services,
such as the JavaScript canvas.
\begin{enumerate}
\item A first approach is to provide, in Haskell, 
foreign function ``hooks'' to key JavaScript functionality.
Haskell already provides many similar hooks into the RTS,
so why not into the JavaScript engine?
For this to work efficiently, the compiler
needs to target JavaScript. There are already a
number is systems doing this~\cite{...}.
If executed well, this would be ideal,
there are shortcomings: many standard libraries
are not supported directly, the generated
code is not as efficiently executed as native Haskell,
and the compilers are still immature and incomplete.

\item An second approach is to keep the 
foreign function ``hooks'' to key JavaScript functionality,
but instead run Haskell as a server that JavaScript
and the browser interacts with.
Unfortunately, every JavaScript call becomes an expensive proposition: an RPC call
to a browser.
Though some straight-line calls can be batched together --
our own blank-canvas hackage package~\cite{..} was built on this idea --
the granularity of interaction through JavaScript call is just too fine for
this idea to scale well.

\item A first, alternative approach is to use Haskell to generate JavaScript from
an abstract syntax tree of the JavaScript code. There have
been a number of attempt to have DSLs that do the in other spaces,
and functional languages have good support for generating trees.
However, this approach works well for data-flow, for example
as use in Lava. When describing control-flow,
as would be typically in a JavaScript program,  writing such an AST always
feels forced, primarily because there is a mismatch between
bindings in the native language (Haskell) and the non-native
generated code (JavaScript variable names).

\end{enumerate}

This paper investigate the expansion of scope of the second option,
specifically adding binding to the Haskell code
that can be batch together and send to the JavaScript engine,
and then scaling the language to support larger examples.
Binding is done using regular Haskell monadic binding, and fits naturally
into what we expect from a monadic API. Until recently,
it was thought that it was impossible to use a regular
Haskell monad for this purpose. In a previous paper,
we show that such a construction is possible~\cite{..}
In this paper, we expand on this observation,
and show that this form of reification is useful in practice.

Further building on this capability, we also investigate providing
JavaScript control flow and function abstraction mechanisms
to the Haskell programmer interested in using the browser API.
Though for technical reasons we can not compile the transitional
pattern matching and let-binding to JavaScript without committing
to a full compiler Haskell to JavaScript compiler, both
control flow and function abstraction can be provided
with a small syntactical overhead.

With these three capabilities, a programmer can start programming
using the provided JavaScript API directly, and refine
their program to migrate more and more computation
from the server into the browser. In a real sense,
this blurs the distention between an RPC approach (2),
and a full Haskell to JavaScript compiler (1),
built using existing infrastructure, and not needing
the full compiler.




\begin{comment}
\section{Example Sunroof Program}
\label{sec:simple-example}

A simple drawing program, that we build up step by step.
Perhaps bouncing ball.
Perhaps drawing line.
\end{comment}

%%%%%%%%%%%%%%%%%%%%%%%%%%%%%%%%%%%%%%%%%%%%%%%%%%%%%%%%%%%%%%%%%%%%%%%%%%%%%%%%%%%%%%%%%%%%%%%%
%%%%%%%%%%%%%%%%%%%%%%%%%%%%%%%%%%%%%%%%%%%%%%%%%%%%%%%%%%%%%%%%%%%%%%%%%%%%%%%%%%%%%%%%%%%%%%%%
%%%%%%%%%%%%%%%%%%%%%%%%%%%%%%%%%%%%%%%%%%%%%%%%%%%%%%%%%%%%%%%%%%%%%%%%%%%%%%%%%%%%%%%%%%%%%%%%

\section{Calling JavaScript from Haskell}
\label{sec:js-rpc}

From a programmers' point of view, calling JavaScript functions
appears straightforward. We, as a community know how to reflect an
API into Haskell, using the IO monad. Furthermore, objects in
the target API become handles in Haskell. 

As a first example, consider this simplified example of Sunroof code, and corresponding JavaScript.
\noindent
\begin{Code}
-- Haskell                          // JavaScript
ioCode :: IO ()
ioCode = send jsCode

jsCode :: JS ()
jsCode = do                        function jsCode() {
   name <- prompt "Your name?"       var v0 = prompt("Your name?"); 
   alert ("Your name: " <> name)     alert("Your name: " + v0); 
                                   }
\end{Code}%     
%\noindent 

Here, we use a new monad, the \Src{JS} monad, our JavaScript
analog to the \Src{IO} monad,
and an explicit \Src{send} command that sends the JavaScript to the browser.
This reversal of control, where the
server sends the clients commands is called the Ajax Comet~\cite{..},
or simply long polling.
This interface also bundles the \Src{prompt} and \Src{alert} commands
into one interaction transaction.
It is this flavor of interface we want to support in our Sunroof compiler
and web server.  

To make Sunroof a viable interface to JavaScript, we need to
resolve the following issues:
\begin{itemize}
\item JavaScript is an object-based, imperative, dynamically typed language.
Haskell is a pure, function-based, statically typed language.
Specifically how do these two worlds map onto each other.
In section~\ref{sec:object-model}, we introduce expressions and statements
in Sunroof, and how they jointly form an object model.
%
\item We need to select a concurrent model for Sunroof.
JavaScript natively only supports non-blocking threads.
In section~\ref{sec:threading-models}, we give the
Sunroof interface for providing both non-blocking and blocking cooperative threads,
using the type system to delimit the two concurrency models.
%
\item We choice to provide a way of defining functions
in Sunroof in a way that they are first-class functions
in JavaScript. This uses our \Src{JS} monad, and
is discussed in section~\ref{sec:functions-continuations}.
%
\item We need to provide a foreign function interface,
to allow us to call specific JavaScript-native functions,
like \Src{prompt} and \Src{alert}.
In section~\ref{sec:ffi} we present this interface.
%
\item Critically, we need to be able to compile our Sunroof DSL
into JavaScript. We look at this in section~\ref{sec:compiler}.
\item To enable the compiled code to dynamically interact with
a web browser, we provide an expansion of the \Src{send} idea above,
which we discuss in section~\ref{sec:server}.
\end{itemize}

We close with Sunroof examples, related work, and our conclusions.

%%%%%%%%%%%%%%%%%%%%%%%%%%%%%%%%%%%%%%%%%%%%%%%%%%%%%%%%%%%%%%%%%%%%%%%%%%%%%%%%%%%%%%%%%%%%%%%%
%%%%%%%%%%%%%%%%%%%%%%%%%%%%%%%%%%%%%%%%%%%%%%%%%%%%%%%%%%%%%%%%%%%%%%%%%%%%%%%%%%%%%%%%%%%%%%%%
%%%%%%%%%%%%%%%%%%%%%%%%%%%%%%%%%%%%%%%%%%%%%%%%%%%%%%%%%%%%%%%%%%%%%%%%%%%%%%%%%%%%%%%%%%%%%%%%


\section{Sunroof Values}
\label{sec:object-model}

In the Sunroof DSL, like Haskell itself, we make a distinction between expressions and statements.

\subsection{Sunroof Expressions}

In JavaScript, there are a small number of core types, such as \Src{Object}, \Src{Array}, and \Src{String}.
In our JavaScript object model, there is a reflection of this family of core types, all prefixed with \Src{JS}.
There is support for booleans, strings, numbers, functions,
object and arrays, and other common programming structures.
The use of \Src{JSNumber}, rather than (say) \Src{Double},
explicitly reminds us of the enforced diminished capabilities of
being within an embedded language,
and all of the \Src{JS}-types are representable in our target language.
Table~\ref{tab:sunroof-types} enumerates major \Src{JS} types used in Sunroof.

\begin{table}[t]
\begin{center}
\begin{tabular}{r@{\quad}l@{\quad}l@{\quad}c}
\hline\rule{0pt}{12pt}%
  Constraint
  & Sunroof Type $\tau$
  & Haskell Analog \HaskellAnalog{$\tau$}
  & \Src{js} \\ \hline\rule{0pt}{12pt}%
  
  & \Src{()}       & \Src{()}     & $\checkmark$ \\
  & \Src{JSBool}   & \Src{Bool}   & $\checkmark$ \\
  & \Src{JSNumber} & \Src{Double} & $\checkmark$ \\
  & \Src{JSString} & \Src{String} & $\checkmark$ \\
  
  \Src{Sunroof $\alpha$}
  & \Src{JSArray $\alpha$} 
  & \Src{[$\HaskellAnalog{\alpha}$]}
  & \\
  
  \Src{SunroofKey $\alpha$}
  & \Src{JSMap $\alpha$ $\beta$}
  & \Src{Map $\HaskellAnalog{\alpha}$ $\HaskellAnalog{\beta}$}
  & \\
  \Src{Sunroof $\beta$} \\
  
  \Src{SunroofArgument $\alpha$}
  & \Src{JSFunction $\alpha$ $\beta$ }
  & \Src{$\HaskellAnalog{\alpha}$ $\rightarrow$ JS$_\Src{A}$ $\HaskellAnalog{\beta}$} 
  & \\
  \Src{Sunroof $\beta$} \\
  
  \Src{SunroofArgument $\alpha$}
  & \Src{JSMVar $\alpha$}
  & \Src{MVar $\HaskellAnalog{\alpha}$}
  & \\
  
  \Src{SunroofArgument $\alpha$}
  & \Src{JSChan $\alpha$}
  & \Src{Chan $\HaskellAnalog{\alpha}$}
  & \\[2pt]
\hline
\end{tabular}
\end{center}
\caption{Sunroof types and their Haskell pendant.}
\label{tab:sunroof-types}
\end{table} 

The left-hand column of table~\ref{tab:sunroof-types} gives constraints,
implemented via Haskell type classes.
\begin{itemize}
\item  All our core types are brought together with the \Src{Sunroof} class.
If a type is an instance of \Src{Sunroof}, then that type can be realized inside JavaScript.
All the \Src{JS} types are instances of \Src{Sunroof}.

\item Instances of the \Src{SunroofArgument} class
are values that be passed to JavaScript functions and methods,
including tuples of values which are used for representing multi-argument 
function calls.
The asymmetry here is a reflection of the JavaScript
asymmetry inherited from C: you can pass multiple arguments
to a function, but only get a single thing back.

\item Finally, there is a third class, \Src{SunroofKey}, which is
is a JavaScript version of the Haskell \Src{Show} class,
but specifically for generating JavaScript object keys.
which are realized as JavaScript strings.
\end{itemize}

Using our \Src{JS}-types, we can build expressions using the traditional Haskell expression-build
mechanisms. \Src{JSNumber} is overloaded, and thus can be used directly for arithmetic.
\Src{JSBool} is an instance of the relevant classes in the hackage package \verb|boolean|.
Further, \Src{JSString} is a \Src{Monoid}~\cite{..}, which is only possible because JavaScript
strings are immutable values.

To help the conversion between classical Haskell values and Sunroof values,
we provide the overloaded function \Src{js}. The right-hand column of 
table~\ref{tab:sunroof-types} shows what types can be converted on the fly.
Putting the expression-building capability together, 
along with a show function called \Src{string}, we can write:
\begin{Code}
alert ("n =  " <> js (n :: Int) <> " n " <> string (m :: JSNumber))
\end{Code}
Here, \Src{n} is a Haskell \Src{Int}, and \Src{m} is a Sunroof number,
presumably the result of a previous computation. Given that we are 
bundling expressions to send to a browser for execution, \Src{n}
is a static value, and \Src{m} is a dynamic value, unobservable until inside our browser's execution.

One design decision in Sunroof is that we enforce a stronger typing than JavaScript itself would.
Specifically, there are restrictions on the type arguments of our container types,
like \Src{JSArray}. What can be seen from this is that we
enforce a Hindley–Milner style thinking to our containers,
which is distinct from JavaScript dynamic typing.
Some types involve
phantom types to give imposed type safety \cite{Cheney:03:FirstClassPhantomTypes}.
Thus, \Src{JSArray} is a restricted type of JavaScript Array, that, like Haskell,
support collections of the same type.

From experience with using Sunroof,
the mis-match in typing between Haskell and Sunroof/JavaScript
is not large problem in practice.
We also provide an explicit dynamic \Src{cast},
for use where the type-systems differ,
and both sides of the \Src{cast} are instances of \Src{Sunroof}.
\begin{Code}
cast :: (Sunroof a, Sunroof b) => a -> b
\end{Code}

\subsection{Sunroof Statements}

The building block of object-oriented programing is calling an object's method.
This is almost universally done using the dot (\Src{.}) operator. JavaScript, or target language, 
follows this trend, with the method-call syntax being as follows:
\begin{Code}
  // JavaScript
  object.method(a1,a2,a3,...,aN)
\end{Code}
For our Sunroof object model, we want to model this as close as possible.
We, by convention, have our method calls take the object as the {\em last\/} argument,
and use a monad, called the \Src{JS} monad, for the result because method calls are effectful.
\begin{Code}
// Shape of a Sunroof method
(SunroofArgument args, Sunroof res) => args -> (Object -> JS res)
\end{Code}
A neat way of writing method calls in Haskell is transcribing the JavaScript dot, which is already use
for both namespace resolution and function composition in Haskell, with the \Src{\#} combinator~\cite{Shields:01:Babel}.
The \Src{\#} combinator simply applies the second argument to the first (object) argument.
\begin{Code}
// Sunroof        
(#) :: a -> (a -> JS b) -> JS b
(#) obj act = act obj
\end{Code}

\noindent
This gives the follow fragment for a Sunroof call to \Src{method} on \Src{object}.
\begin{Code}
  object # method (a1,a2,a3,...,aN)
\end{Code}
In this way, JavaScript can be transliterated into Haskell where needed,
native JavaScript call idioms can be used,
while other Haskell abstraction mechanisms can be used for the interface that calls Sunroof code.

We piece together our Sunroof method calls using the Haskell support for monads.
\begin{Code}
  r1 <- object # method1 (a1,a2,a3,...,aN)
  r2 <- object # method2 (... can use r1 ...)
  ....
\end{Code} 
In this fragment, \Src{r1} is bound to the result of the \Src{method1} call,
and can be used as an argument to \Src{method2}.

Sometimes, a JavaScript API requires direct access to object attributes.
We do so using a typed \Src{JSSelector}.
\begin{Code}
label :: JSString -> JSSelector a
(!)   :: (Sunroof o, Sunroof a) => o -> JSSelector a -> a
(:=)  :: (Sunroof a, Sunroof o) => JSSelector a -> a -> (o -> JS t ())
\end{Code}
We can build a selector (\Src{label}), use a selector to access a attribute in a specific
object (\Src{!}), or update an attribute in a specific object (\Src{:=}).
The update is in our object-normal-form, that is the object is the final argument.
Sunroof is, in essence, a strict functional language, with monads for effect.
We choose to support direct (non-monadic) reading of attributes, but
all writing of an 
This is a design decision we may return to in the future.

The net effect is that assignments to fields can be neatly expressed,
in a syntax is this close to JavaScript itself. Our HTML5 canvas library,
build on top of Sunroof primitives, allows the following statement:
\begin{Code}
  c # color := "red"
\end{Code}


%%%%%%%%%%%%%%%%%%%%%%%%%%%%%%%%%%%%%%%%%%%%%%%%%%%%%%%%%%%%%%%%%%%%%%%%%%%%%%%%%%%%%%%%%%%%%%%%
%%%%%%%%%%%%%%%%%%%%%%%%%%%%%%%%%%%%%%%%%%%%%%%%%%%%%%%%%%%%%%%%%%%%%%%%%%%%%%%%%%%%%%%%%%%%%%%%
%%%%%%%%%%%%%%%%%%%%%%%%%%%%%%%%%%%%%%%%%%%%%%%%%%%%%%%%%%%%%%%%%%%%%%%%%%%%%%%%%%%%%%%%%%%%%%%%


\section{Threading Models}
\label{sec:threading-models}

JavaScript uses a callback-centric model of computation. There
is no support for concurrency, only a single central loop that executes
callbacks, which should be non-blocking, as events occur.
In contrast, Haskell has robust concurrency and wide-spread 
abstractions for synchronization, e.g. \Src{MVar}s and \Src{Chan}s
\cite{Jones:96:ConcurrentHaskell}.
So the question arises: do we generate atomic JavaScript code, 
and keep the callback centric model, or do we add concurrency
support as value-added of our transliteration to JavaScript?

In our earlier work~\cite{...}, we prototyped both models
of computation, and observed that both choices had poor consequences
when mapping Haskell to JavaScript.
\begin{itemize}
\item Existing JavaScript API assume atomic, non-blocking
semantics. Therefore, if we compile a higher-order function
for a callback, it must not block. 
Further, if this higher-order function returns a result, 
the function must respect this non-blocking requirement.
\item The lure of threads, and the abstractions they allow, 
is strong. The code is -- from experience -- cleaner if such abstraction
are used, rather that being faked in JavaScript itself.
There are new libraries in JavaScript that encode
concurrency~\cite{...}; Sunroof is a chance to
perform the a cooperative concurrency translation
when generating JavaScript.
\end{itemize}
In Sunroof, we decided to explicitly support both,
and make both first-class threading {\em strategies\/} in Sunroof.
This means that the programmer can choose what model fits
the idiom being coded.

In terms of user-interface, we parameterize the \JS-monad
with a phantom type that represents the threading model used, 
with \Src{A} for \Src{A}tomic,
and \Src{B} for \Src{B}locking threads. 
Atomic threads are classical JavaScript computations that
cannot be interrupted and actively use the callback
mechanism. Blocking threads can
support suspending operations and cooperative concurrency
abstractions as known from Haskell. By using phantom
types, we can express the necessary
restrictions on specific combinators, as well
as provide combinators to allow both types of
threads to cooperate.

The blocking model hides the callback mechanism behind abstractions.
This implies that every atomic computation can be converted into 
a blocking computation. \Src{liftJS} achieves this.
\begin{Code}
liftJS :: Sunroof a => JS A a -> JS t a
\end{Code}

When suspending, we register our current
continuation as a callback to resume later. This gives other 
threads (registered continuations) a chance to run.
Of course, this model depends on cooperation between the threads,
because a not terminating or suspending thread will keep others from running.

There are three main primitives for the blocking model:
\begin{Code}
forkJS      :: SunroofThread t1 => JS t1 () -> JS t2 ()
yield       :: JS B ()
threadDelay :: JSNumber -> JS B ()
\end{Code}
They can all be seen as analogues of their \IO~counterparts.
\Src{forkJS} resembles \Src{forkIO}.
It registers the given computation as a callback. 
In \Src{forkJS}, the \Src{SunroofThread} constrain allows the compiler
to know if this callback should be blocking or non-blocking,
based on the type of \Src{t1}. Both blocking and non-blocking
threads can fork new threads, so \Src{t2} is unconstrained.
\Src{yield} suspends the current thread by 
registering the current continuation as a callback,
giving other threads time to run.
\Src{threadDelay} is a form of \Src{yield} that sets 
the callback to be called after a certain amount of time.
We rely on the JavaScript function \Src{window.setTimeout} 
\cite{whatwg:timers} to register our callbacks.

Based on these primitive combinators, we also offer a Sunroof 
version of Haskell's \Src{MVar} and \Src{Chan}: \Src{JSMVar} and \Src{JSChan}.
\begin{Code}
newMVar      :: (SunroofArgument a) => a -> JS t (JSMVar a)
newEmptyMVar :: (SunroofArgument a) => JS t (JSMVar a)
putMVar      :: (SunroofArgument a) => a -> JSMVar a -> JS B ()
takeMVar     :: (SunroofArgument a) => JSMVar a -> JS B a

newChan   :: (SunroofArgument a) => JS t (JSChan a)
writeChan :: (SunroofArgument a) => a -> JSChan a -> JS t ()
readChan  :: (SunroofArgument a) => JSChan a -> JS B a
\end{Code}
Note that the types reflect if a specific operation can block.
For example, \Src{writeChan} can never block, so you can use either threading model,
but \Src{readChan} may block, so uses the \Src{B} threading model.


This parameterization of \Src{JS} allows us to use types capture
the blocking or non-blocking semantics of our primitives.
Furthermore, the Haskell type system automatically propagates
the thread semantics. In this way, Sunroof can offer concurrency
an additional, first class, abstraction.

%%%%%%%%%%%%%%%%%%%%%%%%%%%%%%%%%%%%%%%%%%%%%%%%%%%%%%%%%%%%%%%%%%%%%%%%%%%%%%%%%%%%%%%%%%%%%%%%
%%%%%%%%%%%%%%%%%%%%%%%%%%%%%%%%%%%%%%%%%%%%%%%%%%%%%%%%%%%%%%%%%%%%%%%%%%%%%%%%%%%%%%%%%%%%%%%%
%%%%%%%%%%%%%%%%%%%%%%%%%%%%%%%%%%%%%%%%%%%%%%%%%%%%%%%%%%%%%%%%%%%%%%%%%%%%%%%%%%%%%%%%%%%%%%%%

\section{Functions and Continuations}
\label{sec:functions-continuations}

Functions are first class values in Haskell and JavaScript.
Sunroof represents function {\em objects\/} with the type 
\Src{JSFunction $\alpha$ $\beta$}, which resembles 
a function $\alpha \rightarrow \beta$ in JavaScript.


Rather than directly support partial application,
and transliterate curryed Haskell into uncurryed JavaScript,
we only permit uncurryed functions.
Like JavaScript, currying can be encoded if needed.
We introduced \Src{SunroofArgument},
a superclass of \Src{Sunroof}, to constrain the 
types that may be used as arguments for functions.
\begin{Code}
class SunroofArgument args where
  jsArgs   :: args -> [Expr]
  jsValue  :: (UniqM m) => m args
  ...
\end{Code}
This class supports converting each argument into its expression
through \Src{jsArgs} to supply the
arguments to a function call. \Src{jsValue} generates
unique names for each argument, which is needed when compiling
the function itself to a value. 

Now we can provide more then one argument
by providing \Src{SunroofArgument} instances for tuples of Sunroof values.
\begin{Code}
instance (Sunroof a, Sunroof b) => SunroofArgument (a,b) where
  jsArgs (a,b) = [unbox a, unbox b]
  jsValue = liftM2 (,) jsVar jsVar
\end{Code}
Remember that each \Src{Sunroof} value already has to be 
a \Src{SunroofArgument}, which enables us to pass a single argument
to a function. 

We can create function objects with the \Src{function} combinator.
\begin{Code}
function :: (SunroofArgument a, Sunroof b) 
         => (a -> JS A b)  -> JS t (JSFunction a b)
\end{Code}
As a function can have side-effects, its computation and result have to be 
expressed in the \JS-monad. The creation of a function is considered 
a side-effect, due to observable allocation.

Function application is done through 
the \Src{apply} or \Src{\$\$} combinator, they are synonyms.
Functions can only be applied in the \JS-monad, since they can have side-effects.
\begin{Code}
apply, ($$) :: (SunroofArgument a, Sunroof b) 
            => JSFunction args ret -> args -> JS t ret
\end{Code}
Creation and application are implemented using the \Src{JS\_Function}
and \Src{JS\_Invoke} instructions introduced in 
\FigRef{fig:jsi-definition}.

\Src{JSContinuation $\alpha$} is used to model continuations in the \JS-monad. 
It was introduced to work with continuations in Sunroof. 
Technically they are only
specializations of functions, but
restricted to a specific threading model. Continuations
are meant to be a representation of side-effects -- 
ongoing computations inside the \JS-monad -- and may  
not terminate, so they do not return a value. As with functions, 
there is a combinator to create and apply a continuation.
\begin{Code}
continuation :: (SunroofArgument a) 
             => (a -> JS B ()) -> JS t (JSContinuation a)
goto         :: (SunroofArgument a) 
             => JSContinuation a -> a -> JS t b
\end{Code}
The presented \Src{goto} should not be considered 
harmful \cite{Dijkstra:68:GotoConsideredHarmful}.
It calls a continuation,
as \Src{apply} calls functions.
The difference is that a call to \Src{goto} will never
return, as it executes the given continuation and abandons the 
current one. This means \Src{goto} is fully polymorphic
on its (unused and unusable) result.

Access to the current continuations is given through
the powerful call-with-current-continuation 
combinator \Src{callcc}.
\begin{Code}
callcc :: SunroofArgument a 
       => (JSContinuation a -> JS B a) -> JS B a
\end{Code}
The current continuation models everything that 
would usually happen after the call to \Src{callcc}. 
\begin{Code}
callcc f = JS $ \ k -> unJS 
    (continuation (\a -> JS $ \ _ -> k a) >>= f) k

unJS :: JS t a -> (a -> Program (JSI t) ()) -> Program (JSI t) ()
\end{Code}
The implementation of \Src{callcc} is interesting,
because it shows how the \Src{Program}-continuation is translated 
into a \Src{JSContinuation} that is passed to the given function \Src{f}.
\SecRef{sec:threading-models} will show why this function
is important for Sunroof and what it is used for.

Functions and continuations are similar and connected 
to each other, as can be seen in \FigRef{fig:func-cont}.
\Figure%
{fig:func-cont}%
{figures/sunroof-func-cont.pdf}%
{How functions and continuations relate between the Haskell and Sunroof domain.}%
We can go back and forth between the Haskell and the Sunroof
representation of a function or continuation. But once a function
is specialized to a continuation, it is not possible to go back,
because continuations only model the side-effect, but do 
not return anything.


%%%%%%%%%%%%%%%%%%%%%%%%%%%%%%%%%%%%%%%%%%%%%%%%%%%%%%%%%%%%%%%%%%%%%%%%%%%%%%%%%%%%%%%%%%%%%%%%
%%%%%%%%%%%%%%%%%%%%%%%%%%%%%%%%%%%%%%%%%%%%%%%%%%%%%%%%%%%%%%%%%%%%%%%%%%%%%%%%%%%%%%%%%%%%%%%%
%%%%%%%%%%%%%%%%%%%%%%%%%%%%%%%%%%%%%%%%%%%%%%%%%%%%%%%%%%%%%%%%%%%%%%%%%%%%%%%%%%%%%%%%%%%%%%%%

% 
\section{The JavaScript Monad}
\label{sec:js-monad}

The \JS-monad is
used to model sequences of JavaScript statements and
their side-effects. In that sense it is almost
exact analog for the Haskell \IO-monad, except there
is an extra phantom argument~\cite{Leijen:99:Phantom} 
that decides which threading model to use, as
discussed in Section \ref{sec:threading-models}.
For now we can ignore this extra argument.

The basic idea is that each binding becomes an
assignment to a fresh variable in JavaScript. Like
this the results of previous computations are passed on to 
later ones.
An example for this can be seen in Figure~\ref{fig:code-example}.
The binding \Src{name} is translated to the freshly generated
variable \Src{v0}.

Inside this simple example is a challenging problem -- where does
\Src{v0} come from? The bind inside the monadic \Src{do} is
unconstrained:
\begin{verbatim}
(>>=) :: JS t a -> (a -> JS t b) -> JS t b
\end{verbatim}
What we want is:
\begin{verbatim}
(>>=) :: (Sunroof a) => JS t a -> (a -> JS t b) -> JS t b
\end{verbatim}
Where \Src{Sunroof} constrains the bind to
arguments for which we can generate a JavaScript variable.
Counterintuitively, 
it turns out that a specific form of normalization allows 
the ``\Src{a}'' type to be constrained and the bind to 
be an instance of the standard monad class~\cite{Sculthorpe:13:ConstrainedMonads}.
Through this keyhole of {\em monadic reification},
the entire Sunroof language is realized. The 
normalization is done through the Operational package 
\cite{Apfelmus:10:Operational,Hackage:10:Operational}.

It provides the \Src{Program} data type as a monad that
can be equipped with custom primitives or effects.
We represent these primitives with the \JSI~instruction 
(Figure \ref{fig:jsi-definition}) type. It represents the 
abstract JavaScript instructions that are sequenced by the \Src{Program}
monad.
\begin{figure}
\begin{verbatim}
data JSI :: T -> * -> * where
  JS_Invoke   :: (SunroofArgument a, Sunroof r) 
              => a -> JSFunction a r -> JSI t r
  JS_Function :: (SunroofArgument a, Sunroof b) 
              => (a -> JS A b) -> JSI t (JSFunction a b)
  JS_Branch   :: (SunroofThread t, SunroofArgument a, Sunroof bool) 
              => bool -> JS t a -> JS t a  -> JSI t a
  JS_Assign   :: (Sunroof a) 
              => JSSelector a -> a -> JSObject -> JSI t ()
  ...
\end{verbatim}
\caption{Parts of the \JSI nstruction data type.}
\label{fig:jsi-definition}
\end{figure}
The type parameter \Src{t} in \Src{JSI t a} again represents 
the threading model and can be ignored up to Section \ref{sec:threading-models}. 
\Src{a} represents the type of the primitive's return value. 
Figure \ref{fig:jsi-definition} shows some of the instructions
in \JSI.
\Src{JS\_Invoke} calls a function, that has been created with \Src{JS\_Function}.
Branches are represented with \Src{JS\_Branch}. Assignments to variable
are represented by \Src{JS\_Assign\_}.

If it was not for our promise to provide an alternative 
threading model, the normalization through the Operational package
would be enough to offer a monad suitable for Sunroof.
But, because of our threading plans, there is more then just 
normalization going on behind the scenes. 
Actually the \JS-monad is a continuation monad over the 
\Src{Program}-monad.
\begin{verbatim}
data JS :: T -> * -> * where
  JS :: ((a -> Program (JSI t) ()) -> Program (JSI t) ()) -> JS t a
  ...
\end{verbatim}
The monad instance used is the standard implementation of 
a continuation monad (TODO: cite).


TODO: Maybe put this section after the section about the object model.

\begin{comment}
\subsubsection{RESOURCES - REMOVE WHEN FINISHED}

Basic Idea
\begin{itemize}
\item Put the current stuff into the introduction does not belong here
\item \JS-monad is analog of \IO-monad
\item Monad used to model sequences of statements with side-effect in 
JavaScript \cite{Moggi:91:ComputationMonads}
\item binding in Haskell becomes binding in JavaScript (more about this in section \ref{sec:compiler})
\item There also is a expression level (discussed in section \ref{sec:object-model})
\item Feels like a native monad, cf STM.
\item Can build abstractions on top of this.
\end{itemize}
Implementation
\begin{itemize}
\item Problems when using monad: Types need to be constrained to work
\item We use monad reification \cite{Apfelmus:10:Operational,Hackage:10:Operational}
\item But \JS-monad is a little bit more complicated
\item Show \JS-constructor: \Src{JS :: ((a -> Program (JSI t) ()) -> Program (JSI t) ()) -> JS t a}
\item Whats going on here?
\item \JS-monad is a continuation monad on the \Src{Program} type from operational
\item Monad instance is the standard implementation for a continuation 
monad (TODO: cite for continuation monad)
\item This is necessary, because we want to support different threading styles 
(discussed in chapter \ref{sec:threading-models})
\item For now think of it as producing a sequence/list of abstract instructions: \JSI
\item Look at the \JSI\ type and introduce it
\item Explain how it is a high-level representation of JavaScript
\item TODO: Look at all constructors? Or only the interesting ones (marked with "!")?
  \begin{itemize}
  \item ! Function and continuation are technically the same, but different instructions
  \item ! Reason for this is TODO?
  \item ! Why are there two different assignments?
  \item ! One is assignment to variable, other is assignment to the field of an object
  \item ! Eval, this looks weird. Forces evaluation and binding to a variable.
  \item ! Otherwise, we would use macro semantics in many places.
  \item ! Fix is provided to be able to write recursive functions
  \item ! Why isn't that possible without fix? problem boils down to 
        not observable sharing between statements and bindings. This
        leads compiler into infinite loops.
  \item Select selects a field
  \item Delete remove a field
  \item Invoke call a function/method
  \item Return for return of a function
  \item Comment writes some comment, helps for debugging
  \end{itemize}

\end{itemize}
\end{comment}





% 
\section{Foreign Function Interface}
\label{sec:ffi}

Sunroof also offers a 
foreign function interface, which enables us to easily 
access predefined JavaScript. There are four core functions:
\begin{Code}
fun    :: (SunroofArgument a, Sunroof r) 
       => String -> JSFunction a r
object :: String -> JSObject
new    :: (SunroofArgument a) 
       => String a -> JS t JSObject
invoke :: (SunroofArgument a, Sunroof o, Sunroof r) 
       => String -> a -> o -> JS t r
\end{Code}
\Src{fun} is used to create Sunroof functions from their names in JavaScript.
This can happen in two ways: either to call a function in line, or to 
create a real binding for that function. As an example, 
the \Src{alert} function can be called in line through \Src{fun "alert" \$\$ "text"},
or you can provide a binding in form of a Haskell function for it.
\begin{Code}
alert :: JSString -> JS t ()
alert s = fun "alert" $$ s
\end{Code}

Objects can be bound through the \Src{object} function, e.g.
the \Src{document} object is bound through \Src{object "document"}.
Constructors can be called using \Src{new}. To create a new
object you would call \Src{new "Object" ()}.

We can call methods of objects through \Src{invoke}. Again, this 
can be used in line and to create a real binding. An inline 
use of this to produce \Src{document{\linebreak}.getElementById("id")} would look like this: 
\begin{Code}
object "document" # invoke "getElementById" "id"
\end{Code}
where \Src{\#} is just a flipped function application. To provide a binding 
to the \Src{getElementById} method, one can write:
\begin{Code}
getElementById :: JSString -> JSObject -> JS t JSObject
getElementById s = invoke "getElementById" s
\end{Code}

Providing actual bindings ensures that
everything is typed correctly and prevents the need to resolve ambiguities 
through large type annotations inside of code.

The current release of Sunroof already provides bindings for most of the 
core browser API, the HTML5 canvas element, and some of the JQuery API.

\begin{comment}
Table \ref{tab:ffi} gives an overview of
how Sunroof's FFI can be used. 
\begin{table}
\begin{center}
\begin{tabular}{l@{\quad}p{7cm}}
  \hline\rule{0pt}{12pt}%
  JavaScript & 
  Sunroof \\ \hline\rule{0pt}{12pt}%
%
  \Src{alert("Test");} & 
  \Src{fun "alert" \$\$ "Test"} \\[2pt]
%
  \Src{alert} as a Sunroof function & 
  \Src{alert :: JSFunction JSString ()\newline alert = fun "alert"} \\[2pt]
%
  \Src{alert} as a Haskell function & 
  \Src{alert :: JSString -> JS t ()\newline alert s = fun "alert" \$\$ s} \\[2pt]
%
  \Src{document.getElementById("id");} & 
  \Src{object "document" \# invoke "getElementById" "id"} \\[2pt]
%
  \Src{getElementById} as method & 
  \Src{getElementById :: JSString -> JSObject -> JS t JSObject\newline
       getElementById s = invoke "getElementById" s} \\[2pt]
\hline
\end{tabular}
\end{center}
\caption{JavaScript expressed through the Sunroof FFI.}
\label{tab:ffi}
\end{table} 
\end{comment}
%

% 
\section{The Sunroof Compiler}
\label{sec:compiler}

Given the language, and monadic-reification, how do we compile this language?
Through the \JS-monad we produce a \Src{Program (JSI t) ()} instance. We 
translate such a program into a list of statements (\Src{Stmt}) by matching over 
the \JSI constructors.
\begin{Code}
data Stmt 
  = AssignStmt Rhs Expr       -- Assignment
  | DeleteStmt Expr           -- Delete reference
  | ExprStmt Expr             -- Expression as statement
  | ReturnStmt Expr           -- Return statement
  | IfStmt Expr [Stmt] [Stmt] -- If-Then-Else statement
  | WhileStmt Expr [Stmt]     -- While loop
  | CommentStmt String        -- Comment
\end{Code}
The constructors of \Src{Stmt} are straight forward and
directly represent the different statements one can write
in JavaScript.

To translate a \Src{Program (JSI t) ()} into statements we
have to translate it into a \Src{ProgramView} since 
Operational prohibits us from working on \Src{Program} directly.
\Src{ProgramView} can then be taken apart to translate 
the \JSI nstructions inside.
\begin{Code}
compile :: Program (JSI t) () -> CompM [Stmt]
compile p = eval (view p)
  where
    eval :: ProgramView (JSI t) () -> CompM [Stmt]
    ...
\end{Code}
The compilation monad \Src{CompM} provides us with 
the compiler options and a supply of fresh integer
values to generate new variable when needed.

To get a feel for how the compiler works, lets take a look at 
the case for \Src{JS\_Assign}. Recall its signature from 
Figure \ref{fig:jsi-definition}:
\begin{Code}
JS_Assign :: Sunroof a => JSSelector a -> a -> JSObject -> JSI t ()
\end{Code}
This instruction assigns a new value \Src{a} to the attribute
selected by the \Src{JSSelector a} of the \Src{JSObject}.
\begin{Code}
eval (JS_Assign sel a obj :>>= k) = do
  (stmts0,val) <- compileExpr (unbox a)
  stmts1 <- compile (k ())
  return (  stmts0 
         ++ [ AssignStmt (DotRhs (unbox obj) 
                                 (unboxSelector sel)) 
                         val ] 
         ++ stmts1 )
\end{Code}
First we compile the expression that will be assigned. That
produces a series of statements to compute the expression and 
the expressions value \Src{val}.
Then we compile the rest of the Sunroof computation \Src{k}. Since 
the assignment does not produce a interesting value (it has type \Src{JSI t ()}),
we can pass unit to \Src{k}.
We return the concatenation of the produced statements with a statement 
for the assignment in between. The \Src{unbox} functions reveal the 
\Src{Expr} inside of our Sunroof wrapper types.

Lets now look at a more interesting part of the compiler: The translation of 
branches.
Recall the \JSI constructor for branches from Figure \ref{fig:jsi-definition}.
\begin{Code}
JS_Branch :: (SunroofThread t, SunroofArgument a, Sunroof bool) 
          => bool -> JS t a -> JS t a  -> JSI t a
\end{Code}
Aside of the fact that branches may return \Src{SunroofArgument}s
instead of just \Src{Sunroof} values the translation seems to 
be straight forward (Figure \ref{fig:branch-translation}).
\begin{figure}[h]
\begin{Code}
eval (JS_Branch b c1 c2 :>>= k) = do
  (src0, res0) <- compileExpr (unbox b)
  res :: a <- jsValue
  let bindResults :: a -> JS t ()
      bindResults res' =
        sequence_ [ single $ JS_Assign_ v (box $ e :: JSObject)
                  | (Var v, e) <- jsArgs res `zip` jsArgs res' ]
  src1 <- compile $ extractProgramJS bindResults c1
  src2 <- compile $ extractProgramJS bindResults c2
  rest <- compile (k res)
  return (src0 ++ [ IfStmt res0 src1 src2 ] ++ rest)
\end{Code}
\caption{Naive translation of branches in Sunroof.}
\label{fig:branch-translation}
\end{figure}
First we generate the statements to compute the branching condition.
Then we generate unique names for each of the values returned in 
either branch. \Src{bindResults} generates the assignments of
the returned values to the generated variables. When compiling 
the branches we use the function \Src{extractProgramJS}.
\begin{Code}
extractProgramJS :: (a -> JS t ()) -> JS t a -> Program (JSI t) ()
extractProgramJS k m = unJS (m >>= k) return
\end{Code}
It passes the result of a computation \Src{m} 
into the given function \Src{f} and
closes the continuation inside \JS~with \Src{return},
such that the result is a \Src{Program} containing
all instructions of \Src{m >>= k}.

After we have compiled both branches we compile the rest and
construct the list of statements in a canonical fashion.

This works perfectly when we work in the atomic threading model,
but it causes problems when inside the blocking model. To 
see what can go wrong lets look at a small example.
\TODO{Explain callcc in continuation section}
\begin{Code}
branchFail = do
  b <- ifB (true :: JSBool)
           (callcc $ \k -> do comment "True Case"
                              goto k true :: JSB JSBool)
           (callcc $ \k -> do comment "False Case"
                              goto k false :: JSB JSBool)
  fun "CallAfterIf" $$ b
\end{Code}
In either case of the initial branch we want to do something
with our current continuation. We insert a comment for 
our orientation and then call our current continuation to 
proceed with a boolean value that depends on the branch.
After our branch we call a function with the returned boolean value.
\TODO{Is there a simpler example that demonstrates the problem?}
What will this be translated to?
\begin{Code}
var v8 = function() {
  if(true){
    var v2 = function(v1) {
      var v0 = v1;
    };
    /* True Case */
    v2(true);
  } else {
    var v5 = function(v4) {
      var v0 = v4;
    };
    /* False Case */
    v5(false);
  }
  CallAfterIf(v0);
};
v8();
\end{Code}
We can see the branching statement and the function call afterwards.
Inside each branch we can also see a function definition the inserted 
comment and a call to the defined function. The function definitions 
\Src{v2} and \Src{v5} correspond to the continuations \Src{k} in 
our \Src{callcc} calls. So we can see that the current continuations 
captured the assignments that have to be done at the end of either branch 
to pass on the result value. But the assignments are hidden inside 
the scope of the continuations and, because of that, \Src{v0} is 
not defined when \Src{CallAfterIf} is applied to it.

So how can we fix this? If we are working within the blocking 
threading model, we have to handle branches differently.
\begin{Code}
eval (JS_Branch b c1 c2 :>>= k) = 
  case evalStyle (ThreadProxy :: ThreadProxy t) of
    A -> compileBranch_A b c1 c2 k
    B -> compileBranch_B b c1 c2 k
\end{Code}
\Src{evalStyle} is provided by the class \Src{SunroofThread} and
provides the threading model currently used. The call to \Src{compileBranch\_A}
executes our naive definition from Figure \ref{fig:branch-translation}.
\begin{Code}
compileBranch_B b c1 c2 k = do
  fn_e <- compileContinuation $
            \a -> blockableJS $ JS $ \k2 -> k a >>= k2
  fn <- newVar
  (src0, res0) <- compileExpr (unbox b)
  src1 <- compile $ extractProgramJS (apply (var fn)) c1
  src2 <- compile $ extractProgramJS (apply (var fn)) c2
  return ( [mkVarStmt fn fn_e] ++ src0 ++ [ IfStmt res0 src1 src2 ])
\end{Code}
\TODO{Talk about \Src{blockableJS}}
We can see that the rest of our computation is captured in a
continuation \Src{fn\_e} that takes the results of our branching
as arguments. We assign that computation to the new variable \Src{fn}
to avoid the duplication of the continuation in the produced sources.
The key difference lays in the call to \Src{extractProgramJS}. Instead
of creating bindings for each returned value we apply the produced
continuation to the returned values and like that make them visible for
the ongoing computation, even if we capture our current continuation. 
Now the produced code looks like this:
\begin{Code}
var v11 = function() {
  var v2 = function(v0) {
    CallAfterIf(v0);
  };
  if(true){
    var v5 = function(v3) {
      v2(v3);
    };
    /* True Case */
    v5(true);
  } else {
    var v9 = function(v7) {
      v2(v7);
    };
    /* False Case */
    v9(false);
  }
};
v11();
\end{Code}
We can see that the function call after the branch is captured by the 
continuation \Src{v2} and that continuation is called with the 
return values as parameters inside each branch.

Of course, we could compile every branch with the 
continuation variant. But that would unnecessarily 
obfuscate the produced code and lead to many superfluous 
indirections.
\TODO{Look at another interesting part: Fix? Function?}

The compiler can be called through two functions. 
One for each threading model. \TODO{Maybe remove the string argument from JSB}
\begin{Code}
sunroofCompileJSA :: Sunroof a 
                  => CompilerOpts -> String -> JS A a  -> IO String
sunroofCompileJSB :: CompilerOpts -> String -> JS B () -> IO String
\end{Code}

\begin{comment}
Given the language, and monadic-reification, how do we compile this language?
Figure \ref{fig:structure} shows how Sunroof is structured.
On the lowest level we provide an untyped expression language \Src{Expr}
that describes JavaScript expressions. 
To provide type safety when using Sunroof we create
wrappers for each type we want to represent, e.g. \Src{JSNumber} or \Src{JSString}.
The \Src{Sunroof} type class provides an 
interface to create wrapped and unwrapped
instances of our expressions. Based on the wrappers we can provide 
operations specific to a certain type, e.g. a \Src{Num} instance
for \Src{JSNumber} or a \Src{Monoid} instance for \Src{JSString}.



This technique enables us to utilize  Haskells type system when writing JavaScript
and offers an easy way to add new types when needed~\cite{Svenningsson:12:CombiningEmbedding}.
By using phantom types we can also provide more advanced types,
like \Src{JSArray a}.

The next layer provides JavaScript instructions through the type \JSI.
They represents abstract statements. While expressions and values
represented with type wrappers are assumed to be free of side-effects,
the instructions model side-effects in JavaScript. Examples for Instructions
are assignment of an attribute or the application of a function.
\begin{Code}
data JSI :: T -> * -> * where
  JS_Assign :: (...) => JSSelector a -> a -> JSObject -> JSI t ()
  JS_Invoke :: (...) => a -> JSFunction a r -> JSI t r
  JS_Branch :: (...) => bool -> JS t a -> JS t a  -> JSI t a
  ...
\end{Code}
The \JS-monad with its combinators builds a sequence of 
\JSI{}nstructions through the operational
package~\cite{Hackage:10:Operational,Apfelmus:10:Operational}.
All constraints required on instructions are introduced by their 
constructors.
As mentioned earlier the \JS-monad comes in two threading flavors, 
parameterized using a phantom type.
Internally the \JS-monad is implemented using CPS on the 
underlying \Src{Program} type from Operational. 
\begin{Code}
data JS :: T -> * -> * where
  JS   :: ((a -> Program (JSI t) ()) -> Program (JSI t) ()) 
       -> JS t a
  ...
\end{Code}
For atomic
computations we just produce a list of instructions from the continuation. 
When translating possibly blocking code we directly translate that continuation
into JavaScript functions. This gives us the ability to handle 
computations as values in JavaScript and store them if needed.

Blocking operations just store the rest of their computation in a queue.
When the event to unblock occurs the pending computation is registered 
as a callback that will be executed as soon as the current computation
is done.

On top of our \JS-monad we provide ways of specifying (typed)
interfaces to JavaScript capabilities, a Foreign Function Interface.
\begin{Code}
alert :: JSString -> JS t ()
alert = fun "alert"

getElementById :: JSString -> JSObject -> JS t JSCanvas
getElementById = invoke "getElementById"
\end{Code}
Notice, that calling a JavaScript function or method is done by giving its
name to one of the provided combinators. Types can be specialized using 
a Haskell type annotation. A flexible and easy to use approach.
\end{comment}

\begin{comment}
\subsubsection{RESOURCES - REMOVE WHEN FINISHED}

Compiling
\begin{itemize}
\item Short introduction to the compiler interface (signature).
\item Core work done by translating \Src{Program (JSI t) ()}
into a list of \Src{Stmt}s.
\item Introduce statement type, give a short description of 
each constructor (just in the comments)
\item Basic idea: Each \JSI~nstruction is translated 
into a sequence of statements and these are then
concatenated together.
\item Look at interesting parts
\item \TODO{Which parts are interesting? Most parts are too technical}
\item Show how a branch is compiled \TODO{\Src{extractProgramJS} lets things look messy}
\item Talk about how \Src{JS\_Fix} is compiled.
\item \TODO{Understand why fix works.}
\item Difference between compilation of a function and a continuation
\item Function \Src{\textbackslash a -> JS \$ \textbackslash k -> singleton (JS\_Return a) >>= k}
\item \TODO{Why \Src{>>= k}?}
\item Return the result of the current continuation
\item Continuation \Src{\textbackslash \_ -> JS \$ \textbackslash k -> k ()}
\item Just execute it instead of passing it on further. No return value!
\item We use expression sharing through observable sharing (reference Andys paper).
\end{itemize}

\begin{itemize}
\item If we transliterate, we have straight line code, can not pause.
    (Wait for Mvar, for example)
\item If we CPS translate, we can use continuations to capture the
   notion of a paused thread. Works well.
   Problems:
  \begin{itemize}
   \item Can not translate functions, how do they get there return value
   \item (Assumes straight line code.)
   \item Also, the code becomes unreadable to anyone except a die-hard 
       compiler freak.
  \end{itemize}
\end{itemize}
 
Choice:
\begin{itemize}
\item We support both!
\item Phantom argument to JS
\item A = Atomic, B = Blockable.
\end{itemize}
\end{comment}






% 
\section{The Sunroof Server}
\label{sec:server}

The Sunroof compiler can compile JavaScript than can be used
stand-alone inside a web application. But Sunroof really comes
into its own when used with the Sunroof server. There
are three major functions in our server.

\begin{verbatim}
sunroofServer :: SunroofServerOptions -> SunroofApp -> IO ()
syncJS  :: SunroofResult a => SunroofEngine 
        -> JS t a -> IO (ResultOf a)
asyncJS :: SunroofEngine -> JS t () -> IO ()
\end{verbatim}        

\Src{sunroofServer} starts a small web server,
that calls the callback function for each request.
\Src{syncJS} and \Src{asyncJS} allow the Sunroof programmer
to remotely execute monadic JavaScript from the server.
\Src{ResultOf} is a type-function, that maps the 
Sunroof type to a corresponding Haskell type.






% 
\section{Case Study - A Small Calculator}
\label{sec:extended-example}

To see how Sunroof works in practice, we will look into the 
experience we gathered when writing a small calculator
for arithmetic expressions (\FigRef{fig:example-application}). 
We use Sunroof to display our interface
and the results of our computation. Haskell will be used to parse the 
arithmetic expressions and calculate the result. The Sunroof server 
will be used to implement this JavaScript/Haskell hybrid.

\FigureS%
{fig:example-application}%
{figures/example-application.png}%
{The example application running on the Sunroof server.}%
{scale=0.6}

The classical approach to develop an application like this would have 
been to write a server that provides a RESTful interface and replies 
through a JSON data structure. 
The client side of that application would have been written in JavaScript
directly.
This can be seen in \FigRef{fig:example-structure}.

\Figure%
{fig:example-structure}%
{figures/example-structure.pdf}%
{Classical structure and Sunroof structure of a web application.}

How does Sunroof improve or change this classical structure?
First of all, in Sunroof you write the client-side code together with
your server application within Haskell. In our example, all code 
for the server and client is in Haskell. The control logic 
for the client side is provided through the server.
This leads to a tight coupling between both sides. 
This also shows how Sunroof blurs the border between the server 
and client side. You are not restricted by an interface or language 
barrier. If you need the client to do something, you can just 
send arbitrary Sunroof code to execute in the client.

\TabRef{tab:example-statistics} contains a few statistics 
about the size of the code in each part of the client.

The client-server response loop shuffles new input to the server 
and executes the response in the client.
%
Data conversion is needed, because pure Haskell data types
cannot be handled in Sunroof and vice versa. There still
exists a language barrier between JavaScript and Haskell. 
Code to convert between two essentially equal data structures on 
each side must be written, as well as representations of Haskell 
structures in Sunroof. However, there is great potential in automatically 
generating this code using techniques such as template Haskell
\cite{Sheard:02:TemplateMetaProgrammingHaskell}.

The code for displaying the results is basically a 
transliteration of the JavaScript that you would write for this 
purpose.
The transliteration used here is not very appealing. 
In the future, this code can be generated through higher-level 
libraries. Sunroof is intended to deliver a foundation for
this purpose.

The rest of our code to parse the arithmetic expression and calculate 
results is classical Haskell code. 

\begin{table}[t]
\begin{center}
\begin{tabular}{l@{\quad}r@{\quad}r}
\hline\rule{0pt}{12pt}%
Part of Application & Lines of Code & Percentage \\[2pt]
\hline\rule{0pt}{12pt}%
Response loop & 25 & 6.5\% \\[2pt]
Data conversion & 85 & 22.0\% \\[2pt]
Rendering & 190 & 49.5\% \\[2pt]
Parsing and interpretation & 85 & 22.0\% \\[2pt]
\hline
\end{tabular}
\end{center}
\caption{Lines of code needed for the example.}
\label{tab:example-statistics}
\vspace{-0.5cm}
\end{table} 


%\TODO{Should we add the discussion why we choose a deep embedding somewhere?}

% 
\section{Related Work}

There have been several attempts to translate Haskell to JavaScript.
Prominent ones are the compiler backends for 
UHC \cite{Stutterheim:12:ImprovingUHCJavaScriptBackend} and 
GHCJS \cite{project:ghcjs}. There are also projects like Fay \cite{project:fay} 
that compile subsets of Haskell to JavaScript or JMacro \cite{project:jmacro}
which use quasiquotation \cite{Mainland:07:QuasiquotingHaskell} to embed 
a custom-tailored language into Haskell code.

At the same time there are also projects like 
CoffeeScript \cite{project:coffeescript} or LiveScript \cite{project:livescript}
to build custom languages 
that are very similar to JavaScript but add convenient syntax and
support for missing features.

Our approach to cooperative concurrency through continuations in JavaScript has
has been used before 
\cite{Cooper:07:LinksWebProgrammingTiers,Predescu:02:CocoonContinuationBasedControlFlow}.
To our knowledge, creating a direct connection
between Haskell and JavaScript continuations has not been 
attempted before.

Deep embeddings of monads based on data structures have been used before
in Unimo \cite{Lin:06:Unimo} and Operational \cite{Apfelmus:10:Operational,Hackage:10:Operational}. 
The specific approach Sunroof takes 
by using GADTs has been discussed by 
Sculthorpe et al. \cite{Sculthorpe:13:ConstrainedMonads} 
in detail.

The Sunroof server does not have the aim to provide a full-featured 
web framework, as HAppS, Snap or Yesod do. It only provides 
the infrastructure to communicate with the currently calling website
through the Kansas comet \cite{project:kansas-comet} 
push mechanism \cite{pattern:push}. Although all of the
frameworks mentioned above would be able to implement this technique,
to our knowledge, none of them has yet.

To our knowledge, Sunroof is the only library that supports 
generation of JavaScript inside of Haskell using pure Haskell
in a type-safe manner. All other approaches discussed above
either require a separate compilation step or introduce new 
syntax inside of Haskell.

There is an effort to generalize Active \cite{project:active}, a library for animations, and
implement a backend based on Sunroof \cite{project:sunroof-active}.





% 
\section{Conclusion}

We can see that Sunroof took the key idea of monad reification and
sucessfully created the \JS-monad to describe computations
in JavaScript. This work was mainly already done by Farmer and
Gill \cite{Farmer:12:WebDSLs} and has been streamlined during the 
further development of Sunroof. By adding the concept 
of \Src{JSFunction} and \Src{JSContinuation} there now is a connection between
functions as values in the JavaScript and the Sunroof language space 
(Figure \ref{fig:func-cont}). It is possible to go back and forth between 
both worlds. Combining both concepts, functions and the \JS-monad,
we were able to create a second implementation of the monad. This
time based on the direct translation of continuations from Haskell
to JavaScript. It enabled us to build a blocking threading model
on top of JavaScript that resembles the model already known from Haskell.
Based on this model and the provided abstraction over continuations
we can use synchronization primitives like \Src{forkJS} or \Src{yield}.
Higher level abstractions like \Src{JSMVar} and \Src{JSChan} are also
available. 


\section{Acknowledgment}

We want to thank Conal Elliott for his support in adapting 
the Boolean package \cite{project:boolean} and helping us to
extend it with support for deeply embedded numbers.

%
% ---- Bibliography ----
%
\bibliographystyle{splncs03}
\bibliography{sunroof}
\vspace{-0.5cm} % THIS SAVED MY LIFE

\end{document}
