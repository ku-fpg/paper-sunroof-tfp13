% This is LLNCS.DEM the demonstration file of
% the LaTeX macro package from Springer-Verlag
% for Lecture Notes in Computer Science,
% version 2.4 for LaTeX2e as of 16. April 2010
%
\documentclass{llncs}
%
\usepackage{amsfonts}
\usepackage{comment}
\usepackage{graphicx}
\usepackage{caption}
\usepackage{subcaption}
\usepackage{verbatim}
\usepackage{url}

\newcommand{\SunroofAnalog}[1]{#1\ensuremath{_\downarrow}}
\newcommand{\HaskellAnalog}[1]{#1\ensuremath{_\uparrow}}

%\newcommand{\NOTE}[1]{{\Large\textbf{NOTE:}\ #1}}
\newcommand{\TODO}[1]{{(\textbf{TODO:}\ #1)}}
\newcommand{\Src}[1]{{\tt{#1}}}

\newcommand{\IO}{\Src{IO}}
\newcommand{\JS}{\Src{JS}}
\newcommand{\JSI}{\Src{JSI}}
\newcommand{\JSA}{\ensuremath{\Src{JS}_\Src{A}}}
\newcommand{\JSB}{\ensuremath{\Src{JS}_\Src{B}}}

\newcommand{\Figure}[3]{%
\FigureS{#1}{#2}{#3}{scale=0.55,clip=true,trim=0.45cm 0.45cm 0.45cm 0.45cm}
}

\newcommand{\FigureS}[4]{%
\begin{figure}[t]%
%\vspace{-0.5cm}%
\begin{center}%
\includegraphics[#4]{#2}%
\vspace{-0.5cm}%
\end{center}%
\caption{#3}%
\label{#1}%
\vspace{-0.5cm}%
\end{figure}%
}

\newenvironment{Code}{\verbatim}{\endverbatim}

\newcommand{\FigRef}[1]{Fig.~\ref{#1}}
\newcommand{\SecRef}[1]{Section~\ref{#1}}
\newcommand{\TabRef}[1]{Table~\ref{#1}}

% intentional for the referee's copy
\pagestyle{plain}
\newcommand{\CURSOR}{\noindent\rule{\textwidth}{4pt}}

\begin{document}
%
\title{Sunroof: A Monadic DSL for Generating JavaScript}
%\subtitle{}
%
\titlerunning{Sunroof}  % abbreviated title (for running head)
%                                     also used for the TOC unless
%                                     \toctitle is used
%
\author{Andy Gill\inst{1} and Jan Bracker\inst{1,2}}
%
\authorrunning{Jan Bracker \and Andy Gill} % abbreviated author list (for running head)
%
%%%% list of authors for the TOC (use if author list has to be modified)
\tocauthor{Andy Gill, Jan Bracker}
%
\institute{%
ITTC / EECS \\
The University of Kansas, Lawrence, KS 66045\\
~\\
\and
Institut f{\"u}r Informatik\\
Christian-Albrechts-Universit{\"a}t, Kiel, Germany}

\maketitle

\begin{abstract}        
Sunroof is a Haskell-hosted Domain Specific Language (DSL) for generating JavaScript.
Sunroof is built on top of the JavaScript monad, which, like the Haskell \IO-monad, allows 
access to external resources, but specifically JavaScript
resources. As such, Sunroof is primarily a feature-rich 
foreign-function API to the browser's JavaScript engine, and all the browser-specific
functionality, including HTML-based rendering, event handling, and 
drawing to the HTML5 canvas. 

In this paper, we give the design and implementation of Sunroof.
Using monadic reification, we generate JavaScript from
a deep embedding of the JavaScript monad.
The Sunroof DSL has the feel of native Haskell, with a simple
Haskell-based type schema to guide the Sunroof programmer.
Furthermore, because we are generating code,
we can offer Haskell-style concurrency patterns, such as MVars and Channels.
In combination with a web-services package,
the Sunroof compiler offers a robust platform to build interactive web applications.

\keywords{DSLs, JavaScript, Web Technologies, Cloud Computing}
\end{abstract}
%
%%%%%%%%%%%%%%%%%%%%%%%%%%%%%%%%%%%%%%%%%%%%%%%%%%%%%%%%%%%%%%%%%%%%%%%%%%%%%%%%%%%%%%%%%%%%%%%%
%%%%%%%%%%%%%%%%%%%%%%%%%%%%%%%%%%%%%%%%%%%%%%%%%%%%%%%%%%%%%%%%%%%%%%%%%%%%%%%%%%%%%%%%%%%%%%%%
%%%%%%%%%%%%%%%%%%%%%%%%%%%%%%%%%%%%%%%%%%%%%%%%%%%%%%%%%%%%%%%%%%%%%%%%%%%%%%%%%%%%%%%%%%%%%%%%

\section{Introduction}\label{sec:intro}

% Simon: Describe the problem
There are many reasons to want to program in a functional language:
efficiency of development cost, informally reasoning, high-level 
control- and concurrency-structures, like monads~\cite{...}.
However, mainstream languages often have better environmental support
than what is provided by functional languages,
for example the Objective C and the iOS eco-structure, or JavaScript and HTML5
web browsers.
This paper examines the challenges of providing
an intentionally blurred interface between Haskell
and JavaScript, to support the development of web-based applications.

% Simon: State your contributions
JavaScript is an imperative language with access to a wide range
of established and useful services like graphical canvases and event
handling of browser events. 
We want to express JavaScript in Haskell, adding use
of Haskell's static typing, and gaining access to JavaScript services
in the browser directly in Haskell.

In general, there are a number of ways of providing access to non-native services,
such as the JavaScript canvas.
\begin{enumerate}
\item A first approach is to provide, in Haskell, 
foreign function ``hooks'' to key JavaScript functionality.
Haskell already provides many similar hooks into the RTS,
so why not into the JavaScript engine?
For this to work efficiently, the compiler
needs to target JavaScript. There are already a
number is systems doing this~\cite{...}.
If executed well, this would be ideal,
there are shortcomings: many standard libraries
are not supported directly, the generated
code is not as efficiently executed as native Haskell,
and the compilers are still immature and incomplete.

\item An second approach is to keep the 
foreign function ``hooks'' to key JavaScript functionality,
but instead run Haskell as a server that JavaScript
and the browser interacts with.
Unfortunately, every JavaScript call becomes an expensive proposition: an RPC call
to a browser.
Though some straight-line calls can be batched together --
our own blank-canvas hackage package~\cite{..} was built on this idea --
the granularity of interaction through JavaScript call is just too fine for
this idea to scale well.

\item A first, alternative approach is to use Haskell to generate JavaScript from
an abstract syntax tree of the JavaScript code. There have
been a number of attempt to have DSLs that do the in other spaces,
and functional languages have good support for generating trees.
However, this approach works well for data-flow, for example
as use in Lava. When describing control-flow,
as would be typically in a JavaScript program,  writing such an AST always
feels forced, primarily because there is a mismatch between
bindings in the native language (Haskell) and the non-native
generated code (JavaScript variable names).

\end{enumerate}

This paper investigate the expansion of scope of the second option,
specifically adding binding to the Haskell code
that can be batch together and send to the JavaScript engine,
and then scaling the language to support larger examples.
Binding is done using regular Haskell monadic binding, and fits naturally
into what we expect from a monadic API. Until recently,
it was thought that it was impossible to use a regular
Haskell monad for this purpose. In a previous paper,
we show that such a construction is possible~\cite{..}
In this paper, we expand on this observation,
and show that this form of reification is useful in practice.

Further building on this capability, we also investigate providing
JavaScript control flow and function abstraction mechanisms
to the Haskell programmer interested in using the browser API.
Though for technical reasons we can not compile the transitional
pattern matching and let-binding to JavaScript without committing
to a full compiler Haskell to JavaScript compiler, both
control flow and function abstraction can be provided
with a small syntactical overhead.

With these three capabilities, a programmer can start programming
using the provided JavaScript API directly, and refine
their program to migrate more and more computation
from the server into the browser. In a real sense,
this blurs the distention between an RPC approach (2),
and a full Haskell to JavaScript compiler (1),
built using existing infrastructure, and not needing
the full compiler.




\begin{comment}
\section{Example Sunroof Program}
\label{sec:simple-example}

A simple drawing program, that we build up step by step.
Perhaps bouncing ball.
Perhaps drawing line.
\end{comment}

%%%%%%%%%%%%%%%%%%%%%%%%%%%%%%%%%%%%%%%%%%%%%%%%%%%%%%%%%%%%%%%%%%%%%%%%%%%%%%%%%%%%%%%%%%%%%%%%
%%%%%%%%%%%%%%%%%%%%%%%%%%%%%%%%%%%%%%%%%%%%%%%%%%%%%%%%%%%%%%%%%%%%%%%%%%%%%%%%%%%%%%%%%%%%%%%%
%%%%%%%%%%%%%%%%%%%%%%%%%%%%%%%%%%%%%%%%%%%%%%%%%%%%%%%%%%%%%%%%%%%%%%%%%%%%%%%%%%%%%%%%%%%%%%%%

\section{Calling JavaScript from Haskell}
\label{sec:js-rpc}

From a programmers' point of view, calling JavaScript functions
appears straightforward. We, as a community know how to reflect an
API into Haskell, using the IO monad. Furthermore, objects in
the target API become handles in Haskell. 

As a first example, consider this simplified example of Sunroof code, and corresponding JavaScript.
\noindent
\begin{Code}
-- Haskell                          // JavaScript
ioCode :: IO ()
ioCode = send jsCode

jsCode :: JS ()
jsCode = do                        function jsCode() {
   name <- prompt "Your name?"       var v0 = prompt("Your name?"); 
   alert ("Your name: " <> name)     alert("Your name: " + v0); 
                                   }
\end{Code}%     
%\noindent 

Here, we use a new monad, the \Src{JS} monad, our JavaScript
analog to the \Src{IO} monad,
and an explicit \Src{send} command that sends the JavaScript to the browser.
This reversal of control, where the
server sends the clients commands is called the Ajax Comet~\cite{..},
or simply long polling.
This interface also bundles the \Src{prompt} and \Src{alert} commands
into one interaction transaction.
It is this flavor of interface we want to support in our Sunroof compiler
and web server.  

To make Sunroof a viable interface to JavaScript, we need to
resolve the following issues:
\begin{itemize}
\item JavaScript is an object-based, imperative, dynamically typed language.
Haskell is a pure, function-based, statically typed language.
Specifically how do these two worlds map onto each other.
In section~\ref{sec:object-model}, we discuss the chosen object model
and our JS monad.
%
\item We need to provide an imperative and therefor effectful interface,
including control flow, into our target language of JavaScript.
We do this via the \Src{JS} monad, which we discuss
in section~\ref{sec:js-monad}.
\item We choice to provide a way of defining functions
in Sunroof in a way that they are first-class functions
in JavaScript. This uses our \Src{JS} monad, and
is discussed in section~\ref{sec:functions-continuations}.
%
\item We need to provide a foreign function interface,
to allow us to call specific JavaScript-native functions,
like \Src{prompt} and \Src{alert}.
In section~\ref{sec:ffi} we present this interface.
%
\item Critically, we need to be able to compile our Sunroof DSL
into JavaScript. We look at this in section~\ref{sec:compiler}.
\item To enable the compiled code to dynamically interact with
a web browser, we provide an expansion of the \Src{send} idea above,
which we discuss in section~\ref{sec:server}.
\end{itemize}

%%%%%%%%%%%%%%%%%%%%%%%%%%%%%%%%%%%%%%%%%%%%%%%%%%%%%%%%%%%%%%%%%%%%%%%%%%%%%%%%%%%%%%%%%%%%%%%%
%%%%%%%%%%%%%%%%%%%%%%%%%%%%%%%%%%%%%%%%%%%%%%%%%%%%%%%%%%%%%%%%%%%%%%%%%%%%%%%%%%%%%%%%%%%%%%%%
%%%%%%%%%%%%%%%%%%%%%%%%%%%%%%%%%%%%%%%%%%%%%%%%%%%%%%%%%%%%%%%%%%%%%%%%%%%%%%%%%%%%%%%%%%%%%%%%


\section{JavaScript Object Model}
\label{sec:object-model}

In our JavaScript object model, there are a family of core types, all prefixed with JS.
Each one corresponds to a specific JavaScript type.
There is support for booleans, strings, numbers, functions,
object and arrays, and other common programming structures.
The use of \Src{JSNumber}, rather than (say) \Src{Double},
explicitly reminds us of the enforced diminished capabilities of
being within an embedded language.
All of the \Src{JS}-types are representable in our target language.

Using the \Src{JS}-types, we can give types to \Src{alert} and \Src{prompt}.
\begin{Code}
alert :: JSString -> JS ()
prompt :: JSString -> JS String
\end{Code}
In reality, JavaScript overloads \Src{alert} and \Src{prompt},
so this is providing type-safe access to a sub-set of what
alert can do.

There are two primary Haskell classes represent things that can
be passed to and returned from JavaScript functions.
Instances of the \Src{Sunroof} class 
are values that are returned from JavaScript functions,
and represent all basic JavaScript values.
Instances of the \Src{SunroofArgument} class
are values that be passed to JavaScript functions,
including tuples of values which are used for representing multi-argument 
function calls.
The asymmetry here is a reflection of the JavaScript
asymmetry inherited from C: you can pass multiple arguments
to a function, but only get a single thing back.
Finally, there is a third class, \Src{SunroofKey}, which is
is a JavaScript version of the Haskell \Src{Show} class,
but specifically for generating JavaScript object keys.



\begin{table}[t]
\begin{center}
\begin{tabular}{r@{\quad}l@{\quad}l@{\quad}c}
\hline\rule{0pt}{12pt}%
  Constraint
  & Sunroof Type $\tau$
  & Haskell Analog \HaskellAnalog{$\tau$}
  & \Src{js} \\ \hline\rule{0pt}{12pt}%
  
  & \Src{()}       & \Src{()}     & $\checkmark$ \\
  & \Src{JSBool}   & \Src{Bool}   & $\checkmark$ \\
  & \Src{JSNumber} & \Src{Double} & $\checkmark$ \\
  & \Src{JSString} & \Src{String} & $\checkmark$ \\
  
  \Src{Sunroof $\alpha$}
  & \Src{JSArray $\alpha$} 
  & \Src{[$\HaskellAnalog{\alpha}$]}
  & \\
  
  \Src{SunroofKey $\alpha$}
  & \Src{JSMap $\alpha$ $\beta$}
  & \Src{Map $\HaskellAnalog{\alpha}$ $\HaskellAnalog{\beta}$}
  & \\
  \Src{Sunroof $\beta$} \\
  
  \Src{SunroofArgument $\alpha$}
  & \Src{JSFunction $\alpha$ $\beta$ }
  & \Src{$\HaskellAnalog{\alpha}$ $\rightarrow$ JS$_\Src{A}$ $\HaskellAnalog{\beta}$} 
  & \\
  \Src{Sunroof $\beta$} \\
  
  \Src{SunroofArgument $\alpha$}
  & \Src{JSMVar $\alpha$}
  & \Src{MVar $\HaskellAnalog{\alpha}$}
  & \\
  
  \Src{SunroofArgument $\alpha$}
  & \Src{JSChan $\alpha$}
  & \Src{Chan $\HaskellAnalog{\alpha}$}
  & \\[2pt]
\hline
\end{tabular}
\end{center}
\caption{Sunroof types and their Haskell pendant.}
\label{tab:sunroof-types}
\end{table} 


Table~\ref{tab:sunroof-types} enumerates major \Src{JS} types,
and any restrictions on the type arguments enforced by containers,
like \Src{JSArray}. What can be seen from this is that we
enforce a Hindley–Milner style thinking to our containers,
which is distinct from JavaScript dynamic typing.
Some types involve 
phantom types to give more type safety \cite{Cheney:03:FirstClassPhantomTypes}.
The smooth embedding of booleans and numbers is done through
the Boolean package \cite{project:boolean}.

One design decision in Sunroof is that we enforce a stronger typing
than JavaScript itself would, but also provide an explicit \Src{cast},
for use where the type-systems differ.
\begin{Code}
cast :: (Sunroof a, Sunroof b) => a -> b
\end{Code}
From experience with using Sunroof,
the mis-match in typing between Haskell and Sunroof/JavaScript
is not large problem in practice,
many programs translate from Sunroof to JavaScript
maintaining the stronger typing.
Casts are typically used in the same way \Src{show} is used
to map numbers to strings in the context of building
string values that contain numbers. We also provide
some variants of \Src{cast} with more specific types
to help alleviate some of the dynamic typing.

\TabRef{tab:sunroof-types} shows that most basic 
Haskell types have counterparts in
Sunroof. To convert Haskell values into their 
counterparts, we provide the \Src{SunroofValue} class.
\begin{Code}
class SunroofValue a where
  type ValueOf a :: *
  js :: (Sunroof (ValueOf a)) => a -> ValueOf a
\end{Code}
The type function
\Src{ValueOf} \cite{Chakravarty:05:AssociatedTypeSynonyms} 
provides the corresponding Sunroof type.
\Src{js} converts a value from Haskell to Sunroof. 
By design \Src{SunroofValue} 
only provides instances for values that can be converted in a pure
manner. 




%%%%%%%%%%%%%%%%%%%%%%%%%%%%%%%%%%%%%%%%%%%%%%%%%%%%%%%%%%%%%%%%%%%%%%%%%%%%%%%%%%%%%%%%%%%%%%%%
%%%%%%%%%%%%%%%%%%%%%%%%%%%%%%%%%%%%%%%%%%%%%%%%%%%%%%%%%%%%%%%%%%%%%%%%%%%%%%%%%%%%%%%%%%%%%%%%
%%%%%%%%%%%%%%%%%%%%%%%%%%%%%%%%%%%%%%%%%%%%%%%%%%%%%%%%%%%%%%%%%%%%%%%%%%%%%%%%%%%%%%%%%%%%%%%%

%%%%%%%%%%%%%%%%%%%%%%%%%%%%%%%%%%%%%%%%%%%%%%%%%%%%%%%%%%%%%%%%%%%%%%%%%%%%%%%%%%%%%%%%%%%%%%%%
%%%%%%%%%%%%%%%%%%%%%%%%%%%%%%%%%%%%%%%%%%%%%%%%%%%%%%%%%%%%%%%%%%%%%%%%%%%%%%%%%%%%%%%%%%%%%%%%
%%%%%%%%%%%%%%%%%%%%%%%%%%%%%%%%%%%%%%%%%%%%%%%%%%%%%%%%%%%%%%%%%%%%%%%%%%%%%%%%%%%%%%%%%%%%%%%%

%%%%%%%%%%%%%%%%%%%%%%%%%%%%%%%%%%%%%%%%%%%%%%%%%%%%%%%%%%%%%%%%%%%%%%%%%%%%%%%%%%%%%%%%%%%%%%%%
%%%%%%%%%%%%%%%%%%%%%%%%%%%%%%%%%%%%%%%%%%%%%%%%%%%%%%%%%%%%%%%%%%%%%%%%%%%%%%%%%%%%%%%%%%%%%%%%
%%%%%%%%%%%%%%%%%%%%%%%%%%%%%%%%%%%%%%%%%%%%%%%%%%%%%%%%%%%%%%%%%%%%%%%%%%%%%%%%%%%%%%%%%%%%%%%%

% 
\section{JavaScript Object Model}
\label{sec:object-model}

\begin{comment}
JavaScript is object based. It provides various objects,
including numbers, booleans, maps, and others. We
provide in Sunroof about a dozen common object,
including \Src{JSObject} (the generic object type), \Src{JSNumber}
(floating point numbers), \Src{JSCanvas} (HTML5 canvas type) 
and others. A simple
casting function is provided when the type-system
needs to be overwritten. Along with each of these types,
we provide typed methods.
\begin{verbatim}
 jsDrawBox :: JSObject -> JS t ()
 jsDrawBox document = do
     foo <- document # getElementById("foo")
     cxt <- foo # getContext("2d")     cxt # drawRect (0,0,100,100)
\end{verbatim}
Here, \Src{\#} is a reverse apply, so the types
of the function in the above example are
\begin{verbatim}
(#) :: o -> (o -> JS t a) -> JS t a
getElementById :: JSString -> JSObject -> JS t JSContext
getContext :: JSString -> JSContext -> JS t JSCanvas
drawRect :: (JSNumber,JSNumber,JSNumber,JSNumber) -> JSCanvas -> JS t ()
\end{verbatim}        
From experience, even though we are targeting
an untyped language, the type system gets in the
way less than we expected.
\end{comment}

As JavaScript is not statically typed, one goal
of Sunroof is to use Haskell's type system to
increase the correctness of the written JavaScript.
At the same time we have the problem that we can not characterize 
all different types of objects in JavaScript, since 
users can write their own objects. This means our 
system to type JavaScript needs to be extensible.

The approach from Svenningsson \cite{Svenningsson:12:CombiningEmbedding}
gives us this ability. We provide a basic \Src{Expr}ession 
language that has no associated type information to construct 
JavaScript expressions.
\begin{verbatim}
data E expr 
  = Lit String -- Precompiled (atomic) JavaScript literal
  | Var Id     -- Variable
  | Apply expr [expr]    -- Function application
  ...
data ExprE = ExprE (E ExprE)
type Expr  = E ExprE
\end{verbatim}
This core expression type is then wrapped to represent a more specific 
type. Each of these wrappers implements the \Src{Sunroof} type 
class.
\begin{verbatim}
class SunroofArgument a => Sunroof a where
  box :: Expr -> a
  unbox :: a -> Expr
  ...
\end{verbatim}
It marks that these types represent possible values in JavaScript.
Table \ref{tab:sunroof-types} shows some of the most important 
types in Sunroof. Some types like \Src{JSFunction} also involve 
phantom types to give more type safety (TODO: cite phantom types).
\begin{table}
\begin{center}
\begin{tabular}{r@{\quad}l@{\quad}l@{\quad}c}
\hline\rule{0pt}{12pt}%
  Constraint
  & Sunroof Type $\tau$
  & Haskell Analog \HaskellAnalog{$\tau$}
  & \Src{js} \\ \hline\rule{0pt}{12pt}%
  
  & \Src{()}       & \Src{()}     & $\checkmark$ \\
  & \Src{JSBool}   & \Src{Bool}   & $\checkmark$ \\
  & \Src{JSNumber} & \Src{Double} & $\checkmark$ \\
  & \Src{JSString} & \Src{String} & $\checkmark$ \\
  
  \Src{Sunroof $\alpha$}
  & \Src{JSArray $\alpha$} 
  & \Src{[$\HaskellAnalog{\alpha}$]}
  & \\
  
  \Src{SunroofKey $\alpha$}
  & \Src{JSMap $\alpha$ $\beta$}
  & \Src{Map $\HaskellAnalog{\alpha}$ $\HaskellAnalog{\beta}$}
  & \\
  \Src{Sunroof $\beta$} \\
  
  \Src{SunroofArgument $\alpha$}
  & \Src{JSFunction $\alpha$ $\beta$ }
  & \Src{$\HaskellAnalog{\alpha}$ $\rightarrow$ JS$_\Src{A}$ $\HaskellAnalog{\beta}$} 
  & $\checkmark$ \\
  \Src{Sunroof $\beta$} \\
  
  \Src{SunroofArgument $\alpha$}
  & \Src{JSMVar $\alpha$}
  & \Src{MVar $\HaskellAnalog{\alpha}$}
  & \\
  
  \Src{SunroofArgument $\alpha$}
  & \Src{JSChan $\alpha$}
  & \Src{Chan $\HaskellAnalog{\alpha}$}
  & \\[2pt]
\hline
\end{tabular}
\end{center}
\caption{Sunroof types and their Haskell pendant.}
\label{tab:sunroof-types}
\end{table} 
The table also shows that most of basic Haskell types have counterparts in
Sunroof. To convert these core Haskell values into their Sunroof 
counterparts we provide the \Src{SunroofValue} type class.
\begin{verbatim}
class SunroofValue a where
  type ValueOf a :: *
  js :: (Sunroof (ValueOf a)) => a -> ValueOf a
\end{verbatim}
It provides the corresponding Sunroof value through the type function
\Src{ValueOf} (TODO: Cite type functions) and offers the function 
\Src{js} to convert to that type. By design \Src{SunroofValue} does
only provide instances for values that can be converted in a pure
manner. Certain primitive values in JavaScript are referential 
transparent according to \Src{==} while others, like general objects,
are not. As an example, if you call \Src{new Object()} twice you get the 
same empty object, but when compared by \Src{==} they are different. They
are not identical, because in this case reference equality is checked
instead of value equality. We call this observable allocation and handle 
it as a side-effect which should only occur in the \JS-monad.

Sunroof also offers the ability to work with record like data structures to 
keep values that belong together in one place.
For this purpose Sunroof offers the \Src{JSTuple} type class.
\begin{verbatim}
class Sunroof o => JSTuple o where
  type Internals o
  match :: (Sunroof o) => o -> Internals o
  tuple :: Internals o -> JS t o
\end{verbatim}
If you have a record of Sunroof data that you want to
encode as a JavaScript object you can provide this ability 
by implementing \Src{JSTuple}. The \Src{Internals} type function
delivers your record type. Encoding that record as a \Src{Sunroof}
value is done through \Src{tuple}. As objects may be used to 
encode records we have to be inside the \JS-monad to do this,
because of the observable allocation discussed earlier.
Decomposing the encoding is done with \Src{match}. Because the
\Src{JSTuple} idiom is meant to represent immutable data structures, this 
can be done in a pure manner although there are ways to mutate 
values referenced by the decomposed record, since they are only 
references to the actual data in JavaScript.

\subsubsection{RESOURCES - REMOVE WHEN FINISHED}

General Objects Types and Expressions
\begin{itemize}
\item We want to use Haskell's type system to give use type safety in JavaScript/Sunroof.
\item JavaScript is untyped / dynamically typed (TODO: what is it actually? cite/reference?)
\item We can not represent all types in JavaScript
\item We have to give a way to add types later on
\item Use approach from \cite{Svenningsson:12:CombiningEmbedding}
\item There is a core expression language that is used to describe how and expression is built
\item We wrap that core type into wrappers
\item Introduce \Src{Sunroof} to mark wrappers
\item Wrappers have specific functionality for certain type
\item Wrappers can have phantom type to give more type safety
\item Example \Src{JSString} and \Src{JSArray a}
\item Maybe go through example type like \Src{JSString} or \Src{()}.
\item Unavoidable need to cast types some times: \Src{cast}
\item Adding a new type can be done mechanically: Template Haskell
\end{itemize}
Haskell to Sunroof conversion
\begin{itemize}
\item Many haskell types have equivalent or similar types in JavaScript
\item So we offer \Src{SunroofValue}; It connects a Haskell value with 
its JavaScript companion through a type function (cite type functions)
and gives a function to convert: \Src{js}
\item Introduce \Src{SunroofValue}
\item Conversion is pure for most types since atomic values 
are created when converting.
\item Exception \Src{JSFunction}, same reason as for \Src{JSTuple} and \Src{tuple}
\item Show table with types as an overview. Small comment paragraph about table.
\item \Src{JSArray} not possible because of conflicts with \Src{String}
instance (minor point); also issue of mutability of arrays; observable 
allocation issue like with \Src{JSTuple}.
\item Forward reference to section \ref{sec:threading-models} for \Src{JSMVar}
and \Src{JSChan}
\end{itemize}
JSTuple and records in JS
\begin{itemize}
\item Design decision: We do not want to introduce internal
structure for types. (Tuples aren't JS types)
\item \Src{JSTuple} exists to introduce types with custom structure
\item Introduce \Src{JSTuple}
\item manages composition and decomposition of custom types.
\item Meant to codify immutable records in JavaScript
\item They can only be changed
by decomposing them into Haskell and recreating a new structure
with other values.
\item Useful to manage more complex data structures in JavaScript.
\item Decomposition is pure operation: \Src{match}
\item Justification lies in the fact that they are meant to be 
immutable. Although there are possibilities to break this.
\item Composition is monadic effect: \Src{tuple}
\item It captures observable allocation and ensures that a reference to the
allocated value is created and used afterwards.
\item If it was not monadic the let binding would be like a macro
that reallocated the same object at each point. This 
is not desired behavior in most use-cases
\end{itemize}









% 
\section{The JavaScript Monad}
\label{sec:js-monad}

JavaScript is an imperative language with access to a wide range
of established and useful services, like graphical canvases and event
handling. JavaScript as a language also provides features that are
traditionally associated with functional languages, like first-class 
functions. We want to express JavaScript in Haskell, adding use
of Haskell's static typing, and gaining access to JavaScript services.
And we do so using the transitional functional programming 
mechanism for being imperative, a monad~\cite{Moggi:91:ComputationMonads}.

The \JS-monad is the monad of JavaScript effects, as is an almost
exact analog for the Haskell \IO-monad, except there
is an extra phantom argument~\cite{Leijen:99:Phantom} that we will return
to shortly.  Figure~\ref{fig:code-example} gives a first example
of the \JS-monad in use with the generated JavaScript.

Inside this simple example is a challenging problem -- where does
\Src{v0} come from? The bind inside the monadic \Src{do} is
unconstrained:
\begin{verbatim}
(>>=) :: JS t a -> (a -> JS t b) -> JS t b
\end{verbatim}
What we want is:
\begin{verbatim}
(>>=) :: (Sunroof a) => JS t a -> (a -> JS t b) -> JS t b
\end{verbatim}
Where \Src{Sunroof} constrains the bind to
arguments for which we can generate a JavaScript variable.
Counterintuitively, 
it turns out that a specific form of normalization allows 
the ``\Src{a}'' type to be constrained and the bind to 
be an instance of the standard monad class~\cite{Sculthorpe:13:ConstrainedMonads}.
Through this keyhole of {\em monadic reification\/},
the entire Sunroof language is realized.

\subsubsection{RESOURCES - REMOVE WHEN FINISHED}

We use monad reification.
\begin{itemize}
\item binding in Haskell becomes binding in reified language.
\item Feels like a native monad, cf STM.
\item Can build abstractions on top of this.
\end{itemize}






% 
\section{JavaScript Functions and Continuations}
\label{sec:functions-continuations}

Given these threading models, we can realize both as 
JavaScript objects. 
\begin{verbatim}
function     :: (...) => (a -> JS A b)  -> JS t (JSFunction a b)
continuation :: (...) => (a -> JS B ()) -> JS t (JSContinuation a)
\end{verbatim}
\Src{JSFunction a b} and \Src{JSContinuation a},
like \Src{JSObject} and others, are realized as objects
in JavaScript. Thus they can be passed as arguments, returned
from functions and stored in mutable structures.

\Figure%
{fig:func-cont}%
{figures/sunroof-func-cont.pdf}%
{How functions and continuations relate between the Haskell- and JS-domain.}%

Functions and continuations can be called using \Src{apply}
and \Src{goto} respectively. \Src{apply} calls the function
and returns, \Src{goto} calls the continuation, but never
returns.
\begin{verbatim}
apply :: (...) => JSFunction args ret -> args -> JS t ret
goto  :: (...) => JSContinuation args -> args -> JS t a
\end{verbatim}

\begin{table}
\caption{Reifying and calling JavaScript functions}
\begin{center}
\begin{tabular}{r@{\quad}c@{\quad}c@{\quad}c@{\quad}c}
\hline\rule{0pt}{12pt}%

                & Monadic Function      & Reification   & Object in     & Invocation\\
                & in Haskell            & Function      & Javascript    & Function\\
\hline\rule{0pt}{12pt}%
  Functions
  & $\alpha\rightarrow\ $\Src{JS}$_\Src{A}~\beta$
  & \Src{function}
  & \Src{JSFunction}~$\alpha~\beta$
  & \Src{apply} \\
  Continuations
  & $\alpha\rightarrow\ $\Src{JS}$_\Src{B}~\Src{()}$
  & \Src{continuation}
  & \Src{JSContinuation}~$\alpha$
  & \Src{goto}\\
\hline
\end{tabular}
\end{center}
\end{table} 

\subsubsection{RESOURCES - REMOVE WHEN FINISHED}

\begin{itemize}
\item Functions are first class members in Haskell and JavaScript
\item But in JavaScript they are limited (no partial application)
\item That leads to type for functions \Src{JSFunction a b}
\item Functions can take more then one argument; \Src{Sunroof} insufficient: \Src{SunroofArgument}
\item \Src{SunroofArgument} is prerequisite for \Src{Sunroof},
because all JS objects are parameters but pairs of them are not 
JS objects.
\item Introduce \Src{SunroofArgument}.
\item So function in JS are not curried (partial application is questionable in JS)
\item Create a function with \Src{function} combinator.
\item Function creation is a monadic effect for the same reason 
a record creation (\Src{JSTuple}). It represents observable
allocation and we usually want one reference to a function 
instead of coping its definition everywhere.
\item Calling a function or method can have a side-effect
\item That is why the operator \Src{\$\$} for function application is monadic to.
\item All these operations do is to produce \Src{JS\_Function} and \Src{JS\_Invoke}
instructions.
\item Continuations are basically the same as functions at this level.
\item Created with \Src{continuation} and called with \Src{goto}.
\item Implementation of \Src{goto} is interesting. Ignores 
the current continuation in the \JS-monad and continues with the given one.
\item TODO: Further detail on continuations; Here or somewhere else?
\end{itemize}







% 
\section{JavaScript Threading Models}
\label{sec:threading-models}

Sunroof was first documented in our previous 
workshop paper~\cite{Farmer:12:WebDSLs},
where the possibility of monadic reification was observed.
In this paper, we raised an unresolved issue: do you
generate atomic JavaScript code, and keep the callback
centric model of computation, or generate JavaScript
using CPS, and allow for blocking primitives,
like Haskell \Src{MVar}s. The latter, though more powerful, 
precluded using the compiler to generate
code that can be cleanly called from native JavaScript.
Both choices had poor consequences.

So, rather than pick one, we decided to explicitly support both,
and make both first class threading strategies in our compiler.
In terms of user-interface, we parameter the \JS-monad
with a phantom type that represents the threading model
to compile with, with \Src{A} for \Src{A}tomic threads,
and \Src{B} for \Src{B}locking (cooperative concurrency) threads. 
Atomic threads are classical JavaScript threads, and
are never interrupted; while blocking threads can
support suspending operations. By using phantom
types, we can express the necessary
restrictions on specific combinators, as well
as provide combinators to allow both types of
threads to cooperate successfully.

\subsubsection{RESOURCES - REMOVE WHEN FINISHED}




% 
\section{Foreign Function Interface}
\label{sec:ffi}

Sunroof also offers a 
foreign function interface, which enables us to easily 
access predefined JavaScript. There are three core functions.
\begin{Code}
fun    :: (SunroofArgument a, Sunroof r) 
       => String -> JSFunction a r
object :: String -> JSObject
new    :: (SunroofArgument a) 
       => String a -> JS t JSObject
invoke :: (SunroofArgument a, Sunroof o, Sunroof r) 
       => String -> a -> o -> JS t r
\end{Code}
\Src{fun} is used to create Sunroof functions from their names in JavaScript.
This can happen in two ways. Either to call a function in line or to 
create a real binding for that function. As an example 
the \Src{alert} function can be called in line through \Src{fun "alert" \$\$ "text"}
or you can provide a binding in form of a Haskell function for it.
\begin{Code}
alert :: JSString -> JS t ()
alert s = fun "alert" $$ s
\end{Code}

Objects can be bound through the \Src{object} function, e.g.
the \Src{document} object is bound through \Src{object "document"}.
Constructors can be called using \Src{new}. To create a new
object you would call \Src{new "Object" ()}.

We can call methods of objects through \Src{invoke}. Again, this 
can be used in line and to create a real binding. A inline 
use of this to produce \Src{document{\linebreak}.getElementById("id")} would look like this: 
\begin{Code}
object "document" # invoke "getElementById" "id"
\end{Code}
Where \Src{\#} is just a flipped function application. To provide a binding 
to the \Src{getElementById} method one can write:
\begin{Code}
getElementById :: JSString -> JSObject -> JS t JSObject
getElementById s = invoke "getElementById" s
\end{Code}

Providing actual bindings ensures that
everything is typed correctly and prevents resolving ambiguities 
through large type annotations inside of code.

The current release of Sunroof already provides bindings for most of the 
core browser API, the HTML5 canvas element and some of the JQuery API.

\begin{comment}
Table \ref{tab:ffi} gives an overview of
how Sunroof's FFI can be used. 
\begin{table}
\begin{center}
\begin{tabular}{l@{\quad}p{7cm}}
  \hline\rule{0pt}{12pt}%
  JavaScript & 
  Sunroof \\ \hline\rule{0pt}{12pt}%
%
  \Src{alert("Test");} & 
  \Src{fun "alert" \$\$ "Test"} \\[2pt]
%
  \Src{alert} as a Sunroof function & 
  \Src{alert :: JSFunction JSString ()\newline alert = fun "alert"} \\[2pt]
%
  \Src{alert} as a Haskell function & 
  \Src{alert :: JSString -> JS t ()\newline alert s = fun "alert" \$\$ s} \\[2pt]
%
  \Src{document.getElementById("id");} & 
  \Src{object "document" \# invoke "getElementById" "id"} \\[2pt]
%
  \Src{getElementById} as method & 
  \Src{getElementById :: JSString -> JSObject -> JS t JSObject\newline
       getElementById s = invoke "getElementById" s} \\[2pt]
\hline
\end{tabular}
\end{center}
\caption{JavaScript expressed through the Sunroof FFI.}
\label{tab:ffi}
\end{table} 
\end{comment}
%

% 
\section{The Sunroof Compiler}
\label{sec:compiler}

Given the language, and monadic-reification, how do we compile this language?
Figure \ref{fig:structure} shows how Sunroof is structured.

\Figure%
{fig:structure}%
{figures/sunroof-structure.pdf}%
{The structure of Sunroof.}

Through the \JS-monad we produce a \Src{Program (JSI t) ()} instance. We 
translate such a program into a list of statements (\Src{Stmt}) by matching over 
the \JSI constructors.
\begin{verbatim}
data Stmt 
  = AssignStmt Rhs Expr       -- Assignment
  | DeleteStmt Expr           -- Delete reference
  | ExprStmt Expr             -- Expression as statement
  | ReturnStmt Expr           -- Return statement
  | IfStmt Expr [Stmt] [Stmt] -- If-Then-Else statement
  | WhileStmt Expr [Stmt]     -- While loop
  | CommentStmt String        -- Comment
\end{verbatim}
The constructors of \Src{Stmt} are straight forward and
directly represent the different statements one can write
in JavaScript.

To get a feeling of what is happening here, we will have a closer
look to how two of the \JSI instructions are compiled. First
lets have a closer look how branches are compiled.
\begin{verbatim}
compile :: Program (JSI t) () -> CompM [Stmt]
compile = eval . view
  where
    eval :: ProgramView (JSI t) () -> CompM [Stmt]
    eval (JS_Branch b c1 c2 :>>= k) = 
      case evalStyle (ThreadProxy :: ThreadProxy t) of
        A -> do
          (src0, res0) <- compileExpr (unbox b)
          res :: a <- jsValue
          let bindResults :: a -> JS t ()
              bindResults res' =
                sequence_ [ single $ JS_Assign_ v (box $ e :: JSObject)
                          | (Var v, e) <- jsArgs res `zip` jsArgs res'
                          ]
          src1 <- compile $ extractProgramJS bindResults c1
          src2 <- compile $ extractProgramJS bindResults c2
          rest <- compile (k res)
          return (src0 ++ [ IfStmt res0 src1 src2 ] ++ rest)
        B -> do
          fn_e <- compileContinuation 
            (\ a -> blockableJS $ JS $ \ k2 -> k a >>= k2)
          fn           <- newVar
          (src0, res0) <- compileExpr (unbox b)
          src1 <- compile $ extractProgramJS (apply (var fn)) c1
          src2 <- compile $ extractProgramJS (apply (var fn)) c2
          return ( [mkVarStmt fn fn_e] ++ src0 ++ [ IfStmt res0 src1 src2 ])
\end{verbatim}
\TODO This brings up more questions then it answers.


\begin{comment}
Given the language, and monadic-reification, how do we compile this language?
Figure \ref{fig:structure} shows how Sunroof is structured.
On the lowest level we provide an untyped expression language \Src{Expr}
that describes JavaScript expressions. 
To provide type safety when using Sunroof we create
wrappers for each type we want to represent, e.g. \Src{JSNumber} or \Src{JSString}.
The \Src{Sunroof} type class provides an 
interface to create wrapped and unwrapped
instances of our expressions. Based on the wrappers we can provide 
operations specific to a certain type, e.g. a \Src{Num} instance
for \Src{JSNumber} or a \Src{Monoid} instance for \Src{JSString}.



This technique enables us to utilize  Haskells type system when writing JavaScript
and offers an easy way to add new types when needed~\cite{Svenningsson:12:CombiningEmbedding}.
By using phantom types we can also provide more advanced types,
like \Src{JSArray a}.

The next layer provides JavaScript instructions through the type \JSI.
They represents abstract statements. While expressions and values
represented with type wrappers are assumed to be free of side-effects,
the instructions model side-effects in JavaScript. Examples for Instructions
are assignment of an attribute or the application of a function.
\begin{verbatim}
data JSI :: T -> * -> * where
  JS_Assign :: (...) => JSSelector a -> a -> JSObject -> JSI t ()
  JS_Invoke :: (...) => a -> JSFunction a r -> JSI t r
  JS_Branch :: (...) => bool -> JS t a -> JS t a  -> JSI t a
  ...
\end{verbatim}
The \JS-monad with its combinators builds a sequence of 
\JSI{}nstructions through the operational
package~\cite{Hackage:10:Operational,Apfelmus:10:Operational}.
All constraints required on instructions are introduced by their 
constructors.
As mentioned earlier the \JS-monad comes in two threading flavors, 
parameterized using a phantom type.
Internally the \JS-monad is implemented using CPS on the 
underlying \Src{Program} type from Operational. 
\begin{verbatim}
data JS :: T -> * -> * where
  JS   :: ((a -> Program (JSI t) ()) -> Program (JSI t) ()) 
       -> JS t a
  ...
\end{verbatim}
For atomic
computations we just produce a list of instructions from the continuation. 
When translating possibly blocking code we directly translate that continuation
into JavaScript functions. This gives us the ability to handle 
computations as values in JavaScript and store them if needed.

Blocking operations just store the rest of their computation in a queue.
When the event to unblock occurs the pending computation is registered 
as a callback that will be executed as soon as the current computation
is done.

On top of our \JS-monad we provide ways of specifying (typed)
interfaces to JavaScript capabilities, a Foreign Function Interface.
\begin{verbatim}
alert :: JSString -> JS t ()
alert = fun "alert"

getElementById :: JSString -> JSObject -> JS t JSCanvas
getElementById = invoke "getElementById"
\end{verbatim}
Notice, that calling a JavaScript function or method is done by giving its
name to one of the provided combinators. Types can be specialized using 
a Haskell type annotation. A flexible and easy to use approach.
\end{comment}

\begin{comment}
\subsubsection{RESOURCES - REMOVE WHEN FINISHED}

Compiling
\begin{itemize}
\item Short introduction to the compiler interface (signature).
\item Core work done by translating \Src{Program (JSI t) ()}
into a list of \Src{Stmt}s.
\item Introduce statement type, give a short description of 
each constructor (just in the comments)
\item Basic idea: Each \JSI~nstruction is translated 
into a sequence of statements and these are then
concatenated together.
\item Look at interesting parts
\item TODO: Which parts are interesting? Most parts are too technical
\item Show how a branch is compiled (TODO: \Src{extractProgramJS} lets things look messy)
\item Talk about how \Src{JS\_Fix} is compiled.
\item TODO: Understand why fix works.
\item Difference between compilation of a function and a continuation
\item Function \Src{\textbackslash a -> JS \$ \textbackslash k -> singleton (JS\_Return a) >>= k}
\item TODO: Why \Src{>>= k}?
\item Return the result of the current continuation
\item Continuation \Src{\textbackslash \_ -> JS \$ \textbackslash k -> k ()}
\item Just execute it instead of passing it on further. No return value!
\item We use expression sharing through observable sharing (reference Andys paper).
\end{itemize}

\begin{itemize}
\item If we transliterate, we have straight line code, can not pause.
    (Wait for Mvar, for example)
\item If we CPS translate, we can use continuations to capture the
   notion of a paused thread. Works well.
   Problems:
  \begin{itemize}
   \item Can not translate functions, how do they get there return value
   \item (Assumes straight line code.)
   \item Also, the code becomes unreadable to anyone except a die-hard 
       compiler freak.
  \end{itemize}
\end{itemize}
 
Choice:
\begin{itemize}
\item We support both!
\item Phantom argument to JS
\item A = Atomic, B = Blockable.
\end{itemize}
\end{comment}






% 
\section{The Sunroof Server}
\label{sec:server}

The Sunroof compiler can compile JavaScript that can be used
stand-alone. But Sunroof really comes
into its own when used with the Sunroof server.

It provides infrastructure to send arbitrary pieces 
of JavaScript to a calling website for execution. 
Like this it is possible to interleave Haskell and JavaScript 
computations with each other as needed. The three major functions
provided are \Src{sunroofServer}, \Src{syncJS} and \Src{asyncJS}.
\begin{verbatim}
type SunroofApp = SunroofEngine -> IO ()
sunroofServer :: SunroofServerOptions -> SunroofApp -> IO ()

syncJS  :: SunroofResult a 
        => SunroofEngine -> JS t a -> IO (ResultOf a)
asyncJS :: SunroofEngine -> JS t () -> IO ()
\end{verbatim}
\Src{sunroofServer} starts the server.
It will the given callback function (\Src{SunroofApp}) 
for each request to the server.
\Src{syncJS} and \Src{asyncJS} allow the server
to remotely execute monadic JavaScript from inside the 
requesting website.
\Src{asyncJS} executes JavaScript asynchronously without 
waiting for a return value. In contract to that 
\Src{syncJS} waits until the execution is complete and
then sends the result back to the server and converts if 
into a Haskell value that can be processed further. 
Values that can be returned from a synchronous execution 
have to implement the \Src{SunroofResult} class. It contains a type-function
\Src{ResultOf} that maps the 
Sunroof type to a corresponding Haskell type and also 
implements a function to convert incoming data to that 
type.

Besides these basic functions it also provides the abstract 
concepts of \Src{Downlink}s and \Src{Uplink}s. They are 
channels for sending data either from the server or to
the server direction.

\begin{comment}
\subsubsection{RESOURCES - REMOVE WHEN FINISHED}

\begin{itemize}
\item Introduce major idea of a server that can communicate
with the calling website and send arbitrary pieces
of JavaScript to execute on demand.
\item TODO: Go into detail about the kansas-comet stuff 
or just give a reference to the original paper?
\item Introduce the major function for running and communicating:
\Src{sunroofServer}, \Src{syncJS} and \Src{asyncJS}
\item Describe what they do.
\item Specifically look at \Src{syncJS} and \Src{SunroofResult}
used to return actual Haskell values for the sent JavaScript return value.
\item Introduce abstractions \Src{Downlink} and \Src{Uplink} that
can be used to communicate in either direction.
\item TODO: This is only a brief introduction?
\end{itemize}
\end{comment}







% 
\section{Case Study - A Small Calculator}
\label{sec:extended-example}

To see how Sunroof works in practice, we will look into the 
experience we gathered when writing a small calculator
for arithmetic expressions (\FigRef{fig:example-application}). 
We use Sunroof to display our interface
and the results of our computation. Haskell will be used to parse the 
arithmetic expressions and calculate the result. The Sunroof server 
will be used to implement this JavaScript/Haskell hybrid.

\FigureS%
{fig:example-application}%
{figures/example-application.png}%
{The example application running on the Sunroof server.}%
{scale=0.6}

The classical approach to develop an application like this would have 
been to write a server that provides a RESTful interface and replies 
through a JSON data structure. 
The client side of that application would have been written in JavaScript
directly.
This can be seen in \FigRef{fig:example-structure}.

\Figure%
{fig:example-structure}%
{figures/example-structure.pdf}%
{Classical structure and Sunroof structure of a web application.}

How does Sunroof improve or change this classical structure?
First of all, in Sunroof you write the client-side code together with
your server application within Haskell. In our example, all code 
for the server and client is in Haskell. The control logic 
for the client side is provided through the server.
This leads to a tight coupling between both sides. 
This also shows how Sunroof blurs the border between the server 
and client side. You are not restricted by an interface or language 
barrier. If you need the client to do something, you can just 
send arbitrary Sunroof code to execute in the client.

\TabRef{tab:example-statistics} contains a few statistics 
about the size of the code in each part of the client.

The client-server response loop shuffles new input to the server 
and executes the response in the client.
%
Data conversion is needed, because pure Haskell data types
cannot be handled in Sunroof and vice versa. There still
exists a language barrier between JavaScript and Haskell. 
Code to convert between two essentially equal data structures on 
each side must be written, as well as representations of Haskell 
structures in Sunroof. However, there is great potential in automatically 
generating this code using techniques such as template Haskell
\cite{Sheard:02:TemplateMetaProgrammingHaskell}.

The code for displaying the results is basically a 
transliteration of the JavaScript that you would write for this 
purpose.
The transliteration used here is not very appealing. 
In the future, this code can be generated through higher-level 
libraries. Sunroof is intended to deliver a foundation for
this purpose.

The rest of our code to parse the arithmetic expression and calculate 
results is classical Haskell code. 

\pagebreak

\begin{table}[t]
\begin{center}
\begin{tabular}{l@{\quad}r@{\quad}r}
\hline\rule{0pt}{12pt}%
Part of Application & Lines of Code & Percentage \\[2pt]
\hline\rule{0pt}{12pt}%
Response loop & 25 & 6.5\% \\[2pt]
Data conversion & 85 & 22.0\% \\[2pt]
Rendering & 190 & 49.5\% \\[2pt]
Parsing and interpretation & 85 & 22.0\% \\[2pt]
\hline
\end{tabular}
\end{center}
\caption{Lines of code needed for the example.}
\label{tab:example-statistics}
\vspace{-0.5cm}
\end{table} 


%\TODO{Should we add the discussion why we choose a deep embedding somewhere?}

% 
\section{Related Work}

There have been several attempts to translate Haskell to JavaScript.
Prominent ones are the compiler backends for 
UHC \cite{Stutterheim:12:ImprovingUHCJavaScriptBackend} and 
GHCJS \cite{project:ghcjs}. There are also projects like Fay \cite{project:fay} 
that compile subsets of Haskell to JavaScript or JMacro \cite{project:jmacro}
which use quasiquotation \cite{Mainland:07:QuasiquotingHaskell} to embed 
a custom-tailored language into Haskell code.

At the same time there are also projects like 
CoffeeScript \cite{project:coffeescript} or LiveScript \cite{project:livescript}
to build custom languages 
that are very similar to JavaScript but add convenient syntax and
support for missing features.

Our approach to cooperative concurrency through continuations in JavaScript has
has been used before 
\cite{Cooper:07:LinksWebProgrammingTiers,Predescu:02:CocoonContinuationBasedControlFlow}.
To our knowledge, creating a direct connection
between Haskell and JavaScript continuations has not been 
attempted before.

Deep embeddings of monads based on data structures have been used before
in Unimo \cite{Lin:06:Unimo} and Operational \cite{Apfelmus:10:Operational,Hackage:10:Operational}. 
The specific approach Sunroof takes 
by using GADTs has been discussed by 
Sculthorpe et al. \cite{Sculthorpe:13:ConstrainedMonads} 
in detail.

The Sunroof server does not have the aim to provide a full-featured 
web framework, as HAppS, Snap or Yesod do. It only provides 
the infrastructure to communicate with the currently calling website
through the Kansas comet \cite{project:kansas-comet} 
push mechanism \cite{pattern:push}. Although all of the
frameworks mentioned above would be able to implement this technique,
to our knowledge, none of them has yet.

To our knowledge, Sunroof is the only library that supports 
generation of JavaScript inside of Haskell using cheap and cheerful Haskell
in a type-safe manner. All other approaches discussed above
either require a separate compilation step or introduce new 
syntax inside of Haskell.

There is an effort to generalize Active \cite{project:active}, a library for animations, and
implement a backend based on Sunroof \cite{project:sunroof-active}.





% 
\section{Conclusion}

Sunroof took the key idea of monad reification and
successfully created the \JS-monad to describe computations
in JavaScript. This work was started by Farmer and
Gill \cite{Farmer:12:WebDSLs}, with the observation
of the possibility to reify a JS monad.
This paper documents the work since this initial implementation.
By adding the concept 
of \Src{JSFunction} and \Src{JSContinuation}, there now is a 
stronger connection between
functions in the JavaScript and the Sunroof language space 
(\FigRef{fig:func-cont}). It is possible to go back and forth between 
both worlds. Combining both concepts, functions and the \JS-monad,
we were able to create a second implementation of the monad, this
time based on the direct translation of continuations from Haskell
to JavaScript. It enabled us to build a blocking threading model
on top of JavaScript that resembles the model already known from Haskell.
Based on this model and the provided abstraction over continuations,
we can use primitives like \Src{forkJS} or \Src{yield}.
Higher-level abstractions like \Src{JSMVar} and \Src{JSChan} are also
available. 


\section{Acknowledgment}

We want to thank Conal Elliott for his support in adapting 
the Boolean package \cite{project:boolean} and helping us to
extend it with support for deeply embedded numbers.

%
% ---- Bibliography ----
%
\bibliographystyle{splncs03}
\bibliography{sunroof}
\vspace{-0.5cm} % THIS SAVED MY LIFE

\end{document}
