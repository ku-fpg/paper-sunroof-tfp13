% This is LLNCS.DEM the demonstration file of
% the LaTeX macro package from Springer-Verlag
% for Lecture Notes in Computer Science,
% version 2.4 for LaTeX2e as of 16. April 2010
%
\documentclass{llncs}
%
\usepackage{amsfonts}
\usepackage{comment}
\usepackage{graphicx}
\usepackage{caption}
\usepackage{subcaption}

\newcommand{\SunroofAnalog}[1]{#1\ensuremath{_\downarrow}}
\newcommand{\HaskellAnalog}[1]{#1\ensuremath{_\uparrow}}

%\newcommand{\NOTE}[1]{{\Large\textbf{NOTE:}\ #1}}
\newcommand{\TODO}[1]{{(\textbf{TODO:}\ #1)}}
\newcommand{\Src}[1]{{\tt{#1}}}

\newcommand{\IO}{\Src{IO}}
\newcommand{\JS}{\Src{JS}}
\newcommand{\JSI}{\Src{JSI}}
\newcommand{\JSA}{\ensuremath{\Src{JS}_\Src{A}}}
\newcommand{\JSB}{\ensuremath{\Src{JS}_\Src{B}}}

\newcommand{\Figure}[3]{%
\FigureS{#1}{#2}{#3}{scale=0.55,clip=true,trim=0.45cm 0.45cm 0.45cm 0.45cm}
}

\newcommand{\FigureS}[4]{%
\begin{figure}[h]%
\vspace{-0.5cm}%
\begin{center}%
\includegraphics[#4]{#2}%
\vspace{-0.5cm}%
\end{center}%
\caption{#3}%
\label{#1}%
\vspace{-0.5cm}%
\end{figure}%
}

\begin{document}
%
\title{Sunroof: A Monadic DSL to Generate JavaScript}
%\subtitle{}
%
\titlerunning{Sunroof}  % abbreviated title (for running head)
%                                     also used for the TOC unless
%                                     \toctitle is used
%
\author{Jan Bracker\inst{1,2}  \and Andy Gill\inst{1}}
%
\authorrunning{Jan Bracker \and Andy Gill} % abbreviated author list (for running head)
%
%%%% list of authors for the TOC (use if author list has to be modified)
\tocauthor{Jan Bracker, Andy Gill}
%
\institute{%
ITTC / EECS \\
The University of Kansas\\
Lawrence, KS 66045\\
~\\
\and
Institut f{\"u}r Informatik\\
Christian-Albrechts-Universit{\"a}t\\
Kiel, Germany}

\maketitle

\begin{abstract}        
 
Sunroof is a Haskell-hosted Domain Specific Language (DSL) for generating JavaScript.
Sunroof is built on top of the JavaScript monad, which, like the Haskell \IO-monad, allows 
access to external resources, but specifically JavaScript
resources. As such, Sunroof is primarily a feature-rich 
foreign-function API to the browser's JavaScript engine, and all the browser-specific
functionality, including HTML-based rendering, event handling, and 
drawing to the HTML5 canvas. 

In this paper, we give the design and implementation of Sunroof.
Using monadic reification, we generate JavaScript from
a deep embedding of the JavaScript monad.
The Sunroof DSL has the feel of native Haskell, with a simple
Haskell-based type schema to guide the Sunroof programmer.
Furthermore, because we are generating code,
we can offer Haskell-style concurrency patterns, such as MVars and Channels.
In combination with a web-services package,
the Sunroof compiler offers a robust platform to build interactive web applications.
%giving the ability to interleave Haskell and JavaScript computations
%with each other as needed.
\keywords{DSLs, JavaScript, Web Technologies, Cloud Computing}
\end{abstract}
%

\TODO{I ordered the sections such that forward references in 
them only point to the threading model section. These references occur,
because the threading model is a major motivation for many of
the implemented features in Sunroof. Maybe we should explain that in
the introduction?}

 
\section{Introduction}

Sunroofs was developed to support Haskell programmers 
when generating JavaScript. Lets look at the small example
in figure \ref{fig:code-example} to see how it achieves this goal.

\begin{figure}[h]
\vspace{-0.5cm}
\centering
\begin{subfigure}{0.45\textwidth}%
\begin{verbatim}
 jsCode :: JS t ()
 jsCode = do
   name <- prompt "Your name?"
   alert ("Your name: " <> name)
\end{verbatim}%
\end{subfigure}%
\hfill%
\begin{subfigure}{0.45\textwidth}
\vspace{0.25cm}%
\begin{verbatim}
  
  
var v0 = prompt("Your name?"); 
alert("Your name: " + v0);
\end{verbatim}%
\end{subfigure}% 
%\vspace{-0.2cm}%
\caption{Sunroof program and the expected JavaScript on the left.}%
\label{fig:code-example}%
\vspace{-0.5cm}
\end{figure}

We can see what code we expect the program to produce when 
translated to JavaScript on the right. But what makes 
this useful? The JavaScript we want to produce with the given
Haskell is shorter and can easily be written by hand.
So lets look at the advantages of this approach:
\begin{itemize}

\item
Sunroof introduces a threading model (Section \ref{sec:threading-models}) similar
to Haskell's. Like this programmers can reuse known abstractions
and have a known set of powerful tools.

\item 
It also introduces types that support us when 
writing code and reveal common mistakes. The functions
in figure \ref{fig:code-example} have toe following signature:
\begin{verbatim}
prompt :: JSString -> JSString -> JS t JSObject
alert  :: JSString -> JS t ()
\end{verbatim}
This can improve the quality of our code.
Section \ref{sec:object-model} will look into this in more detail.

\item 
Using a deep embedding like this gives opportunity 
for optimizations when producing the actual JavaScript.

\item
At the same time Sunroof offers an interface that is 
close to actual JavaScript making it easy to learn.
It utilizes the \Src{do}-notation (Section \ref{sec:js-monad}) and creates a 
simple foreign function interface to achieve this goal (TODO: SECTION?).
\end{itemize}

\Figure%
{fig:structure}%
{figures/sunroof-structure.pdf}%
{The structure of Sunroof.}
The scope of this paper is to show how Sunroof achieves these goals.
It will cover the most important parts of Sunroof and how they are 
implemented. Sunroofs structure can be seen in Figure \ref{fig:structure}.
In this paper we will:
\begin{itemize}
\item 
Go over the \JS-monad in Section \ref{sec:js-monad}. There
we will show how it is implemented and how we solved
the problem of restricting the types involved in the monadic 
computations.
\item
We will also discuss how we annotate the JavaScript
objects in Sunroof with types and also give the possibility 
to add custom types later on (Section \ref{sec:object-model}).
\item
Section \ref{sec:functions-continuations} will cover how
we model function and continuations in Sunroof and handle
them as first-class values in Sunroof and JavaScript.
\item
How we implemented the two threading models offered by Sunroof is explained 
in Section \ref{sec:threading-models}.
\item
The actual translation of Sunroof to JavaScript is handled in 
Section \ref{sec:compiler}. We will take a close look to the
different layers of abstractions and in detail explain the 
translation of selected language constructs. This is 
especially interesting in the light of continuations, which
are directly translated into JavaScript.
\item
As Sunroof is quite useful in combination with a server,
Section \ref{sec:server} will explain how we can utilize it 
to interleave Haskell and JavaScript computations as needed.
\item
TODO: Extended Example
\end{itemize}



\begin{comment}
Figure \ref{fig:structure} shows how Sunroof is structured.



An essential part of most modern web applications is Javascript.
It opens various possibilities to create browser based applications.
Examples for this development are Googlemail or Facebook.
These applications rely on Javascript heavily. What makes this 
development possible are the fast interpreters available in all
major browsers. Developing applications inside the browser has
the advantage of being independent from the underlying hard- and
software platform. The problem is shifted from the actual machine
to the browser. Also modern web technologies like HTML5 and CSS
give easy to use tools for developing user interfaces for the
mentioned application. They are standardized and improved over years
up to now. 

Yet, writing actual code in Javascript can be tedious and error prone 
due to its not-statically typed nature. Also the object-oriented
paradigm and the callback mechanism used instead of proper multi-threading 
can be a stumbling block for veteran Haskell programmers. Especially when
facing the lack of type safety.

To make the power of Javascript available to Haskell programmers 
we developed Sunroof. It enables Haskell programmers to write
Javascript using Haskell through a deep embedding. This 
approach has several advantages:

\begin{description}
\item[Type Safety:] Sunroof provides custom types for all major Javascript
types. Thus it gives type safety similar to what we are used to from 
normal Haskell programs when writing Javascript.
\item[Paradigms \& Patterns:] As a Haskell EDSL Sunroof follows functional programming 
paradigms and principals. It is possible to reuse common combinators and 
patterns in Sunroof.
\item[Similarity:] At the same time Sunroof offers a API for Javascript 
that allows to translate Javascript into Sunroof code easily.
\item[Optimization:] Due to Sunroofs deep embedded nature it is able
to apply optimizations like expression sharing when compiling to Javascript.
\end{description}

\NOTE{This might be a bit to enthusiastic.}

\end{comment}

\subsubsection{RESOURCES - REMOVE WHEN FINISHED}

\begin{itemize}
\item Give an example of Sunroof 
\item Explain the example
\item Talk about the general structure of Sunroof
\item Relate each part of the structure to a specific section:
  \begin{itemize}
  \item \JSA\ /\ \JSB: \JS-monad in section \ref{sec:js-monad};
  together with \JSI\ and contiuations used by the \JS-monad.
  \item Type Wrappers / Expr: Discussed in section \ref{sec:object-model}
  \end{itemize}
\item Details about functions and continuations are given in section \ref{sec:functions-continuations}
\item 
Sunroof was first documented in our previous 
workshop paper~\cite{Farmer:12:WebDSLs},
where the possibility of monadic reification was observed.
In this paper, we raised an unresolved issue:

Threading Model: Built on top of \JS-monad (section \ref{sec:threading-models})
\item Explain how the compiler works and what it does \ref{sec:compiler}
\item In section \ref{sec:server} we will look at the sunroof server 
\item Extended example in section \ref{sec:extended-example}
\end{itemize}

JavaScript is an imperative language with access to a wide range
of established and useful services, like graphical canvases and event
handling. JavaScript as a language also provides features that are
traditionally associated with functional languages, like first-class 
functions. We want to express JavaScript in Haskell, adding use
of Haskell's static typing, and gaining access to JavaScript services.
And we do so using the transitional functional programming 
mechanism for being imperative, a monad~\cite{Moggi:91:ComputationMonads}.


We want to use the Javascript API. Fast interpreters,
running in a browser advantage, 

We do not want to program in Javascript.

We start with an overview of the language, then discuss
how the language is implemented, and discuss interesting
techniques used inside Sunroof. We close with a more
complete web application.






\begin{comment}
\section{Example Sunroof Program}
\label{sec:simple-example}

A simple drawing program, that we build up step by step.
Perhaps bouncing ball.
Perhaps drawing line.
\end{comment}

 
\section{The JavaScript Monad}
\label{sec:js-monad}

The \JS-monad is
used to model sequences of JavaScript statements and
their side-effects. In that sense it is almost
exact analog for the Haskell \IO-monad, except there
is an extra phantom argument~\cite{Leijen:99:Phantom} 
that decides which threading model to use, as
discussed in Section \ref{sec:threading-models}.
For now we can ignore this extra argument.

The basic idea is that each binding becomes an
assignment to a fresh variable in JavaScript. Like
this the results of previous computations are passed on to 
later ones.
An example for this can be seen in Figure~\ref{fig:code-example}.
The binding \Src{name} is translated to the freshly generated
variable \Src{v0}.

Inside this simple example is a challenging problem -- where does
\Src{v0} come from? The bind inside the monadic \Src{do} is
unconstrained:
\begin{verbatim}
(>>=) :: JS t a -> (a -> JS t b) -> JS t b
\end{verbatim}
What we want is:
\begin{verbatim}
(>>=) :: (Sunroof a) => JS t a -> (a -> JS t b) -> JS t b
\end{verbatim}
Where \Src{Sunroof} constrains the bind to
arguments for which we can generate a JavaScript variable.
Counterintuitively, 
it turns out that a specific form of normalization allows 
the ``\Src{a}'' type to be constrained and the bind to 
be an instance of the standard monad class~\cite{Sculthorpe:13:ConstrainedMonads}.
Through this keyhole of {\em monadic reification},
the entire Sunroof language is realized. The 
normalization is done through the Operational package 
\cite{Apfelmus:10:Operational,Hackage:10:Operational}.

It provides the \Src{Program} data type as a monad that
can be equipped with custom primitives or effects.
We represent these primitives with the \JSI~instruction 
(Figure \ref{fig:jsi-definition}) type. It represents the 
abstract JavaScript instructions that are sequenced by the \Src{Program}
monad.
\begin{figure}
\begin{verbatim}
data JSI :: T -> * -> * where
  JS_Invoke   :: (SunroofArgument a, Sunroof r) 
              => a -> JSFunction a r -> JSI t r
  JS_Function :: (SunroofArgument a, Sunroof b) 
              => (a -> JS A b) -> JSI t (JSFunction a b)
  JS_Branch   :: (SunroofThread t, SunroofArgument a, Sunroof bool) 
              => bool -> JS t a -> JS t a  -> JSI t a
  JS_Assign   :: (Sunroof a) 
              => JSSelector a -> a -> JSObject -> JSI t ()
  ...
\end{verbatim}
\caption{Parts of the \JSI nstruction data type.}
\label{fig:jsi-definition}
\end{figure}
The type parameter \Src{t} in \Src{JSI t a} again represents 
the threading model and can be ignored up to Section \ref{sec:threading-models}. 
\Src{a} represents the type of the primitive's return value. 
Figure \ref{fig:jsi-definition} shows some of the instructions
in \JSI.
\Src{JS\_Invoke} calls a function, that has been created with \Src{JS\_Function}.
Branches are represented with \Src{JS\_Branch}. Assignments to variable
are represented by \Src{JS\_Assign\_}.

If it was not for our promise to provide an alternative 
threading model, the normalization through the Operational package
would be enough to offer a monad suitable for Sunroof.
But, because of our threading plans, there is more then just 
normalization going on behind the scenes. 
Actually the \JS-monad is a continuation monad over the 
\Src{Program}-monad.
\begin{verbatim}
data JS :: T -> * -> * where
  JS :: ((a -> Program (JSI t) ()) -> Program (JSI t) ()) -> JS t a
  ...
\end{verbatim}
The monad instance used is the standard implementation of 
a continuation monad (TODO: cite).


TODO: Maybe put this section after the section about the object model.

\begin{comment}
\subsubsection{RESOURCES - REMOVE WHEN FINISHED}

Basic Idea
\begin{itemize}
\item Put the current stuff into the introduction does not belong here
\item \JS-monad is analog of \IO-monad
\item Monad used to model sequences of statements with side-effect in 
JavaScript \cite{Moggi:91:ComputationMonads}
\item binding in Haskell becomes binding in JavaScript (more about this in section \ref{sec:compiler})
\item There also is a expression level (discussed in section \ref{sec:object-model})
\item Feels like a native monad, cf STM.
\item Can build abstractions on top of this.
\end{itemize}
Implementation
\begin{itemize}
\item Problems when using monad: Types need to be constrained to work
\item We use monad reification \cite{Apfelmus:10:Operational,Hackage:10:Operational}
\item But \JS-monad is a little bit more complicated
\item Show \JS-constructor: \Src{JS :: ((a -> Program (JSI t) ()) -> Program (JSI t) ()) -> JS t a}
\item Whats going on here?
\item \JS-monad is a continuation monad on the \Src{Program} type from operational
\item Monad instance is the standard implementation for a continuation 
monad (TODO: cite for continuation monad)
\item This is necessary, because we want to support different threading styles 
(discussed in chapter \ref{sec:threading-models})
\item For now think of it as producing a sequence/list of abstract instructions: \JSI
\item Look at the \JSI\ type and introduce it
\item Explain how it is a high-level representation of JavaScript
\item TODO: Look at all constructors? Or only the interesting ones (marked with "!")?
  \begin{itemize}
  \item ! Function and continuation are technically the same, but different instructions
  \item ! Reason for this is TODO?
  \item ! Why are there two different assignments?
  \item ! One is assignment to variable, other is assignment to the field of an object
  \item ! Eval, this looks weird. Forces evaluation and binding to a variable.
  \item ! Otherwise, we would use macro semantics in many places.
  \item ! Fix is provided to be able to write recursive functions
  \item ! Why isn't that possible without fix? problem boils down to 
        not observable sharing between statements and bindings. This
        leads compiler into infinite loops.
  \item Select selects a field
  \item Delete remove a field
  \item Invoke call a function/method
  \item Return for return of a function
  \item Comment writes some comment, helps for debugging
  \end{itemize}

\end{itemize}
\end{comment}





 
\section{JavaScript Object Model}
\label{sec:object-model}

One goal of Sunroof is to use Haskell's type system to
increase the correctness of expressed JavaScript.
At the same time we can not characterize 
all different types of objects in JavaScript, since 
users can create their own objects. Thus our 
system to type JavaScript needs to be extensible.

Our solution is to provide a basic \Src{Expr}ession
language to construct JavaScript expressions 
that has no associated type information. Simplified slightly, we have:
\begin{Code}
data Expr 
  = Lit String        -- Precompiled (atomic) JavaScript literal
  | Var Id            -- Variable
  | Apply Expr [Expr] -- Function application
  ...
\end{Code}
In reality, we generalize \Src{Expr} with the recursive type,
to facilitate the usage of observable sharing \cite{Gill:09:TypeSafeReification}
and allow sub-expression computations to be shared.

This core expression type 
is then wrapped to represent a more specific type. 
Each of these wrappers implements the \Src{Sunroof} type class.
\begin{Code}
class SunroofArgument a => Sunroof a where
  box   :: Expr -> a
  unbox :: a -> Expr
  ...
\end{Code}
It marks these types as possible values in JavaScript.
The \Src{SunroofArgument} prerequisite permits them 
to be function arguments (Section \ref{sec:functions-continuations}).

An example of this is \Src{JSString}, the representation
of JavaScript strings.
\begin{Code}
data JSString = JSString Expr
instance Sunroof JSString where
  box                = JSString
  unbox (JSString e) = e
\end{Code}
But what do we gain through a wrapper? We can
provide specific functionality for each distinct type.
Our example type \Src{JSString} has a \Src{Monoid} and a 
\Src{IsString} instance that are not provided for other 
wrappers, e.g. \Src{JSBool} or \Src{JSNumber}.
This approach was first introduced by 
Svenningsson \cite{Svenningsson:12:CombiningEmbedding}.

Table \ref{tab:sunroof-types} gives a summary of the 
prominent Sunroof types. Some types involve 
phantom types to give more type safety \cite{Cheney:03:FirstClassPhantomTypes}.
\begin{table}
\begin{center}
\begin{tabular}{r@{\quad}l@{\quad}l@{\quad}c}
\hline\rule{0pt}{12pt}%
  Constraint
  & Sunroof Type $\tau$
  & Haskell Analog \HaskellAnalog{$\tau$}
  & \Src{js} \\ \hline\rule{0pt}{12pt}%
  
  & \Src{()}       & \Src{()}     & $\checkmark$ \\
  & \Src{JSBool}   & \Src{Bool}   & $\checkmark$ \\
  & \Src{JSNumber} & \Src{Double} & $\checkmark$ \\
  & \Src{JSString} & \Src{String} & $\checkmark$ \\
  
  \Src{Sunroof $\alpha$}
  & \Src{JSArray $\alpha$} 
  & \Src{[$\HaskellAnalog{\alpha}$]}
  & \\
  
  \Src{SunroofKey $\alpha$}
  & \Src{JSMap $\alpha$ $\beta$}
  & \Src{Map $\HaskellAnalog{\alpha}$ $\HaskellAnalog{\beta}$}
  & \\
  \Src{Sunroof $\beta$} \\
  
  \Src{SunroofArgument $\alpha$}
  & \Src{JSFunction $\alpha$ $\beta$ }
  & \Src{$\HaskellAnalog{\alpha}$ $\rightarrow$ JS$_\Src{A}$ $\HaskellAnalog{\beta}$} 
  & \\
  \Src{Sunroof $\beta$} \\
  
  \Src{SunroofArgument $\alpha$}
  & \Src{JSMVar $\alpha$}
  & \Src{MVar $\HaskellAnalog{\alpha}$}
  & \\
  
  \Src{SunroofArgument $\alpha$}
  & \Src{JSChan $\alpha$}
  & \Src{Chan $\HaskellAnalog{\alpha}$}
  & \\[2pt]
\hline
\end{tabular}
\end{center}
\caption{Sunroof types and their Haskell pendant.}
\label{tab:sunroof-types}
\end{table} 
The smooth embedding of booleans and numbers is done through
the Boolean package \cite{project:boolean}.

The table shows that most basic Haskell types have counterparts in
Sunroof. To convert Haskell values into their 
counterparts we provide the \Src{SunroofValue} class.
\begin{Code}
class SunroofValue a where
  type ValueOf a :: *
  js :: (Sunroof (ValueOf a)) => a -> ValueOf a
\end{Code}
The type function
\Src{ValueOf} \cite{Chakravarty:05:AssociatedTypeSynonyms} 
provides the corresponding Sunroof type.
\Src{js} converts a value from Haskell to Sunroof. 
By design \Src{SunroofValue} does
only provide instances for values that can be converted in a pure
manner. Some types in JavaScript are referential 
transparent according to \Src{==} while others, like general objects,
are not. As an example, if you call \Src{new Object()} twice you get the 
same empty object, but when compared by \Src{==} they are different. They
are not identical, because in this case reference equality is checked
instead of value equality. We call this observable allocation and handle 
it as a side-effect which may only occur in the \JS-monad. 

This approach ensures to bind a new value to a variable,
instead of creating copies of that value everywhere
it is used. This resolves some unwanted macro behavior 
of Sunroof.

% JSTuple is not a major selling point nor is it essential to sunroof.
\begin{comment}
Sunroof also offers the ability to work with record like data structures.
For this purpose Sunroof offers the \Src{JSTuple} type class.
\begin{Code}
class Sunroof o => JSTuple o where
  type Internals o
  match :: (Sunroof o) => o -> Internals o
  tuple :: Internals o -> JS t o
\end{Code}
If you have a record of Sunroof data that you want to
encode as a JavaScript object you can provide this ability 
by implementing \Src{JSTuple}. The \Src{Internals} type function
delivers your record type. Encoding that record as a \Src{Sunroof}
value is done through \Src{tuple}. 
Because of the observable allocation issue, we have to be 
inside the \JS-monad to do this.
Decomposing the encoding is done with \Src{match}. Because the
\Src{JSTuple} idiom is meant to represent immutable data structures, this 
can be done in a pure manner although there are ways to mutate 
values referenced by the decomposed record, since they are only 
references to the actual data in JavaScript.
\end{comment}

\begin{comment}
\subsubsection{RESOURCES - REMOVE WHEN FINISHED}

General Objects Types and Expressions
\begin{itemize}
\item We want to use Haskell's type system to give use type safety in JavaScript/Sunroof.
\item JavaScript is untyped / dynamically typed \TODO{what is it actually? cite/reference?}
\item We can not represent all types in JavaScript
\item We have to give a way to add types later on
\item Use approach from \cite{Svenningsson:12:CombiningEmbedding}
\item There is a core expression language that is used to describe how and expression is built
\item We wrap that core type into wrappers
\item Introduce \Src{Sunroof} to mark wrappers
\item Wrappers have specific functionality for certain type
\item Wrappers can have phantom type to give more type safety
\item Example \Src{JSString} and \Src{JSArray a}
\item Maybe go through example type like \Src{JSString} or \Src{()}.
\item Unavoidable need to cast types some times: \Src{cast}
\item Adding a new type can be done mechanically: Template Haskell
\end{itemize}
Haskell to Sunroof conversion
\begin{itemize}
\item Many haskell types have equivalent or similar types in JavaScript
\item So we offer \Src{SunroofValue}; It connects a Haskell value with 
its JavaScript companion through a type function (cite type functions)
and gives a function to convert: \Src{js}
\item Introduce \Src{SunroofValue}
\item Conversion is pure for most types since atomic values 
are created when converting.
\item Exception \Src{JSFunction}, same reason as for \Src{JSTuple} and \Src{tuple}
\item Show table with types as an overview. Small comment paragraph about table.
\item \Src{JSArray} not possible because of conflicts with \Src{String}
instance (minor point); also issue of mutability of arrays; observable 
allocation issue like with \Src{JSTuple}.
\item Forward reference to section \ref{sec:threading-models} for \Src{JSMVar}
and \Src{JSChan}
\end{itemize}
JSTuple and records in JS
\begin{itemize}
\item Design decision: We do not want to introduce internal
structure for types. (Tuples aren't JS types)
\item \Src{JSTuple} exists to introduce types with custom structure
\item Introduce \Src{JSTuple}
\item manages composition and decomposition of custom types.
\item Meant to codify immutable records in JavaScript
\item They can only be changed
by decomposing them into Haskell and recreating a new structure
with other values.
\item Useful to manage more complex data structures in JavaScript.
\item Decomposition is pure operation: \Src{match}
\item Justification lies in the fact that they are meant to be 
immutable. Although there are possibilities to break this.
\item Composition is monadic effect: \Src{tuple}
\item It captures observable allocation and ensures that a reference to the
allocated value is created and used afterwards.
\item If it was not monadic the let binding would be like a macro
that reallocated the same object at each point. This 
is not desired behavior in most use-cases
\end{itemize}
\end{comment}








 
\section{Functions and Continuations}
\label{sec:functions-continuations}

Functions are first class values in Haskell and JavaScript.
To represent a function in Sunroof, we introduce the type 
\Src{JSFunction $\alpha$ $\beta$}. This represents a function
$\alpha \rightarrow \beta$ in JavaScript. Functions should
be able to take more then one argument, which means a 
\Src{Sunroof $\alpha$} constraint would be to restrictive.
Thus we introduced \Src{SunroofArgument} to constrain the 
types that may be used as arguments for functions.
\begin{Code}
class SunroofArgument args where
  jsArgs   :: args -> [Expr]
  jsValue  :: (UniqM m) => m args
  ...
\end{Code}
It enables use to convert each argument into its expression
through \Src{jsArgs}, which is used to supply the
arguments to a function call. \Src{jsValue} generates
new unique names for each argument, which is needed when compiling
the function itself to a value. 

Since the usefulness of partial application is questionable in
the context of JavaScript, we choose to only permit uncurried functions,
by providing instances for tuples of sunroof values.
\begin{Code}
instance (Sunroof a, Sunroof b) => SunroofArgument (a,b) where
  jsArgs ~(a,b) = [unbox a, unbox b]
  jsValue = liftM2 (,) jsVar jsVar
\end{Code}
Remember that each \Src{Sunroof} value already has to be 
a \Src{SunroofArgument}, which also enables us to pass a single argument
to a function. 

To create a JavaScript function we provide the \Src{function} combinator.
\begin{Code}
function :: (SunroofArgument a, Sunroof b) 
         => (a -> JS A b)  -> JS t (JSFunction a b)
\end{Code}
As a function can have side-effects its computation and result has to be 
expressed through the \JS-monad. Again, due to observable allocation,
the creation of a function is considered a side-effect.

A function can be applied to arguments through the \Src{apply} combinator
or the \Src{\$\$} operator.
Functions can only be applied in the \JS-monad since they can have side-effects.
\begin{Code}
apply, ($$) :: (SunroofArgument a, Sunroof b) 
            => JSFunction args ret -> args -> JS t ret
\end{Code}
Creation and application are implemented using the \Src{JS\_Function}
and \Src{JS\_Invoke} instructions introduced in 
Figure \ref{fig:jsi-definition}.

\Src{JSContinuation} models continuations as they are used 
inside the \JS-monad. They were introduced to work and actively manipulate 
continuations on the Sunroof level. Technically they are only
specializations of the general function type, that are 
restricted to the second threading models. As the continuations
are meant to be a representation of side-effects -- 
ongoing computations inside the \JS-monad -- and might 
not terminate, they do not return a value. As with functions 
there is a combinator to create and apply a continuation.
\begin{Code}
continuation :: (SunroofArgument a) 
             => (a -> JS B ()) -> JS t (JSContinuation a)
goto         :: (SunroofArgument a) 
             => JSContinuation a -> a -> JS t b
\end{Code}
The major difference is that a call to \Src{goto} will never
return, as it execute the given continuation and abandons the 
current one. This allows \Src{goto} to be fully polymorphic
on its result. 

To give access to the internal continuations of the \JS-monad
Sunroof offers the powerful call-with-current-continuation 
function \Src{callcc}.
\begin{Code}
callcc :: SunroofArgument a 
       => (JSContinuation a -> JS B a) -> JS B a
\end{Code}
It enables us to use the current continuation, which 
models everything that would usually happen after the call
to \Src{callcc}. 
\begin{Code}
callcc f = JS $ \ k -> unJS 
    (continuation (\a -> JS $ \ _ -> k a) >>= f) k

unJS :: JS t a -> (a -> Program (JSI t) ()) -> Program (JSI t) ()
\end{Code}
The implementation of \Src{callcc} is interesting,
because it shows how the \Src{Program}-continuation is translated 
into a \Src{JSContinuation} that is passed to the given function \Src{f}.
Section \ref{sec:threading-models} will show why this function
is important for Sunroof and what it is used for.

Functions and continuations are similar and connected 
to each other, as can be seen in Figure \ref{fig:func-cont}.
\Figure%
{fig:func-cont}%
{figures/sunroof-func-cont.pdf}%
{How functions and continuations relate between the Haskell- and Sunroof-domain.}%
We can go back and forth between the Haskell and the Sunroof
representation of a function or continuation. But once a function
was specialized to a continuation it is not possible to go back,
because continuations only model the side-effect, but do 
not return anything.


\begin{comment}

\begin{table}
\caption{Reifying and calling JavaScript functions}
\begin{center}
\begin{tabular}{r@{\quad}c@{\quad}c@{\quad}c@{\quad}c}
\hline\rule{0pt}{12pt}%

                & Monadic Function      & Reification   & Object in     & Invocation\\
                & in Haskell            & Function      & Javascript    & Function\\
\hline\rule{0pt}{12pt}%
  Functions
  & $\alpha\rightarrow\ $\Src{JS}$_\Src{A}~\beta$
  & \Src{function}
  & \Src{JSFunction}~$\alpha~\beta$
  & \Src{apply} \\
  Continuations
  & $\alpha\rightarrow\ $\Src{JS}$_\Src{B}~\Src{()}$
  & \Src{continuation}
  & \Src{JSContinuation}~$\alpha$
  & \Src{goto}\\
\hline
\end{tabular}
\end{center}
\end{table}

\end{comment}

\begin{comment}
\subsubsection{RESOURCES - REMOVE WHEN FINISHED}

\begin{itemize}
\item Functions are first class members in Haskell and JavaScript
\item But in JavaScript they are limited (no partial application)
\item That leads to type for functions \Src{JSFunction a b}
\item Functions can take more then one argument; \Src{Sunroof} insufficient: \Src{SunroofArgument}
\item \Src{SunroofArgument} is prerequisite for \Src{Sunroof},
because all JS objects are parameters but pairs of them are not 
JS objects.
\item Introduce \Src{SunroofArgument}.
\item So function in JS are not curried (partial application is questionable in JS)
\item Create a function with \Src{function} combinator.
\item Function creation is a monadic effect for the same reason 
a record creation (\Src{JSTuple}). It represents observable
allocation and we usually want one reference to a function 
instead of coping its definition everywhere.
\item Calling a function or method can have a side-effect
\item That is why the operator \Src{\$\$} for function application is monadic to.
\item All these operations do is to produce \Src{JS\_Function} and \Src{JS\_Invoke}
instructions.
\item Continuations are basically the same as functions at this level.
\item Created with \Src{continuation} and called with \Src{goto}.
\item Implementation of \Src{goto} is interesting. Ignores 
the current continuation in the \JS-monad and continues with the given one
by actually invoking the JavaScript function representing it.
\item Major difference in translation between both on compiler level (section \ref{sec:compiler})
\item \TODO{Further detail on continuations; Here or somewhere else?}
\end{itemize}
\end{comment}







 
\section{Threading Models}
\label{sec:threading-models}

JavaScript uses a callback centric model of computation. There
is no concurrency, only a central loop that executes callbacks
when events occur.

In contrast Haskell has real concurrency and wide-spread 
abstractions for synchronization, e.g. \Src{MVar}s and \Src{Chan}s
\cite{Jones:96:ConcurrentHaskell}.
So the question arises: do we generate atomic JavaScript code, 
and keep the callback centric model, or generate JavaScript
using CPS \cite{Claessen:99:PoorMansConcurrencyMonad}, 
and allow for blocking primitives and
cooperative concurrency. The latter, though more powerful, 
precluded using the compiler to generate
code that can be cleanly called from native JavaScript.
Both choices had poor consequences.

We decided to explicitly support both,
and make both first class threading strategies in Sunroof.
In terms of user-interface, we parameter the \JS-monad
with a phantom type that represents the threading model used, 
with \Src{A} for \Src{A}tomic,
and \Src{B} for \Src{B}locking threads. 
Atomic threads are classical JavaScript computations that
can not be interrupted and actively use callback
mechanism. Blocking threads can
support suspending operations and cooperative concurrency
abstractions as known from Haskell. By using phantom
types, we can express the necessary
restrictions on specific combinators, as well
as provide combinators to allow both types of
threads to cooperate.

The blocking model uses the callback mechanism and hides it 
behind abstractions.
This implies that every atomic computation can be converted into 
a blocking computation. \Src{liftJS} achieves this.
\begin{Code}
liftJS :: Sunroof a => JS A a -> JS t a
\end{Code}

When suspending, we register our current
continuation as a callback to resume later. This gives other 
threads (registered continuations) a chance to run.
Of course, this model depends on cooperation between the threads,
because a not terminating or suspending thread will keep others from running.

There are three main primitives for the blocking model.
\begin{Code}
forkJS      :: SunroofThread t1 => JS t1 () -> JS t2 ()
threadDelay :: JSNumber -> JS B ()
yield       :: JS B ()
\end{Code}
They can all be seen as analogues of their \IO~counterparts.
\Src{forkJS} resembles \Src{forkIO}.
It registers the continuation of the given computation as a callback. 
\Src{yield} suspends the current thread by 
registering the current continuation as a callback,
giving other thread time to run.
\Src{threadDelay} is a form of \Src{yield} that sets 
the callback to be called after a certain amount of time.
We rely on the JavaScript function \Src{window.setTimeout} 
\cite{whatwg:timers} to register our callbacks.

The class \Src{SunroofThread} offers functions to retrieve the 
current threading model (\Src{evalStyle}) and to create a possible
blocking computation (\Src{blockableJS}).
\begin{Code}
class SunroofThread (t :: T) where
  evalStyle    :: ThreadProxy t -> T
  blockableJS :: (Sunroof a) => JS t a -> JS B a
\end{Code}
Based on this primitive combinators we also offer a Sunroof 
version of \Src{MVar} and \Src{Chan}: \Src{JSMVar} and \Src{JSChan}.
\begin{Code}
newMVar      :: (SunroofArgument a) => a -> JS t (JSMVar a)
newEmptyMVar :: (SunroofArgument a) => JS t (JSMVar a)
putMVar      :: (SunroofArgument a) => a -> JSMVar a -> JS B ()
takeMVar     :: (SunroofArgument a) => JSMVar a -> JS B a

newChan   :: (SunroofArgument a) => JS t (JSChan a)
writeChan :: (SunroofArgument a) => a -> JSChan a -> JS t ()
readChan  :: (SunroofArgument a) => JSChan a -> JS B a
\end{Code}
Both implementation use arrays to store the waiting readers and
writers in form of continuations. Note that all functions
are able to handle \Src{SunroofArgument}s, not just \Src{Sunroof}
types. This is possible, because the computations themselves
(their current continuation) are
stored in the lists through \Src{callcc}.
When a arguments is written either the waiting
continuation is called with those arguments or a new continuation 
that applies an incoming one with those arguments is created.

\begin{comment}
\subsubsection{RESOURCES - REMOVE WHEN FINISHED}

\begin{itemize}
\item JavaScript does not have a threading model.
\item There is only one thread and callbacks can be registered to be
called when events happen once thread of execution has ended.
\item Using our CPS in the \JS-monad and translating it directly 
to JavaScript we can simulate/emulate a threading model similar to
Haskell's.
\item Basic idea is to use callbacks as mechanism to continue
suspended computations (in form of continuations).
\item \TODO{Is there a paper about this technique?}
\item We support both styles.
\item Decision which one to use is made by the first 
type parameter of the \JS-monad. A - atomic / B - blocking
\item Compiler comes in these two flavors: \Src{sunroofCompileJSA} and \Src{sunroofCompileJSB}
\end{itemize}
Concurrency primitives
\begin{itemize}
\item A primitives for the blocking threading model we provide:
\Src{forkJS}, \Src{threadDelay} and \Src{yield}
\item Show type signatures for the three methods
\item \Src{forkJS} can be thought of an equivalent to \Src{forkIO},
it registers a callback for the given computation, so 
it is done once the current computation is done.
\item \Src{threadDelay} registers a callback to execute 
the rest of this computation (the rest of this continuation)
as soon as the given amount of time has passed. This thread 
of computation ends here.
\item \Src{yield} sets the timeout of \Src{threadDelay} to zero.
\end{itemize}
Concurrency abstractions
\begin{itemize}
\item As a further abstraction we also provide \Src{JSMVar} and \Src{JSChan}.
\item These are equivalents of \Src{MVar} and \Src{Chan} in Sunroof.
\item Both are expressed purely in terms of the above primitives 
and continuations.
\item Both utilize two lists to store the processes that 
are waiting for data and those that are trying to write data.
and register them to be run through \Src{forkIO} and \Src{goto}
as needed.
\item Show interface for both types
\item Remark how they enforce \JSB\ to ensure that 
CPS is translated down to JavaScript
\item Spare the implementation as it is not interesting in the 
scope of this paper.
\item \TODO{Reference MVar and Chan papers / 
are there parallels to their implementation?}
\item \TODO{Cite: Continuations to store computations?}
\end{itemize}
\end{comment}







 
\section{The Sunroof Compiler}
\label{sec:compiler}

Given the language, and monadic-reification, how do we compile this language?
Through the \JS-monad we produce a \Src{Program (JSI t) ()} instance. We 
translate such a program into a list of statements (\Src{Stmt}) by matching over 
the \JSI constructors.
\begin{Code}
data Stmt 
  = AssignStmt Rhs Expr       -- Assignment
  | DeleteStmt Expr           -- Delete reference
  | ExprStmt Expr             -- Expression as statement
  | ReturnStmt Expr           -- Return statement
  | IfStmt Expr [Stmt] [Stmt] -- If-Then-Else statement
  | WhileStmt Expr [Stmt]     -- While loop
  | CommentStmt String        -- Comment
\end{Code}
The constructors of \Src{Stmt} are straight forward and
directly represent the different statements one can write
in JavaScript.

To translate a \Src{Program (JSI t) ()} into statements we
have to translate it into a \Src{ProgramView} since 
Operational prohibits us from working on \Src{Program} directly.
\Src{ProgramView} can then be taken apart to translate 
the \JSI nstructions inside.
\begin{Code}
compile :: Program (JSI t) () -> CompM [Stmt]
compile p = eval (view p)
  where
    eval :: ProgramView (JSI t) () -> CompM [Stmt]
    ...
\end{Code}
The compilation monad \Src{CompM} provides us with 
the compiler options and a supply of fresh integer
values to generate new variable when needed.

To get a feel for how the compiler works, lets take a look at 
the case for \Src{JS\_Assign}. Recall its signature from 
Figure \ref{fig:jsi-definition}:
\begin{Code}
JS_Assign :: Sunroof a => JSSelector a -> a -> JSObject -> JSI t ()
\end{Code}
This instruction assigns a new value \Src{a} to the attribute
selected by the \Src{JSSelector a} of the \Src{JSObject}.
\begin{Code}
eval (JS_Assign sel a obj :>>= k) = do
  (stmts0,val) <- compileExpr (unbox a)
  stmts1 <- compile (k ())
  return (  stmts0 
         ++ [ AssignStmt (DotRhs (unbox obj) 
                                 (unboxSelector sel)) 
                         val ] 
         ++ stmts1 )
\end{Code}
First we compile the expression that will be assigned. That
produces a series of statements to compute the expression and 
the expressions value \Src{val}.
Then we compile the rest of the Sunroof computation \Src{k}. Since 
the assignment does not produce a interesting value (it has type \Src{JSI t ()}),
we can pass unit to \Src{k}.
We return the concatenation of the produced statements with a statement 
for the assignment in between. The \Src{unbox} functions reveal the 
\Src{Expr} inside of our Sunroof wrapper types.

Lets now look at a more interesting part of the compiler: The translation of 
branches.
Recall the \JSI constructor for branches from Figure \ref{fig:jsi-definition}.
\begin{Code}
JS_Branch :: (SunroofThread t, SunroofArgument a, Sunroof bool) 
          => bool -> JS t a -> JS t a  -> JSI t a
\end{Code}
Aside of the fact that branches may return \Src{SunroofArgument}s
instead of just \Src{Sunroof} values the translation seems to 
be straight forward (Figure \ref{fig:branch-translation}).
\begin{figure}[h]
\begin{Code}
eval (JS_Branch b c1 c2 :>>= k) = do
  (src0, res0) <- compileExpr (unbox b)
  res :: a <- jsValue
  let bindResults :: a -> JS t ()
      bindResults res' =
        sequence_ [ single $ JS_Assign_ v (box $ e :: JSObject)
                  | (Var v, e) <- jsArgs res `zip` jsArgs res' ]
  src1 <- compile $ extractProgramJS bindResults c1
  src2 <- compile $ extractProgramJS bindResults c2
  rest <- compile (k res)
  return (src0 ++ [ IfStmt res0 src1 src2 ] ++ rest)
\end{Code}
\caption{Naive translation of branches in Sunroof.}
\label{fig:branch-translation}
\end{figure}
First we generate the statements to compute the branching condition.
Then we generate unique names for each of the values returned in 
either branch. \Src{bindResults} generates the assignments of
the returned values to the generated variables. When compiling 
the branches we use the function \Src{extractProgramJS}.
\begin{Code}
extractProgramJS :: (a -> JS t ()) -> JS t a -> Program (JSI t) ()
extractProgramJS k m = unJS (m >>= k) return
\end{Code}
It passes the result of a computation \Src{m} 
into the given function \Src{f} and
closes the continuation inside \JS~with \Src{return},
such that the result is a \Src{Program} containing
all instructions of \Src{m >>= k}.

After we have compiled both branches we compile the rest and
construct the list of statements in a canonical fashion.

This works perfectly when we work in the atomic threading model,
but it causes problems when inside the blocking model. To 
see what can go wrong lets look at a small example.
\TODO{Explain callcc in continuation section}
\begin{Code}
branchFail = do
  b <- ifB (true :: JSBool)
           (callcc $ \k -> do comment "True Case"
                              goto k true :: JSB JSBool)
           (callcc $ \k -> do comment "False Case"
                              goto k false :: JSB JSBool)
  fun "CallAfterIf" $$ b
\end{Code}
In either case of the initial branch we want to do something
with our current continuation. We insert a comment for 
our orientation and then call our current continuation to 
proceed with a boolean value that depends on the branch.
After our branch we call a function with the returned boolean value.
\TODO{Is there a simpler example that demonstrates the problem?}
What will this be translated to?
\begin{Code}
var v8 = function() {
  if(true){
    var v2 = function(v1) {
      var v0 = v1;
    };
    /* True Case */
    v2(true);
  } else {
    var v5 = function(v4) {
      var v0 = v4;
    };
    /* False Case */
    v5(false);
  }
  CallAfterIf(v0);
};
v8();
\end{Code}
We can see the branching statement and the function call afterwards.
Inside each branch we can also see a function definition the inserted 
comment and a call to the defined function. The function definitions 
\Src{v2} and \Src{v5} correspond to the continuations \Src{k} in 
our \Src{callcc} calls. So we can see that the current continuations 
captured the assignments that have to be done at the end of either branch 
to pass on the result value. But the assignments are hidden inside 
the scope of the continuations and, because of that, \Src{v0} is 
not defined when \Src{CallAfterIf} is applied to it.

So how can we fix this? If we are working within the blocking 
threading model, we have to handle branches differently.
\begin{Code}
eval (JS_Branch b c1 c2 :>>= k) = 
  case evalStyle (ThreadProxy :: ThreadProxy t) of
    A -> compileBranch_A b c1 c2 k
    B -> compileBranch_B b c1 c2 k
\end{Code}
\Src{evalStyle} is provided by the class \Src{SunroofThread} and
provides the threading model currently used. The call to \Src{compileBranch\_A}
executes our naive definition from Figure \ref{fig:branch-translation}.
\begin{Code}
compileBranch_B b c1 c2 k = do
  fn_e <- compileContinuation $
            \a -> blockableJS $ JS $ \k2 -> k a >>= k2
  fn <- newVar
  (src0, res0) <- compileExpr (unbox b)
  src1 <- compile $ extractProgramJS (apply (var fn)) c1
  src2 <- compile $ extractProgramJS (apply (var fn)) c2
  return ( [mkVarStmt fn fn_e] ++ src0 ++ [ IfStmt res0 src1 src2 ])
\end{Code}
\TODO{Talk about \Src{blockableJS}}
We can see that the rest of our computation is captured in a
continuation \Src{fn\_e} that takes the results of our branching
as arguments. We assign that computation to the new variable \Src{fn}
to avoid the duplication of the continuation in the produced sources.
The key difference lays in the call to \Src{extractProgramJS}. Instead
of creating bindings for each returned value we apply the produced
continuation to the returned values and like that make them visible for
the ongoing computation, even if we capture our current continuation. 
Now the produced code looks like this:
\begin{Code}
var v11 = function() {
  var v2 = function(v0) {
    CallAfterIf(v0);
  };
  if(true){
    var v5 = function(v3) {
      v2(v3);
    };
    /* True Case */
    v5(true);
  } else {
    var v9 = function(v7) {
      v2(v7);
    };
    /* False Case */
    v9(false);
  }
};
v11();
\end{Code}
We can see that the function call after the branch is captured by the 
continuation \Src{v2} and that continuation is called with the 
return values as parameters inside each branch.

Of course, we could compile every branch with the 
continuation variant. But that would unnecessarily 
obfuscate the produced code and lead to many superfluous 
indirections.
\TODO{Look at another interesting part: Fix? Function?}

The compiler can be called through two functions. 
One for each threading model. \TODO{Maybe remove the string argument from JSB}
\begin{Code}
sunroofCompileJSA :: Sunroof a 
                  => CompilerOpts -> String -> JS A a  -> IO String
sunroofCompileJSB :: CompilerOpts -> String -> JS B () -> IO String
\end{Code}

\begin{comment}
Given the language, and monadic-reification, how do we compile this language?
Figure \ref{fig:structure} shows how Sunroof is structured.
On the lowest level we provide an untyped expression language \Src{Expr}
that describes JavaScript expressions. 
To provide type safety when using Sunroof we create
wrappers for each type we want to represent, e.g. \Src{JSNumber} or \Src{JSString}.
The \Src{Sunroof} type class provides an 
interface to create wrapped and unwrapped
instances of our expressions. Based on the wrappers we can provide 
operations specific to a certain type, e.g. a \Src{Num} instance
for \Src{JSNumber} or a \Src{Monoid} instance for \Src{JSString}.



This technique enables us to utilize  Haskells type system when writing JavaScript
and offers an easy way to add new types when needed~\cite{Svenningsson:12:CombiningEmbedding}.
By using phantom types we can also provide more advanced types,
like \Src{JSArray a}.

The next layer provides JavaScript instructions through the type \JSI.
They represents abstract statements. While expressions and values
represented with type wrappers are assumed to be free of side-effects,
the instructions model side-effects in JavaScript. Examples for Instructions
are assignment of an attribute or the application of a function.
\begin{Code}
data JSI :: T -> * -> * where
  JS_Assign :: (...) => JSSelector a -> a -> JSObject -> JSI t ()
  JS_Invoke :: (...) => a -> JSFunction a r -> JSI t r
  JS_Branch :: (...) => bool -> JS t a -> JS t a  -> JSI t a
  ...
\end{Code}
The \JS-monad with its combinators builds a sequence of 
\JSI{}nstructions through the operational
package~\cite{Hackage:10:Operational,Apfelmus:10:Operational}.
All constraints required on instructions are introduced by their 
constructors.
As mentioned earlier the \JS-monad comes in two threading flavors, 
parameterized using a phantom type.
Internally the \JS-monad is implemented using CPS on the 
underlying \Src{Program} type from Operational. 
\begin{Code}
data JS :: T -> * -> * where
  JS   :: ((a -> Program (JSI t) ()) -> Program (JSI t) ()) 
       -> JS t a
  ...
\end{Code}
For atomic
computations we just produce a list of instructions from the continuation. 
When translating possibly blocking code we directly translate that continuation
into JavaScript functions. This gives us the ability to handle 
computations as values in JavaScript and store them if needed.

Blocking operations just store the rest of their computation in a queue.
When the event to unblock occurs the pending computation is registered 
as a callback that will be executed as soon as the current computation
is done.

On top of our \JS-monad we provide ways of specifying (typed)
interfaces to JavaScript capabilities, a Foreign Function Interface.
\begin{Code}
alert :: JSString -> JS t ()
alert = fun "alert"

getElementById :: JSString -> JSObject -> JS t JSCanvas
getElementById = invoke "getElementById"
\end{Code}
Notice, that calling a JavaScript function or method is done by giving its
name to one of the provided combinators. Types can be specialized using 
a Haskell type annotation. A flexible and easy to use approach.
\end{comment}

\begin{comment}
\subsubsection{RESOURCES - REMOVE WHEN FINISHED}

Compiling
\begin{itemize}
\item Short introduction to the compiler interface (signature).
\item Core work done by translating \Src{Program (JSI t) ()}
into a list of \Src{Stmt}s.
\item Introduce statement type, give a short description of 
each constructor (just in the comments)
\item Basic idea: Each \JSI~nstruction is translated 
into a sequence of statements and these are then
concatenated together.
\item Look at interesting parts
\item \TODO{Which parts are interesting? Most parts are too technical}
\item Show how a branch is compiled \TODO{\Src{extractProgramJS} lets things look messy}
\item Talk about how \Src{JS\_Fix} is compiled.
\item \TODO{Understand why fix works.}
\item Difference between compilation of a function and a continuation
\item Function \Src{\textbackslash a -> JS \$ \textbackslash k -> singleton (JS\_Return a) >>= k}
\item \TODO{Why \Src{>>= k}?}
\item Return the result of the current continuation
\item Continuation \Src{\textbackslash \_ -> JS \$ \textbackslash k -> k ()}
\item Just execute it instead of passing it on further. No return value!
\item We use expression sharing through observable sharing (reference Andys paper).
\end{itemize}

\begin{itemize}
\item If we transliterate, we have straight line code, can not pause.
    (Wait for Mvar, for example)
\item If we CPS translate, we can use continuations to capture the
   notion of a paused thread. Works well.
   Problems:
  \begin{itemize}
   \item Can not translate functions, how do they get there return value
   \item (Assumes straight line code.)
   \item Also, the code becomes unreadable to anyone except a die-hard 
       compiler freak.
  \end{itemize}
\end{itemize}
 
Choice:
\begin{itemize}
\item We support both!
\item Phantom argument to JS
\item A = Atomic, B = Blockable.
\end{itemize}
\end{comment}






 
\section{The Sunroof Server}
\label{sec:server}

The Sunroof compiler can compile JavaScript than can be used
stand-alone inside a web application. But Sunroof really comes
into its own when used with the Sunroof server. There
are three major functions in our server.

\begin{verbatim}
sunroofServer :: SunroofServerOptions -> SunroofApp -> IO ()
syncJS  :: SunroofResult a => SunroofEngine 
        -> JS t a -> IO (ResultOf a)
asyncJS :: SunroofEngine -> JS t () -> IO ()
\end{verbatim}        

\Src{sunroofServer} starts a small web server,
that calls the callback function for each request.
\Src{syncJS} and \Src{asyncJS} allow the Sunroof programmer
to remotely execute monadic JavaScript from the server.
\Src{ResultOf} is a type-function, that maps the 
Sunroof type to a corresponding Haskell type.






 
\section{Case Study - A Small Calculator}
\label{sec:extended-example}

To see how Sunroof works in practice, we will look into the 
experience we gathered when writing a small calculator
for arithmetic expressions (\FigRef{fig:example-application}). 
We use Sunroof to display our interface
and the results of our computation. Haskell will be used to parse the 
arithmetic expressions and calculate the result. The Sunroof server 
will be used to implement this JavaScript/Haskell hybrid.

\FigureS%
{fig:example-application}%
{figures/example-application.png}%
{The example application running on the Sunroof server.}%
{scale=0.6}

The classical approach to develop an application like this would have 
been to write a server that provides a RESTful interface and replies 
through a JSON data structure. 
The client side of that application would have been written in JavaScript
directly.
This can be seen in \FigRef{fig:example-structure}.

\Figure%
{fig:example-structure}%
{figures/example-structure.pdf}%
{Classical structure and Sunroof structure of a web application.}

How does Sunroof improve or change this classical structure?
First of all, in Sunroof you write the client-side code together with
your server application within Haskell. In our example, all code 
for the server and client is in Haskell. The control logic 
for the client side is provided through the server.
This leads to a tight coupling between both sides. 
This also shows how Sunroof blurs the border between the server 
and client side. You are not restricted by an interface or language 
barrier. If you need the client to do something, you can just 
send arbitrary Sunroof code to execute in the client.

\TabRef{tab:example-statistics} contains a few statistics 
about the size of the code in each part of the client.

The client-server response loop shuffles new input to the server 
and executes the response in the client.
%
Data conversion is needed, because pure Haskell data types
cannot be handled in Sunroof and vice versa. There still
exists a language barrier between JavaScript and Haskell. 
Code to convert between two essentially equal data structures on 
each side must be written, as well as representations of Haskell 
structures in Sunroof. However, there is great potential in automatically 
generating this code using techniques such as template Haskell
\cite{Sheard:02:TemplateMetaProgrammingHaskell}.

The code for displaying the results is basically a 
transliteration of the JavaScript that you would write for this 
purpose.
The transliteration used here is not very appealing. 
In the future, this code can be generated through higher-level 
libraries. Sunroof is intended to deliver a foundation for
this purpose.

The rest of our code to parse the arithmetic expression and calculate 
results is classical Haskell code. 

\begin{table}[t]
\begin{center}
\begin{tabular}{l@{\quad}r@{\quad}r}
\hline\rule{0pt}{12pt}%
Part of Application & Lines of Code & Percentage \\[2pt]
\hline\rule{0pt}{12pt}%
Response loop & 25 & 6.5\% \\[2pt]
Data conversion & 85 & 22.0\% \\[2pt]
Rendering & 190 & 49.5\% \\[2pt]
Parsing and interpretation & 85 & 22.0\% \\[2pt]
\hline
\end{tabular}
\end{center}
\caption{Lines of code needed for the example.}
\label{tab:example-statistics}
\vspace{-0.5cm}
\end{table} 


\section{Background - RESOURCE COLLECTION}
\label{sec:background}

We chose to implement Sunroof as deep embedded DSL for different
reasons. Porting a compiler, like GHC, to Javascript would have
cost a lot of effort. There is no clear way to port the existing
API and there are many language features that would be difficult to
translate. Providing a foreign function interface would 
introduce the cost of translating back- and forward. 
\TODO{Not even sure how this would work or where cost comes in.}
Another possibility would be to develop a custom language. 
Besides that people would need to learn a third language for this 
approach, it would involve effort similar to porting a compiler.
As a deep embedded DSL people only have to know Haskell to learn
the new language. The cost of developing the DSL remains light, 
because we can build upon a mature language and its features.

Choices when compiling to Javascript
\begin{itemize}
\item Port a compiler to Javascript - issues, mismatching API, other efforts.
\item Provide an FFI - back and forward cost.
\item Compile a custom, cut down FP language. - example
\item DSL, restrictive, syntactically, how do we handle binding?
\end{itemize}

\subsection{Basic Idea}

As we have seen in the previous example application Sunroof uses a 
monad to model computations that can be translated to Javascript.
A binding operation is thought to be an assignment to 
a fresh variable in the reified language.



Design objectives and contributions.
\begin{itemize}
\item 
Specifically, we want to explore how similar to a native Haskell monad,
like the IO or STM monads, can we make the JS monad feel to Sunroof users.
\item We want to reflect objects. (easy)

\item We want also investigate the relationship between Haskell functions
and Javascript functions. In particular, there is an isomorphism
between monadic functions in Haskell, with the type \verb|a -> JS b|,
and Javascript functions, which we will notate using \verb|JSFunction|.
This relationship is more interesting than a Haskell synonym;
it reflects the reification options for capturing functions.
\end{itemize}

Highlights
\begin{itemize}
\item function / continuation
\item implementing wait \& fork
\end{itemize}

Up to this point, Sunroof is academically interesting, but in a real sense
we are writing Javascript using Haskell syntax, so why not just write
Javascript? There are three things we have bought by using our DSL.
\begin{itemize}
\item Typing -- We have a simple monomorphic type system that helps development of Sunroof code.
\item Haskell Abstractions --
\item Dynamic Generation -- 
\end{itemize}

 
\section{Conclusion}

We can see that Sunroof took the key idea of monad reification and
sucessfully created the \JS-monad to describe computations
in JavaScript. This work was mainly already done by Farmer and
Gill \cite{Farmer:12:WebDSLs} and has been streamlined during the 
further development of Sunroof. By adding the concept 
of \Src{JSFunction} and \Src{JSContinuation} there now is a connection between
functions as values in the JavaScript and the Sunroof language space 
(Figure \ref{fig:func-cont}). It is possible to go back and forth between 
both worlds. Combining both concepts, functions and the \JS-monad,
we were able to create a second implementation of the monad. This
time based on the direct translation of continuations from Haskell
to JavaScript. It enabled us to build a blocking threading model
on top of JavaScript that resembles the model already known from Haskell.
Based on this model and the provided abstraction over continuations
we can use synchronization primitives like \Src{forkJS} or \Src{yield}.
Higher level abstractions like \Src{JSMVar} and \Src{JSChan} are also
available. 


 
\section{Related Work}

There have been several attempts to translate Haskell to JavaScript.
Prominent ones are the compiler backends for 
UHC \cite{Stutterheim:12:ImprovingUHCJavaScriptBackend} and 
GHCJS \cite{project:ghcjs}. There are also projects like Fay \cite{project:fay} 
that compile subsets of Haskell to JavaScript or JMacro \cite{project:jmacro}
which use quasiquotation \cite{Mainland:07:QuasiquotingHaskell} to embed 
a custom-tailored language into Haskell code.

At the same time there are also projects like 
CoffeeScript \cite{project:coffeescript} or LiveScript \cite{project:livescript}
to build custom languages 
that are very similar to JavaScript but add convenient syntax and
support for missing features.

Our approach to cooperative concurrency through continuations in JavaScript has
has been used before 
\cite{Cooper:07:LinksWebProgrammingTiers,Predescu:02:CocoonContinuationBasedControlFlow}.
To our knowledge, creating a direct connection
between Haskell and JavaScript continuations has not been 
attempted before.

Deep embeddings of monads based on data structures have been used before
in Unimo \cite{Lin:06:Unimo} and Operational \cite{Apfelmus:10:Operational,Hackage:10:Operational}. 
The specific approach Sunroof takes 
by using GADTs has been discussed by 
Sculthorpe et al. \cite{Sculthorpe:13:ConstrainedMonads} 
in detail.

The Sunroof server does not have the aim to provide a full-featured 
web framework, as HAppS, Snap or Yesod do. It only provides 
the infrastructure to communicate with the currently calling website
through the Kansas comet \cite{project:kansas-comet} 
push mechanism \cite{pattern:push}. Although all of the
frameworks mentioned above would be able to implement this technique,
to our knowledge, none of them has yet.

To our knowledge, Sunroof is the only library that supports 
generation of JavaScript inside of Haskell using pure Haskell
in a type-safe manner. All other approaches discussed above
either require a separate compilation step or introduce new 
syntax inside of Haskell.

There is an effort to generalize Active \cite{project:active}, a library for animations, and
implement a backend based on Sunroof \cite{project:sunroof-active}.





 
\section{Acknowledgment}

Conal Elliott - For his support with Data.Boolean / Number abstraction






%
% ---- Bibliography ----
%


\bibliographystyle{splncs03}
\bibliography{sunroof}

\end{document}
