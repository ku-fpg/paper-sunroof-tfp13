% This is LLNCS.DEM the demonstration file of
% the LaTeX macro package from Springer-Verlag
% for Lecture Notes in Computer Science,
% version 2.4 for LaTeX2e as of 16. April 2010
%
\documentclass{llncs}
%
\usepackage{amsfonts}
\usepackage{comment}
\usepackage{graphicx}
\usepackage{caption}
\usepackage{subcaption}

\newcommand{\SunroofAnalog}[1]{#1\ensuremath{_\downarrow}}
\newcommand{\HaskellAnalog}[1]{#1\ensuremath{_\uparrow}}

\newcommand{\NOTE}[1]{{\Large\textbf{NOTE:}\ #1}}
\newcommand{\TODO}[1]{{\textbf{TODO:}\ #1}}
\newcommand{\Src}[1]{{\tt{#1}}}

\newcommand{\IO}{\Src{IO}}
\newcommand{\JS}{\Src{JS}}
\newcommand{\JSI}{\Src{JSI}}
\newcommand{\JSA}{\ensuremath{\Src{JS}_\Src{A}}}
\newcommand{\JSB}{\ensuremath{\Src{JS}_\Src{B}}}

\newcommand{\Figure}[3]{%
\FigureS{#1}{#2}{#3}{scale=0.55,clip=true,trim=0.45cm 0.45cm 0.45cm 0.45cm}
}

\newcommand{\FigureS}[4]{%
\begin{figure}[h]%
\vspace{-0.5cm}%
\begin{center}%
\includegraphics[#4]{#2}%
\vspace{-0.5cm}%
\end{center}%
\caption{#3}%
\label{#1}%
\vspace{-0.5cm}%
\end{figure}%
}

\begin{document}
%
\title{Sunroof: A Monadic DSL to Generate JavaScript}
%\subtitle{}
%
\titlerunning{Sunroof}  % abbreviated title (for running head)
%                                     also used for the TOC unless
%                                     \toctitle is used
%
\author{Jan Bracker\inst{1,2}  \and Andy Gill\inst{1}}
%
\authorrunning{Jan Bracker \and Andy Gill} % abbreviated author list (for running head)
%
%%%% list of authors for the TOC (use if author list has to be modified)
\tocauthor{Jan Bracker, Andy Gill}
%
\institute{%
ITTC / EECS \\
The University of Kansas\\
Lawrence, KS 66045\\
~\\
\and
Institut f{\"u}r Informatik\\
Christian-Albrechts-Universit{\"a}t\\
Kiel, Germany}

\maketitle

\begin{abstract}        
 
Sunroof is a Haskell-hosted Domain Specific Language (DSL) for generating JavaScript.
Sunroof is build on top of the JavaScript monad, which, like the Haskell \IO-monad, allows 
access to external resources, but specifically JavaScript
resources. As such, Sunroof is primarily a feature-rich 
foreign-function API to the browser's JavaScript engine, and all the browser-specific
functionality, including HTML-based rendering, event handling, and 
drawing to the HTML5 canvas. 

In this paper, we give the design and implementation of Sunroof.
Using monadic reification, we generate JavaScript from
the a deep embedding of the JavaScript monad.
The Sunroof DSL has the feel of native Haskell, with a simple
Haskell-based type schema to guide the Sunroof programmer.
Furthermore, because we are generating code,
we can offer Haskell-style concurrency patterns, such as MVars and Channels.
In combination with a web-services package such as Scotty,
the Sunroof compiler offers a robust platform to build interactive web applications,
giving the ability to interleave Haskell and JavaScript computations
with each other as needed.
\keywords{DSLs, JavaScript, Web Technologies, Cloud Computing}
\end{abstract}
%
 
\section{Introduction}

% Simon: Describe the problem
JavaScript is an imperative language with access to a wide range
of established and useful services, like graphical canvases and event
handling. JavaScript also provides features that are associated with 
functional languages, like first-class functions. 
We want to express JavaScript in Haskell, adding use
of Haskell's static typing, and gaining access to JavaScript services
in the Browser from Haskell.

% Simon: State your contributions
Sunroofs was developed to tackle this goal.
The small example in Figure \ref{fig:code-example} 
shows how Sunroof feels.
% Simon: An illustrative example
\begin{figure}[h]
\vspace{-0.5cm}
\centering
\begin{subfigure}{0.45\textwidth}%
\begin{Code}
 jsCode :: JS t ()
 jsCode = do
   name <- prompt "Your name?"
   alert ("Your name: " <> name)
\end{Code}%
\end{subfigure}%
\hfill%
\begin{subfigure}{0.45\textwidth}
\vspace{0.25cm}%
\begin{Code}
  
  
var v0 = prompt("Your name?"); 
alert("Your name: " + v0);
\end{Code}%
\end{subfigure}% 
%\vspace{-0.2cm}%
\caption{Sunroof program and the expected JavaScript on the left.}%
\label{fig:code-example}%
\vspace{-0.5cm}
\end{figure}
% Simon: Why is this useful?
The expected JavaScript output is shown on the right. But what makes 
this useful? We produced code that is shorter 
then the Sunroof code on the left and can easily be written by hand.

% Simon: List of contributions
Sunroof's approach has several advantages. 
It introduces a threading model and abstraction similar
to Haskell's. Like this programmers can reuse well known 
techniques from Haskell when writing JavaScript.
We also utilize the static type system to support us when 
writing code. The functions
in Figure \ref{fig:code-example} have the following signatures:
\begin{Code}
prompt :: JSString -> JSString -> JS t JSObject
alert  :: JSString -> JS t ()
\end{Code}
This improves the quality of written code.
Using a deep embedding gives opportunity 
for optimizations when producing the actual JavaScript.
At the same time Sunroof offers an interface that is 
close to actual JavaScript making it easy to use.
The interface is easily extendable through
a foreign function interface.
We model the imperative nature of JavaScript
using the transitional functional programming 
mechanism to be imperative, a monad~\cite{Moggi:91:ComputationMonads}.
% The scope of this paper is to show how Sunroof achieves these goals.
% It will cover the most important parts of Sunroof and how they are 
% implemented.
\Figure%
{fig:structure}%
{figures/sunroof-structure.pdf}%
{The structure of Sunroof.}

We will go cover each of the layers in Figure \ref{fig:structure}
throughout the paper:
\begin{itemize}
\item
The \JS-monad together with the underlying \JSI~primitives 
will be explained in Section \ref{sec:js-monad}. 
We will show how it is implemented and how we solved
the problem of constraining types involved in the monadic 
computations.
\item
Section \ref{sec:object-model} will discuss how we annotate 
JavaScript objects with types using wrappers 
and offer the possibility to add custom types later on.
\item
The special role of functions and continuations and
how we model them as first-class values in Haskell and JavaScript
will be covered in Section \ref{sec:functions-continuations}. .
\item
The two threading models offered by Sunroof are explained 
in Section \ref{sec:threading-models}.
\item
Section \ref{sec:ffi} introduces Sunroof's foreign function interface.
\item
Translation of Sunroof to JavaScript is handled in 
Section \ref{sec:compiler}. We will explain the 
compilation of selected language constructs. This is 
especially interesting in the light of our use of continuations
and their translation to JavaScript.
\item
The ability to interleave Haskell and JavaScript computations as needed
through the Sunroof server will be highlighted in Section \ref{sec:server}.
\item
Section \ref{sec:extended-example} will cover a small application 
written in Sunroof to surveys how usable Sunroof is in the 
context of application development. 
%It shows how Sunroof can be used to utilize the 
%display capabilities of a browser, while still using Haskell when needed
%and where it is strong.
\end{itemize}

\begin{comment}
\subsubsection{RESOURCES - REMOVE WHEN FINISHED}

\begin{itemize}
\item Give an example of Sunroof 
\item Explain the example
\item Talk about the general structure of Sunroof
\item Relate each part of the structure to a specific section:
  \begin{itemize}
  \item \JSA\ /\ \JSB: \JS-monad in section \ref{sec:js-monad};
  together with \JSI\ and contiuations used by the \JS-monad.
  \item Type Wrappers / Expr: Discussed in section \ref{sec:object-model}
  \end{itemize}
\item Details about functions and continuations are given in section \ref{sec:functions-continuations}
\item 
Sunroof was first documented in our previous 
workshop paper~\cite{Farmer:12:WebDSLs},
where the possibility of monadic reification was observed.
In this paper, we raised an unresolved issue:

Threading Model: Built on top of \JS-monad (section \ref{sec:threading-models})
\item Explain how the compiler works and what it does \ref{sec:compiler}
\item In section \ref{sec:server} we will look at the sunroof server 
\item Extended example in section \ref{sec:extended-example}

\end{itemize}

We want to use the Javascript API. Fast interpreters,
running in a browser advantage, 

We do not want to program in Javascript.

\end{comment}




\begin{comment}
\section{Example Sunroof Program}
\label{sec:simple-example}

A simple drawing program, that we build up step by step.
Perhaps bouncing ball.
Perhaps drawing line.
\end{comment}

 
\section{The JavaScript Monad}
\label{sec:js-monad}

JavaScript is an imperative language with access to a wide range
of established and useful services, like graphical canvases and event
handling. JavaScript as a language also provides features that are
traditionally associated with functional languages, like first-class 
functions. We want to express JavaScript in Haskell, adding use
of Haskell's static typing, and gaining access to JavaScript services.
And we do so using the transitional functional programming 
mechanism for being imperative, a monad~\cite{Moggi:91:ComputationMonads}.

The \JS-monad is the monad of JavaScript effects, as is an almost
exact analog for the Haskell \IO-monad, except there
is an extra phantom argument~\cite{Leijen:99:Phantom} that we will return
to shortly.  Figure~\ref{fig:code-example} gives a first example
of the \JS-monad in use with the generated JavaScript.

Inside this simple example is a challenging problem -- where does
\Src{v0} come from? The bind inside the monadic \Src{do} is
unconstrained:
\begin{verbatim}
(>>=) :: JS t a -> (a -> JS t b) -> JS t b
\end{verbatim}
What we want is:
\begin{verbatim}
(>>=) :: (Sunroof a) => JS t a -> (a -> JS t b) -> JS t b
\end{verbatim}
Where \Src{Sunroof} constrains the bind to
arguments for which we can generate a JavaScript variable.
Counterintuitively, 
it turns out that a specific form of normalization allows 
the ``\Src{a}'' type to be constrained and the bind to 
be an instance of the standard monad class~\cite{Sculthorpe:13:ConstrainedMonads}.
Through this keyhole of {\em monadic reification\/},
the entire Sunroof language is realized.

\subsubsection{RESOURCES - REMOVE WHEN FINISHED}

We use monad reification.
\begin{itemize}
\item binding in Haskell becomes binding in reified language.
\item Feels like a native monad, cf STM.
\item Can build abstractions on top of this.
\end{itemize}






 
\section{JavaScript Object Model}
\label{sec:object-model}

\begin{comment}
JavaScript is object based. It provides various objects,
including numbers, booleans, maps, and others. We
provide in Sunroof about a dozen common object,
including \Src{JSObject} (the generic object type), \Src{JSNumber}
(floating point numbers), \Src{JSCanvas} (HTML5 canvas type) 
and others. A simple
casting function is provided when the type-system
needs to be overwritten. Along with each of these types,
we provide typed methods.
\begin{verbatim}
 jsDrawBox :: JSObject -> JS t ()
 jsDrawBox document = do
     foo <- document # getElementById("foo")
     cxt <- foo # getContext("2d")     cxt # drawRect (0,0,100,100)
\end{verbatim}
Here, \Src{\#} is a reverse apply, so the types
of the function in the above example are
\begin{verbatim}
(#) :: o -> (o -> JS t a) -> JS t a
getElementById :: JSString -> JSObject -> JS t JSContext
getContext :: JSString -> JSContext -> JS t JSCanvas
drawRect :: (JSNumber,JSNumber,JSNumber,JSNumber) -> JSCanvas -> JS t ()
\end{verbatim}        
From experience, even though we are targeting
an untyped language, the type system gets in the
way less than we expected.
\end{comment}

As JavaScript is not statically typed, one goal
of Sunroof is to use Haskell's type system to
increase the correctness of the written JavaScript.
At the same time we have the problem that we can not characterize 
all different types of objects in JavaScript, since 
users can write their own objects. This means our 
system to type JavaScript needs to be extensible.

The approach from Svenningsson \cite{Svenningsson:12:CombiningEmbedding}
gives us this ability. We provide a basic \Src{Expr}ession 
language that has no associated type information to construct 
JavaScript expressions.
\begin{verbatim}
data E expr 
  = Lit String -- Precompiled (atomic) JavaScript literal
  | Var Id     -- Variable
  | Apply expr [expr]    -- Function application
  ...
data ExprE = ExprE (E ExprE)
type Expr  = E ExprE
\end{verbatim}
This core expression type is then wrapped to represent a more specific 
type. Each of these wrappers implements the \Src{Sunroof} type 
class.
\begin{verbatim}
class SunroofArgument a => Sunroof a where
  box :: Expr -> a
  unbox :: a -> Expr
  ...
\end{verbatim}
It marks that these types represent possible values in JavaScript.
Table \ref{tab:sunroof-types} shows some of the most important 
types in Sunroof. Some types like \Src{JSFunction} also involve 
phantom types to give more type safety (TODO: cite phantom types).
\begin{table}
\begin{center}
\begin{tabular}{r@{\quad}l@{\quad}l@{\quad}c}
\hline\rule{0pt}{12pt}%
  Constraint
  & Sunroof Type $\tau$
  & Haskell Analog \HaskellAnalog{$\tau$}
  & \Src{js} \\ \hline\rule{0pt}{12pt}%
  
  & \Src{()}       & \Src{()}     & $\checkmark$ \\
  & \Src{JSBool}   & \Src{Bool}   & $\checkmark$ \\
  & \Src{JSNumber} & \Src{Double} & $\checkmark$ \\
  & \Src{JSString} & \Src{String} & $\checkmark$ \\
  
  \Src{Sunroof $\alpha$}
  & \Src{JSArray $\alpha$} 
  & \Src{[$\HaskellAnalog{\alpha}$]}
  & \\
  
  \Src{SunroofKey $\alpha$}
  & \Src{JSMap $\alpha$ $\beta$}
  & \Src{Map $\HaskellAnalog{\alpha}$ $\HaskellAnalog{\beta}$}
  & \\
  \Src{Sunroof $\beta$} \\
  
  \Src{SunroofArgument $\alpha$}
  & \Src{JSFunction $\alpha$ $\beta$ }
  & \Src{$\HaskellAnalog{\alpha}$ $\rightarrow$ JS$_\Src{A}$ $\HaskellAnalog{\beta}$} 
  & $\checkmark$ \\
  \Src{Sunroof $\beta$} \\
  
  \Src{SunroofArgument $\alpha$}
  & \Src{JSMVar $\alpha$}
  & \Src{MVar $\HaskellAnalog{\alpha}$}
  & \\
  
  \Src{SunroofArgument $\alpha$}
  & \Src{JSChan $\alpha$}
  & \Src{Chan $\HaskellAnalog{\alpha}$}
  & \\[2pt]
\hline
\end{tabular}
\end{center}
\caption{Sunroof types and their Haskell pendant.}
\label{tab:sunroof-types}
\end{table} 
The table also shows that most of basic Haskell types have counterparts in
Sunroof. To convert these core Haskell values into their Sunroof 
counterparts we provide the \Src{SunroofValue} type class.
\begin{verbatim}
class SunroofValue a where
  type ValueOf a :: *
  js :: (Sunroof (ValueOf a)) => a -> ValueOf a
\end{verbatim}
It provides the corresponding Sunroof value through the type function
\Src{ValueOf} (TODO: Cite type functions) and offers the function 
\Src{js} to convert to that type. By design \Src{SunroofValue} does
only provide instances for values that can be converted in a pure
manner. Certain primitive values in JavaScript are referential 
transparent according to \Src{==} while others, like general objects,
are not. As an example, if you call \Src{new Object()} twice you get the 
same empty object, but when compared by \Src{==} they are different. They
are not identical, because in this case reference equality is checked
instead of value equality. We call this observable allocation and handle 
it as a side-effect which should only occur in the \JS-monad.

Sunroof also offers the ability to work with record like data structures to 
keep values that belong together in one place.
For this purpose Sunroof offers the \Src{JSTuple} type class.
\begin{verbatim}
class Sunroof o => JSTuple o where
  type Internals o
  match :: (Sunroof o) => o -> Internals o
  tuple :: Internals o -> JS t o
\end{verbatim}
If you have a record of Sunroof data that you want to
encode as a JavaScript object you can provide this ability 
by implementing \Src{JSTuple}. The \Src{Internals} type function
delivers your record type. Encoding that record as a \Src{Sunroof}
value is done through \Src{tuple}. As objects may be used to 
encode records we have to be inside the \JS-monad to do this,
because of the observable allocation discussed earlier.
Decomposing the encoding is done with \Src{match}. Because the
\Src{JSTuple} idiom is meant to represent immutable data structures, this 
can be done in a pure manner although there are ways to mutate 
values referenced by the decomposed record, since they are only 
references to the actual data in JavaScript.

\subsubsection{RESOURCES - REMOVE WHEN FINISHED}

General Objects Types and Expressions
\begin{itemize}
\item We want to use Haskell's type system to give use type safety in JavaScript/Sunroof.
\item JavaScript is untyped / dynamically typed (TODO: what is it actually? cite/reference?)
\item We can not represent all types in JavaScript
\item We have to give a way to add types later on
\item Use approach from \cite{Svenningsson:12:CombiningEmbedding}
\item There is a core expression language that is used to describe how and expression is built
\item We wrap that core type into wrappers
\item Introduce \Src{Sunroof} to mark wrappers
\item Wrappers have specific functionality for certain type
\item Wrappers can have phantom type to give more type safety
\item Example \Src{JSString} and \Src{JSArray a}
\item Maybe go through example type like \Src{JSString} or \Src{()}.
\item Unavoidable need to cast types some times: \Src{cast}
\item Adding a new type can be done mechanically: Template Haskell
\end{itemize}
Haskell to Sunroof conversion
\begin{itemize}
\item Many haskell types have equivalent or similar types in JavaScript
\item So we offer \Src{SunroofValue}; It connects a Haskell value with 
its JavaScript companion through a type function (cite type functions)
and gives a function to convert: \Src{js}
\item Introduce \Src{SunroofValue}
\item Conversion is pure for most types since atomic values 
are created when converting.
\item Exception \Src{JSFunction}, same reason as for \Src{JSTuple} and \Src{tuple}
\item Show table with types as an overview. Small comment paragraph about table.
\item \Src{JSArray} not possible because of conflicts with \Src{String}
instance (minor point); also issue of mutability of arrays; observable 
allocation issue like with \Src{JSTuple}.
\item Forward reference to section \ref{sec:threading-models} for \Src{JSMVar}
and \Src{JSChan}
\end{itemize}
JSTuple and records in JS
\begin{itemize}
\item Design decision: We do not want to introduce internal
structure for types. (Tuples aren't JS types)
\item \Src{JSTuple} exists to introduce types with custom structure
\item Introduce \Src{JSTuple}
\item manages composition and decomposition of custom types.
\item Meant to codify immutable records in JavaScript
\item They can only be changed
by decomposing them into Haskell and recreating a new structure
with other values.
\item Useful to manage more complex data structures in JavaScript.
\item Decomposition is pure operation: \Src{match}
\item Justification lies in the fact that they are meant to be 
immutable. Although there are possibilities to break this.
\item Composition is monadic effect: \Src{tuple}
\item It captures observable allocation and ensures that a reference to the
allocated value is created and used afterwards.
\item If it was not monadic the let binding would be like a macro
that reallocated the same object at each point. This 
is not desired behavior in most use-cases
\end{itemize}









 
\section{JavaScript Functions and Continuations}
\label{sec:functions-continuations}

Given these threading models, we can realize both as 
JavaScript objects. 
\begin{verbatim}
function     :: (...) => (a -> JS A b)  -> JS t (JSFunction a b)
continuation :: (...) => (a -> JS B ()) -> JS t (JSContinuation a)
\end{verbatim}
\Src{JSFunction a b} and \Src{JSContinuation a},
like \Src{JSObject} and others, are realized as objects
in JavaScript. Thus they can be passed as arguments, returned
from functions and stored in mutable structures.

\Figure%
{fig:func-cont}%
{figures/sunroof-func-cont.pdf}%
{How functions and continuations relate between the Haskell- and JS-domain.}%

Functions and continuations can be called using \Src{apply}
and \Src{goto} respectively. \Src{apply} calls the function
and returns, \Src{goto} calls the continuation, but never
returns.
\begin{verbatim}
apply :: (...) => JSFunction args ret -> args -> JS t ret
goto  :: (...) => JSContinuation args -> args -> JS t a
\end{verbatim}

\begin{table}
\caption{Reifying and calling JavaScript functions}
\begin{center}
\begin{tabular}{r@{\quad}c@{\quad}c@{\quad}c@{\quad}c}
\hline\rule{0pt}{12pt}%

                & Monadic Function      & Reification   & Object in     & Invocation\\
                & in Haskell            & Function      & Javascript    & Function\\
\hline\rule{0pt}{12pt}%
  Functions
  & $\alpha\rightarrow\ $\Src{JS}$_\Src{A}~\beta$
  & \Src{function}
  & \Src{JSFunction}~$\alpha~\beta$
  & \Src{apply} \\
  Continuations
  & $\alpha\rightarrow\ $\Src{JS}$_\Src{B}~\Src{()}$
  & \Src{continuation}
  & \Src{JSContinuation}~$\alpha$
  & \Src{goto}\\
\hline
\end{tabular}
\end{center}
\end{table} 

\subsubsection{RESOURCES - REMOVE WHEN FINISHED}

\begin{itemize}
\item Functions are first class members in Haskell and JavaScript
\item But in JavaScript they are limited (no partial application)
\item That leads to type for functions \Src{JSFunction a b}
\item Functions can take more then one argument; \Src{Sunroof} insufficient: \Src{SunroofArgument}
\item \Src{SunroofArgument} is prerequisite for \Src{Sunroof},
because all JS objects are parameters but pairs of them are not 
JS objects.
\item Introduce \Src{SunroofArgument}.
\item So function in JS are not curried (partial application is questionable in JS)
\item Create a function with \Src{function} combinator.
\item Function creation is a monadic effect for the same reason 
a record creation (\Src{JSTuple}). It represents observable
allocation and we usually want one reference to a function 
instead of coping its definition everywhere.
\item Calling a function or method can have a side-effect
\item That is why the operator \Src{\$\$} for function application is monadic to.
\item All these operations do is to produce \Src{JS\_Function} and \Src{JS\_Invoke}
instructions.
\item Continuations are basically the same as functions at this level.
\item Created with \Src{continuation} and called with \Src{goto}.
\item Implementation of \Src{goto} is interesting. Ignores 
the current continuation in the \JS-monad and continues with the given one.
\item TODO: Further detail on continuations; Here or somewhere else?
\end{itemize}








 
\section{JavaScript Threading Models}
\label{sec:threading-models}

Sunroof was first documented in our previous 
workshop paper~\cite{Farmer:12:WebDSLs},
where the possibility of monadic reification was observed.
In this paper, we raised an unresolved issue: do you
generate atomic JavaScript code, and keep the callback
centric model of computation, or generate JavaScript
using CPS, and allow for blocking primitives,
like Haskell \Src{MVar}s. The latter, though more powerful, 
precluded using the compiler to generate
code that can be cleanly called from native JavaScript.
Both choices had poor consequences.

So, rather than pick one, we decided to explicitly support both,
and make both first class threading strategies in our compiler.
In terms of user-interface, we parameter the \JS-monad
with a phantom type that represents the threading model
to compile with, with \Src{A} for \Src{A}tomic threads,
and \Src{B} for \Src{B}locking (cooperative concurrency) threads. 
Atomic threads are classical JavaScript threads, and
are never interrupted; while blocking threads can
support suspending operations. By using phantom
types, we can express the necessary
restrictions on specific combinators, as well
as provide combinators to allow both types of
threads to cooperate successfully.

\subsubsection{RESOURCES - REMOVE WHEN FINISHED}




 
\section{The Sunroof Compiler}
\label{sec:compiler}

Given the language, and monadic-reification, how do we compile this language?
Figure \ref{fig:structure} shows how Sunroof is structured.

\Figure%
{fig:structure}%
{figures/sunroof-structure.pdf}%
{The structure of Sunroof.}

Through the \JS-monad we produce a \Src{Program (JSI t) ()} instance. We 
translate such a program into a list of statements (\Src{Stmt}) by matching over 
the \JSI constructors.
\begin{verbatim}
data Stmt 
  = AssignStmt Rhs Expr       -- Assignment
  | DeleteStmt Expr           -- Delete reference
  | ExprStmt Expr             -- Expression as statement
  | ReturnStmt Expr           -- Return statement
  | IfStmt Expr [Stmt] [Stmt] -- If-Then-Else statement
  | WhileStmt Expr [Stmt]     -- While loop
  | CommentStmt String        -- Comment
\end{verbatim}
The constructors of \Src{Stmt} are straight forward and
directly represent the different statements one can write
in JavaScript.

To get a feeling of what is happening here, we will have a closer
look to how two of the \JSI instructions are compiled. First
lets have a closer look how branches are compiled.
\begin{verbatim}
compile :: Program (JSI t) () -> CompM [Stmt]
compile = eval . view
  where
    eval :: ProgramView (JSI t) () -> CompM [Stmt]
    eval (JS_Branch b c1 c2 :>>= k) = 
      case evalStyle (ThreadProxy :: ThreadProxy t) of
        A -> do
          (src0, res0) <- compileExpr (unbox b)
          res :: a <- jsValue
          let bindResults :: a -> JS t ()
              bindResults res' =
                sequence_ [ single $ JS_Assign_ v (box $ e :: JSObject)
                          | (Var v, e) <- jsArgs res `zip` jsArgs res'
                          ]
          src1 <- compile $ extractProgramJS bindResults c1
          src2 <- compile $ extractProgramJS bindResults c2
          rest <- compile (k res)
          return (src0 ++ [ IfStmt res0 src1 src2 ] ++ rest)
        B -> do
          fn_e <- compileContinuation 
            (\ a -> blockableJS $ JS $ \ k2 -> k a >>= k2)
          fn           <- newVar
          (src0, res0) <- compileExpr (unbox b)
          src1 <- compile $ extractProgramJS (apply (var fn)) c1
          src2 <- compile $ extractProgramJS (apply (var fn)) c2
          return ( [mkVarStmt fn fn_e] ++ src0 ++ [ IfStmt res0 src1 src2 ])
\end{verbatim}
\TODO This brings up more questions then it answers.


\begin{comment}
Given the language, and monadic-reification, how do we compile this language?
Figure \ref{fig:structure} shows how Sunroof is structured.
On the lowest level we provide an untyped expression language \Src{Expr}
that describes JavaScript expressions. 
To provide type safety when using Sunroof we create
wrappers for each type we want to represent, e.g. \Src{JSNumber} or \Src{JSString}.
The \Src{Sunroof} type class provides an 
interface to create wrapped and unwrapped
instances of our expressions. Based on the wrappers we can provide 
operations specific to a certain type, e.g. a \Src{Num} instance
for \Src{JSNumber} or a \Src{Monoid} instance for \Src{JSString}.



This technique enables us to utilize  Haskells type system when writing JavaScript
and offers an easy way to add new types when needed~\cite{Svenningsson:12:CombiningEmbedding}.
By using phantom types we can also provide more advanced types,
like \Src{JSArray a}.

The next layer provides JavaScript instructions through the type \JSI.
They represents abstract statements. While expressions and values
represented with type wrappers are assumed to be free of side-effects,
the instructions model side-effects in JavaScript. Examples for Instructions
are assignment of an attribute or the application of a function.
\begin{verbatim}
data JSI :: T -> * -> * where
  JS_Assign :: (...) => JSSelector a -> a -> JSObject -> JSI t ()
  JS_Invoke :: (...) => a -> JSFunction a r -> JSI t r
  JS_Branch :: (...) => bool -> JS t a -> JS t a  -> JSI t a
  ...
\end{verbatim}
The \JS-monad with its combinators builds a sequence of 
\JSI{}nstructions through the operational
package~\cite{Hackage:10:Operational,Apfelmus:10:Operational}.
All constraints required on instructions are introduced by their 
constructors.
As mentioned earlier the \JS-monad comes in two threading flavors, 
parameterized using a phantom type.
Internally the \JS-monad is implemented using CPS on the 
underlying \Src{Program} type from Operational. 
\begin{verbatim}
data JS :: T -> * -> * where
  JS   :: ((a -> Program (JSI t) ()) -> Program (JSI t) ()) 
       -> JS t a
  ...
\end{verbatim}
For atomic
computations we just produce a list of instructions from the continuation. 
When translating possibly blocking code we directly translate that continuation
into JavaScript functions. This gives us the ability to handle 
computations as values in JavaScript and store them if needed.

Blocking operations just store the rest of their computation in a queue.
When the event to unblock occurs the pending computation is registered 
as a callback that will be executed as soon as the current computation
is done.

On top of our \JS-monad we provide ways of specifying (typed)
interfaces to JavaScript capabilities, a Foreign Function Interface.
\begin{verbatim}
alert :: JSString -> JS t ()
alert = fun "alert"

getElementById :: JSString -> JSObject -> JS t JSCanvas
getElementById = invoke "getElementById"
\end{verbatim}
Notice, that calling a JavaScript function or method is done by giving its
name to one of the provided combinators. Types can be specialized using 
a Haskell type annotation. A flexible and easy to use approach.
\end{comment}

\begin{comment}
\subsubsection{RESOURCES - REMOVE WHEN FINISHED}

Compiling
\begin{itemize}
\item Short introduction to the compiler interface (signature).
\item Core work done by translating \Src{Program (JSI t) ()}
into a list of \Src{Stmt}s.
\item Introduce statement type, give a short description of 
each constructor (just in the comments)
\item Basic idea: Each \JSI~nstruction is translated 
into a sequence of statements and these are then
concatenated together.
\item Look at interesting parts
\item TODO: Which parts are interesting? Most parts are too technical
\item Show how a branch is compiled (TODO: \Src{extractProgramJS} lets things look messy)
\item Talk about how \Src{JS\_Fix} is compiled.
\item TODO: Understand why fix works.
\item Difference between compilation of a function and a continuation
\item Function \Src{\textbackslash a -> JS \$ \textbackslash k -> singleton (JS\_Return a) >>= k}
\item TODO: Why \Src{>>= k}?
\item Return the result of the current continuation
\item Continuation \Src{\textbackslash \_ -> JS \$ \textbackslash k -> k ()}
\item Just execute it instead of passing it on further. No return value!
\item We use expression sharing through observable sharing (reference Andys paper).
\end{itemize}

\begin{itemize}
\item If we transliterate, we have straight line code, can not pause.
    (Wait for Mvar, for example)
\item If we CPS translate, we can use continuations to capture the
   notion of a paused thread. Works well.
   Problems:
  \begin{itemize}
   \item Can not translate functions, how do they get there return value
   \item (Assumes straight line code.)
   \item Also, the code becomes unreadable to anyone except a die-hard 
       compiler freak.
  \end{itemize}
\end{itemize}
 
Choice:
\begin{itemize}
\item We support both!
\item Phantom argument to JS
\item A = Atomic, B = Blockable.
\end{itemize}
\end{comment}






 
\section{The Sunroof Server}
\label{sec:server}

The Sunroof compiler can compile JavaScript that can be used
stand-alone. But Sunroof really comes
into its own when used with the Sunroof server.

It provides infrastructure to send arbitrary pieces 
of JavaScript to a calling website for execution. 
Like this it is possible to interleave Haskell and JavaScript 
computations with each other as needed. The three major functions
provided are \Src{sunroofServer}, \Src{syncJS} and \Src{asyncJS}.
\begin{verbatim}
type SunroofApp = SunroofEngine -> IO ()
sunroofServer :: SunroofServerOptions -> SunroofApp -> IO ()

syncJS  :: SunroofResult a 
        => SunroofEngine -> JS t a -> IO (ResultOf a)
asyncJS :: SunroofEngine -> JS t () -> IO ()
\end{verbatim}
\Src{sunroofServer} starts the server.
It will the given callback function (\Src{SunroofApp}) 
for each request to the server.
\Src{syncJS} and \Src{asyncJS} allow the server
to remotely execute monadic JavaScript from inside the 
requesting website.
\Src{asyncJS} executes JavaScript asynchronously without 
waiting for a return value. In contract to that 
\Src{syncJS} waits until the execution is complete and
then sends the result back to the server and converts if 
into a Haskell value that can be processed further. 
Values that can be returned from a synchronous execution 
have to implement the \Src{SunroofResult} class. It contains a type-function
\Src{ResultOf} that maps the 
Sunroof type to a corresponding Haskell type and also 
implements a function to convert incoming data to that 
type.

Besides these basic functions it also provides the abstract 
concepts of \Src{Downlink}s and \Src{Uplink}s. They are 
channels for sending data either from the server or to
the server direction.

\begin{comment}
\subsubsection{RESOURCES - REMOVE WHEN FINISHED}

\begin{itemize}
\item Introduce major idea of a server that can communicate
with the calling website and send arbitrary pieces
of JavaScript to execute on demand.
\item TODO: Go into detail about the kansas-comet stuff 
or just give a reference to the original paper?
\item Introduce the major function for running and communicating:
\Src{sunroofServer}, \Src{syncJS} and \Src{asyncJS}
\item Describe what they do.
\item Specifically look at \Src{syncJS} and \Src{SunroofResult}
used to return actual Haskell values for the sent JavaScript return value.
\item Introduce abstractions \Src{Downlink} and \Src{Uplink} that
can be used to communicate in either direction.
\item TODO: This is only a brief introduction?
\end{itemize}
\end{comment}







 
\section{Case Study - A Small Calculator}
\label{sec:extended-example}

To see how Sunroof works in practice, we will look into the 
experience we gathered when writing a small calculator
for arithmetic expressions (\FigRef{fig:example-application}). 
We use Sunroof to display our interface
and the results of our computation. Haskell will be used to parse the 
arithmetic expressions and calculate the result. The Sunroof server 
will be used to implement this JavaScript/Haskell hybrid.

\FigureS%
{fig:example-application}%
{figures/example-application.png}%
{The example application running on the Sunroof server.}%
{scale=0.6}

The classical approach to develop an application like this would have 
been to write a server that provides a RESTful interface and replies 
through a JSON data structure. 
The client side of that application would have been written in JavaScript
directly.
This can be seen in \FigRef{fig:example-structure}.

\Figure%
{fig:example-structure}%
{figures/example-structure.pdf}%
{Classical structure and Sunroof structure of a web application.}

How does Sunroof improve or change this classical structure?
First of all, in Sunroof you write the client-side code together with
your server application within Haskell. In our example, all code 
for the server and client is in Haskell. The control logic 
for the client side is provided through the server.
This leads to a tight coupling between both sides. 
This also shows how Sunroof blurs the border between the server 
and client side. You are not restricted by an interface or language 
barrier. If you need the client to do something, you can just 
send arbitrary Sunroof code to execute in the client.

\TabRef{tab:example-statistics} contains a few statistics 
about the size of the code in each part of the client.

The client-server response loop shuffles new input to the server 
and executes the response in the client.
%
Data conversion is needed, because pure Haskell data types
cannot be handled in Sunroof and vice versa. There still
exists a language barrier between JavaScript and Haskell. 
Code to convert between two essentially equal data structures on 
each side must be written, as well as representations of Haskell 
structures in Sunroof. However, there is great potential in automatically 
generating this code using techniques such as template Haskell
\cite{Sheard:02:TemplateMetaProgrammingHaskell}.

The code for displaying the results is basically a 
transliteration of the JavaScript that you would write for this 
purpose.
The transliteration used here is not very appealing. 
In the future, this code can be generated through higher-level 
libraries. Sunroof is intended to deliver a foundation for
this purpose.

The rest of our code to parse the arithmetic expression and calculate 
results is classical Haskell code. 

\pagebreak

\begin{table}[t]
\begin{center}
\begin{tabular}{l@{\quad}r@{\quad}r}
\hline\rule{0pt}{12pt}%
Part of Application & Lines of Code & Percentage \\[2pt]
\hline\rule{0pt}{12pt}%
Response loop & 25 & 6.5\% \\[2pt]
Data conversion & 85 & 22.0\% \\[2pt]
Rendering & 190 & 49.5\% \\[2pt]
Parsing and interpretation & 85 & 22.0\% \\[2pt]
\hline
\end{tabular}
\end{center}
\caption{Lines of code needed for the example.}
\label{tab:example-statistics}
\vspace{-0.5cm}
\end{table} 


\section{Background - RESOURCE COLLECTION}
\label{sec:background}

We chose to implement Sunroof as deep embedded DSL for different
reasons. Porting a compiler, like GHC, to Javascript would have
cost a lot of effort. There is no clear way to port the existing
API and there are many language features that would be difficult to
translate. Providing a foreign function interface would 
introduce the cost of translating back- and forward. 
\TODO{Not even sure how this would work or where cost comes in.}
Another possibility would be to develop a custom language. 
Besides that people would need to learn a third language for this 
approach, it would involve effort similar to porting a compiler.
As a deep embedded DSL people only have to know Haskell to learn
the new language. The cost of developing the DSL remains light, 
because we can build upon a mature language and its features.

Choices when compiling to Javascript
\begin{itemize}
\item Port a compiler to Javascript - issues, mismatching API, other efforts.
\item Provide an FFI - back and forward cost.
\item Compile a custom, cut down FP language. - example
\item DSL, restrictive, syntactically, how do we handle binding?
\end{itemize}

\subsection{Basic Idea}

As we have seen in the previous example application Sunroof uses a 
monad to model computations that can be translated to Javascript.
A binding operation is thought to be an assignment to 
a fresh variable in the reified language.



Design objectives and contributions.
\begin{itemize}
\item 
Specifically, we want to explore how similar to a native Haskell monad,
like the IO or STM monads, can we make the JS monad feel to Sunroof users.
\item We want to reflect objects. (easy)

\item We want also investigate the relationship between Haskell functions
and Javascript functions. In particular, there is an isomorphism
between monadic functions in Haskell, with the type \verb|a -> JS b|,
and Javascript functions, which we will notate using \verb|JSFunction|.
This relationship is more interesting than a Haskell synonym;
it reflects the reification options for capturing functions.
\end{itemize}

Highlights
\begin{itemize}
\item function / continuation
\item implementing wait \& fork
\end{itemize}

Up to this point, Sunroof is academically interesting, but in a real sense
we are writing Javascript using Haskell syntax, so why not just write
Javascript? There are three things we have bought by using our DSL.
\begin{itemize}
\item Typing -- We have a simple monomorphic type system that helps development of Sunroof code.
\item Haskell Abstractions --
\item Dynamic Generation -- 
\end{itemize}

 
\section{Conclusion}

Sunroof took the key idea of monad reification and
successfully created the \JS-monad to describe computations
in JavaScript. This work was started by Farmer and
Gill \cite{Farmer:12:WebDSLs}, with the observation
of the possibility to reify a JS monad.
This paper documents the work since this initial implementation.
By adding the concept 
of \Src{JSFunction} and \Src{JSContinuation}, there now is a 
stronger connection between
functions in the JavaScript and the Sunroof language space 
(\FigRef{fig:func-cont}). It is possible to go back and forth between 
both worlds. Combining both concepts, functions and the \JS-monad,
we were able to create a second implementation of the monad, this
time based on the direct translation of continuations from Haskell
to JavaScript. It enabled us to build a blocking threading model
on top of JavaScript that resembles the model already known from Haskell.
Based on this model and the provided abstraction over continuations,
we can use primitives like \Src{forkJS} or \Src{yield}.
Higher-level abstractions like \Src{JSMVar} and \Src{JSChan} are also
available. 


 
\section{Related Work}

There have been several attempts to translate Haskell to JavaScript.
Prominent ones are the compiler backends for 
UHC \cite{Stutterheim:12:ImprovingUHCJavaScriptBackend} and 
GHCJS \cite{project:ghcjs}. There are also projects like Fay \cite{project:fay} 
that compile subsets of Haskell to JavaScript or JMacro \cite{project:jmacro}
which use quasiquotation \cite{Mainland:07:QuasiquotingHaskell} to embed 
a custom-tailored language into Haskell code.

At the same time there are also projects like 
CoffeeScript \cite{project:coffeescript} or LiveScript \cite{project:livescript}
to build custom languages 
that are very similar to JavaScript but add convenient syntax and
support for missing features.

Our approach to cooperative concurrency through continuations in JavaScript has
has been used before 
\cite{Cooper:07:LinksWebProgrammingTiers,Predescu:02:CocoonContinuationBasedControlFlow}.
To our knowledge, creating a direct connection
between Haskell and JavaScript continuations has not been 
attempted before.

Deep embeddings of monads based on data structures have been used before
in Unimo \cite{Lin:06:Unimo} and Operational \cite{Apfelmus:10:Operational,Hackage:10:Operational}. 
The specific approach Sunroof takes 
by using GADTs has been discussed by 
Sculthorpe et al. \cite{Sculthorpe:13:ConstrainedMonads} 
in detail.

The Sunroof server does not have the aim to provide a full-featured 
web framework, as HAppS, Snap or Yesod do. It only provides 
the infrastructure to communicate with the currently calling website
through the Kansas comet \cite{project:kansas-comet} 
push mechanism \cite{pattern:push}. Although all of the
frameworks mentioned above would be able to implement this technique,
to our knowledge, none of them has yet.

To our knowledge, Sunroof is the only library that supports 
generation of JavaScript inside of Haskell using cheap and cheerful Haskell
in a type-safe manner. All other approaches discussed above
either require a separate compilation step or introduce new 
syntax inside of Haskell.

There is an effort to generalize Active \cite{project:active}, a library for animations, and
implement a backend based on Sunroof \cite{project:sunroof-active}.





 
\section{Acknowledgment}

We want to thank Conal Elliott for his support in adapting 
the Boolean package \cite{project:boolean} and helping us to
extend it with support for deeply embedded numbers.






%
% ---- Bibliography ----
%


\bibliographystyle{splncs03}
\bibliography{sunroof}

\end{document}
