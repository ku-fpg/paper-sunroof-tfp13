% This is LLNCS.DEM the demonstration file of
% the LaTeX macro package from Springer-Verlag
% for Lecture Notes in Computer Science,
% version 2.4 for LaTeX2e as of 16. April 2010
%
\documentclass{llncs}
%
\usepackage{amsfonts}
\usepackage{comment}
\usepackage{graphicx}
\usepackage{caption}
\usepackage{subcaption}

\newcommand{\SunroofAnalog}[1]{#1\ensuremath{_\downarrow}}
\newcommand{\HaskellAnalog}[1]{#1\ensuremath{_\uparrow}}

\newcommand{\NOTE}[1]{{\Large\textbf{NOTE:}\ #1}}
\newcommand{\TODO}[1]{{\textbf{TODO:}\ #1}}
\newcommand{\Src}[1]{{\tt{#1}}}

\newcommand{\IO}{\Src{IO}}
\newcommand{\JS}{\Src{JS}}
\newcommand{\JSA}{\ensuremath{\Src{JS}_\Src{A}}}
\newcommand{\JSB}{\ensuremath{\Src{JS}_\Src{B}}}

\newcommand{\Figure}[3]{%
\FigureS{#1}{#2}{#3}{scale=0.55,clip=true,trim=0.45cm 0.45cm 0.45cm 0.45cm}
}

\newcommand{\FigureS}[4]{%
\begin{figure}[h]%
\vspace{-0.5cm}%
\begin{center}%
\includegraphics[#4]{#2}%
\vspace{-0.5cm}%
\end{center}%
\caption{#3}%
\label{#1}%
\vspace{-0.5cm}%
\end{figure}%
}


\begin{document}
%
\title{Sunroof: A JavaScript Monad Compiler}
%\subtitle{}
%
\titlerunning{Sunroof}  % abbreviated title (for running head)
%                                     also used for the TOC unless
%                                     \toctitle is used
%
\author{Jan Bracker\inst{1,2}  \and Andy Gill\inst{1}}
%
\authorrunning{Jan Bracker \and Andy Gill} % abbreviated author list (for running head)
%
%%%% list of authors for the TOC (use if author list has to be modified)
\tocauthor{Jan Bracker, Andy Gill}
%
\institute{%
ITTC / EECS \\
The University of Kansas\\
Lawrence, KS 66045\\
~\\
\and
Institut f{\"u}r Informatik\\
Christian-Albrechts-Universit{\"a}t\\
Kiel, Germany}


\maketitle

\begin{abstract}        
Sunroof is a Haskell-hosted Domain Specific Language (DSL) for generating JavaScript.
Sunroof is build on top of the \JS-monad, which, like the Haskell \IO-monad, allows 
access to external resources, but specifically JavaScript
resources. As such, Sunroof is primarily a feature-rich foreign
function API to the browser's JavaScript engine, and all the browser-specific
functionality, including HTML-based rendering, event handling, and 
drawing to the HTML5 canvas. 

In this paper, we give the design and implementation of Sunroof.
Using monadic reification, we reify a deep embedding of the \JS-monad,
and from this embedding we generate JavaScript.
The Sunroof DSL has the feel of native Haskell, with a simple
Haskell-based type schema to guide the Sunroof programmer.
Furthermore, because we are generating code,
we can offer Haskell-style concurrency patterns, like MVars and Channels.
In combination with a web services package like Scotty,
the Sunroof compiler offers a robust platform to build interactive web applications,
giving the ability to interleave Haskell and JavaScript computations
with each other as needed.
\keywords{DSLs, JavaScript, Web Technologies, Cloud Computing}
\end{abstract}
%
\section{Introduction}

Sunroofs was developed to support Haskell programmers 
when generating JavaScript. Lets look at the small example
in figure \ref{fig:code-example} to see how it achieves this goal.

\begin{figure}[h]
\vspace{-0.5cm}
\centering
\begin{subfigure}{0.45\textwidth}%
\begin{verbatim}
 jsCode :: JS t ()
 jsCode = do
   name <- prompt "Your name?"
   alert ("Your name: " <> name)
\end{verbatim}%
\end{subfigure}%
\hfill%
\begin{subfigure}{0.45\textwidth}
\vspace{0.25cm}%
\begin{verbatim}
  
  
var v0 = prompt("Your name?"); 
alert("Your name: " + v0);
\end{verbatim}%
\end{subfigure}% 
%\vspace{-0.2cm}%
\caption{Sunroof program and the expected JavaScript on the left.}%
\label{fig:code-example}%
\vspace{-0.5cm}
\end{figure}

We can see what code we expect the program to produce when 
translated to JavaScript on the right. But what makes 
this useful? The JavaScript we want to produce with the given
Haskell is shorter and can easily be written by hand.
So lets look at the advantages of this approach:
\begin{itemize}

\item
At the same time Sunroof offers an interface that is 
close to actual JavaScript making it easy to learn.
It utilizes the \Src{do}-notation (Section \ref{sec:js-monad}) and creates a 
simple foreign function interface to achieve this goal (SECTION?).

\item
Sunroof also introduces a threading model (Section \ref{sec:threading-models}) similar
to Haskell's. Like this programmers can reuse known abstractions
and have a known set of powerful tools.

\item 
It also introduces types that support us when 
writing code and reveal common mistakes. The functions
in figure \ref{fig:code-example} have toe following signature:
\begin{verbatim}
prompt :: JSString -> JSString -> JS t JSObject
alert  :: JSString -> JS t ()
\end{verbatim}
This can improve the quality of our code.
Section \ref{sec:object-model} will look into this in more detail.

\item 
Using a deep embedding like this gives opportunity 
for optimizations when producing the actual JavaScript,
e.g. we can introduce expression sharing.
The details of the translation into JavaScript and
technical background is explained in Section \ref{sec:background}.





\end{itemize}


\begin{comment}

An essential part of most modern web applications is Javascript.
It opens various possibilities to create browser based applications.
Examples for this development are Googlemail or Facebook.
These applications rely on Javascript heavily. What makes this 
development possible are the fast interpreters available in all
major browsers. Developing applications inside the browser has
the advantage of being independent from the underlying hard- and
software platform. The problem is shifted from the actual machine
to the browser. Also modern web technologies like HTML5 and CSS
give easy to use tools for developing user interfaces for the
mentioned application. They are standardized and improved over years
up to now. 

Yet, writing actual code in Javascript can be tedious and error prone 
due to its not-statically typed nature. Also the object-oriented
paradigm and the callback mechanism used instead of proper multi-threading 
can be a stumbling block for veteran Haskell programmers. Especially when
facing the lack of type safety.

To make the power of Javascript available to Haskell programmers 
we developed Sunroof. It enables Haskell programmers to write
Javascript using Haskell through a deep embedding. This 
approach has several advantages:

\begin{description}
\item[Type Safety:] Sunroof provides custom types for all major Javascript
types. Thus it gives type safety similar to what we are used to from 
normal Haskell programs when writing Javascript.
\item[Paradigms \& Patterns:] As a Haskell EDSL Sunroof follows functional programming 
paradigms and principals. It is possible to reuse common combinators and 
patterns in Sunroof.
\item[Similarity:] At the same time Sunroof offers a API for Javascript 
that allows to translate Javascript into Sunroof code easily.
\item[Optimization:] Due to Sunroofs deep embedded nature it is able
to apply optimizations like expression sharing when compiling to Javascript.
\end{description}

\NOTE{This might be a bit to enthusiastic.}

\end{comment}

We want to use the Javascript API. Fast interpreters,
running in a browser advantage, 

We do not want to program in Javascript.

We start with an overview of the language, then discuss
how the language is implemented, and discuss interesting
techniques used inside Sunroof. We close with a more
complete web application.

\section{Example Sunroof Program}

A simple drawing program, that we build up step by step.
Perhaps bouncing ball.
Perhaps drawing line.

\section{The JavaScript Monad}
\label{sec:js-monad}

JavaScript is an imperative language with access to a wide range
of established and useful services, like graphical canvases and event
handling. JavaScript as a language also provides features that are
traditionally associated with functional languages, like first-class 
functions. We want to express JavaScript in Haskell, adding use
of Haskell's static typing, and gaining access to JavaScript services.
And we do so using the transitional functional programming 
mechanism for being imperative, a monad~\cite{Moggi:91:ComputationMonads}.

The \JS-monad is the monad of JavaScript effects, as is an almost
exact analog for the Haskell \IO-monad, except there
is an extra phantom argument~\cite{Leijen:99:Phantom} that we will return
to shortly.  Figure~\ref{fig:code-example} gives a first example
of the \JS-monad in use with the generated JavaScript.

Inside this simple example is a challenging problem -- where does
\Src{v0} come from? The bind inside the monadic \Src{do} is
unconstrained:
\begin{verbatim}
(>>=) :: JS t a -> (a -> JS t b) -> JS t b
\end{verbatim}
What we want is:
\begin{verbatim}
(>>=) :: (Sunroof a) => JS t a -> (a -> JS t b) -> JS t b
\end{verbatim}
Where \Src{Sunroof} constrains the bind to
arguments for which we can generate a JavaScript variable.
Counterintuitively, 
it turns out that a specific form of normalization allows 
the ``\Src{a}'' type to be constrained and the bind to 
be an instance of the standard monad class~\cite{Sculthorpe:13:ConstrainedMonads}.
Through this keyhole of {\em monadic reification\/},
the entire Sunroof language is realized.
The operational package~\cite{Hackage:10:Operational,Apfelmus:10:Operational}
is used to normalize the \JS-monad. 

\section{JavaScript Threading Models}
\label{sec:threading-models}

Sunroof was first documented in our previous 
workshop paper~\cite{Farmer:12:WebDSLs},
where the possibility of monadic reification was observed.
In this paper, we raised an unresolved issue: do you
generate atomic JavaScript code, and keep the callback
centric model of computation, or generate JavaScript
using CPS, and allow for blocking primitives,
like Haskell \Src{MVar}s. The latter, though more powerful, 
precluded using the compiler to generate
code that can be cleanly called from native JavaScript.
Both choices had poor consequences.

So, rather than pick one, we decided to explicitly support both,
and make both first class threading strategies in our compiler.
In terms of user-interface, we parameter the \JS-monad
with a phantom type that represents the threading model
to compile with, with \Src{A} for \Src{A}tomic threads,
and \Src{B} for \Src{B}locking (cooperative concurrency) threads. 
Atomic threads are classical JavaScript threads, and
are never interrupted; while blocking threads can
support suspending operations. By using phantom
types, we can express the necessary
restrictions on specific combinators, as well
as provide combinators to allow both types of
threads to cooperate successfully.

\section{JavaScript Object Model}
\label{sec:object-model}

JavaScript is object based. It provides various objects,
including numbers, booleans, maps, and others. We
provide in Sunroof about a dozen common object,
including \Src{JSObject} (the generic object type), \Src{JSNumber}
(floating point numbers), \Src{JSCanvas} (HTML5 canvas type) 
and others. A simple
casting function is provided when the type-system
needs to be overwritten. Along with each of these types,
we provide typed methods.
\begin{verbatim}
 jsDrawBox :: JSObject -> JS t ()
 jsDrawBox document = do
     foo <- document # getElementById("foo")
     cxt <- foo # getContext("2d")     cxt # drawRect (0,0,100,100)
\end{verbatim}
Here, \Src{\#} is a reverse apply, so the types
of the function in the above example are
\begin{verbatim}
(#) :: o -> (o -> JS t a) -> JS t a
getElementById :: JSString -> JSObject -> JS t JSContext
getContext :: JSString -> JSContext -> JS t JSCanvas
drawRect :: (JSNumber,JSNumber,JSNumber,JSNumber) -> JSCanvas -> JS t ()
\end{verbatim}        
From experience, even though we are targeting
an untyped language, the type system gets in the
way less than we expected.

\section{JavaScript Functions and Continuations}
\label{sec:functions-continuations}

Given these threading models, we can realize both as 
JavaScript objects. 
\begin{verbatim}
function     :: (...) => (a -> JS A b)  -> JS t (JSFunction a b)
continuation :: (...) => (a -> JS B ()) -> JS t (JSContinuation a)
\end{verbatim}
\Src{JSFunction a b} and \Src{JSContinuation a},
like \Src{JSObject} and others, are realized as objects
in JavaScript. Thus they can be passed as arguments, returned
from functions and stored in mutable structures.

\Figure%
{fig:func-cont}%
{figures/sunroof-func-cont.pdf}%
{How functions and continuations relate between the Haskell- and JS-domain.}%

Functions and continuations can be called using \Src{apply}
and \Src{goto} respectively. \Src{apply} calls the function
and returns, \Src{goto} calls the continuation, but never
returns.
\begin{verbatim}
apply :: (...) => JSFunction args ret -> args -> JS t ret
goto  :: (...) => JSContinuation args -> args -> JS t a
\end{verbatim}

\section{The Sunroof Compiler}
\label{sec:compiler}

Given the language, and monadic-reification, how do we  compile this language? 
Figure \ref{fig:structure} shows how Sunroof is structured.
On the lowest level we provide an untyped expression language \Src{Expr}
that describes JavaScript expressions. 
To provide type safety when using Sunroof we create
wrappers for each type we want to represent, e.g. \Src{JSNumber} or \Src{JSString}.
The \Src{Sunroof} type class provides an 
interface to create wrapped and unwrapped
instances of our expressions. Based on the wrappers we can provide 
operations specific to a certain type, e.g. a \Src{Num} instance
for \Src{JSNumber} or a \Src{Monoid} instance for \Src{JSString}.

\Figure%
{fig:structure}%
{figures/sunroof-structure.pdf}%
{The structure of Sunroof.}%

This technique enables us to utilize  Haskells type system when writing JavaScript
and offers an easy way to add new types when needed~\cite{Svenningsson:12:CombiningEmbedding}.
By using phantom types we can also provide more advanced types,
like \Src{JSArray a}.

The next layer provides JavaScript instructions through the type \Src{JSI}.
They represents abstract statements. While expressions and values
represented with type wrappers are assumed to be free of side-effect,
the instructions model side-effects in JavaScript. Examples for Instructions
are assignment of an attribute or the application of a function.
\begin{verbatim}
data JSI :: T -> * -> * where
  JS_Assign :: (...) => JSSelector a -> a -> JSObject -> JSI t ()
  JS_Invoke :: (...) => a -> JSFunction a r -> JSI t r
  JS_Branch :: (...) => bool -> JS t a -> JS t a  -> JSI t a
  ...
\end{verbatim}
The \Src{JS} monad with its combinators builds a sequence of 
\Src{JSI}nstructions through the operational
package~\cite{Hackage:10:Operational,Apfelmus:10:Operational}.
As mentioned earlier the \Src{JS}-monad comes in two threading flavors, 
parameterized using a phantom type.
Internally the \Src{JS}-monad is implemented using CPS. For atomic
computations we just produce a list of instructions from the continuation. 
When translating possibly blocking code we directly translate that continuation
into JavaScript functions. This gives us the power to handle 
computations as values in the JavaScript and store them if needed.

Blocking operations just store the rest of their computation in a queue.
When the event to unblock occurs the pending computation is registered 
as a callback that will be executed as soon as the current computation
is done.

On top of our \Src{JS}-monad we provide ways of specifying (typed)
interfaces to JavaScript capabilities, a Foreign Function Interface.
\begin{verbatim}
alert :: JSString -> JS t ()
alert = fun "alert"

getElementById :: JSString -> JSObject -> JS t JSCanvas
getElementById = invoke "getElementById"
\end{verbatim}
Notice, that calling a JavaScript function or method is done by giving its
name to one of the provided combinators. Types can be specialized using 
a Haskell type annotation. A flexible and easy to use approach.

\section{The Sunroof Server}
\label{sec:server}

The Sunroof compiler can compile JavaScript than can be used
stand-alone inside a web application. But Sunroof really comes
into its own when used with the Sunroof server. There
are three major functions in our server.

\begin{verbatim}
sunroofServer :: ... -> (SunroofEngine -> IO ()) -> IO ()
sync          :: SunroofEngine -> JS t a  -> IO (ResultOf a)
async         :: SunroofEngine -> JS t () -> IO ()
\end{verbatim}        

\Src{sunroofServer} starts a small web server,
that calls the callback function for each request.
\Src{sync} and \Src{async} allow the Sunroof programmer
to remotely execute monadic JavaScript from the server.
\Src{ResultOf} is a type-function, that maps the 
Sunroof type to a corresponding Haskell type.

\section{Extended Example}
\label{sec:extended-example}

\FigureS%
{fig:example-application}%
{figures/example-application.png}%
{The example application running on the Sunroof server.}%
{scale=0.8}

As an example we developed is a small calculator. It uses Sunroof to
render its interface in the browser (Figure \ref{fig:example-application}). The 
Haskell server processes the arithmetic expressions by parsing them
into an AST and calculating the result of the expression.
The result is then sent back to the interface and displayed,
using Sunroof-generated JavaScript, 
utilizing the browsers rendering capabilities.

\section{Background - RESOURCE COLLECTION}
\label{sec:background}

We chose to implement Sunroof as deep embedded DSL for different
reasons. Porting a compiler, like GHC, to Javascript would have
cost a lot of effort. There is no clear way to port the existing
API and there are many language features that would be difficult to
translate. Providing a foreign function interface would 
introduce the cost of translating back- and forward. 
\TODO{Not even sure how this would work or where cost comes in.}
Another possibility would be to develop a custom language. 
Besides that people would need to learn a third language for this 
approach, it would involve effort similar to porting a compiler.
As a deep embedded DSL people only have to know Haskell to learn
the new language. The cost of developing the DSL remains light, 
because we can build upon a mature language and its features.

Choices when compiling to Javascript
\begin{itemize}
\item Port a compiler to Javascript - issues, mismatching API, other efforts.
\item Provide an FFI - back and forward cost.
\item Compile a custom, cut down FP language. - example
\item DSL, restrictive, syntactically, how do we handle binding?
\end{itemize}

\subsection{Basic Idea}

As we have seen in the previous example application Sunroof uses a 
monad to model computations that can be translated to Javascript.
A binding operation is thought to be an assignment to 
a fresh variable in the reified language.

We use monad reification.
\begin{itemize}
\item binding in Haskell becomes binding in reified language.
\item Feels like a native monad, cf STM.
\item Can build abstractions on top of this.
\end{itemize}

Design objectives and contributions.
\begin{itemize}
\item 
Specifically, we want to explore how similar to a native Haskell monad,
like the IO or STM monads, can we make the JS monad feel to Sunroof users.
\item We want to reflect objects. (easy)

\item We want also investigate the relationship between Haskell functions
and Javascript functions. In particular, there is an isomorphism
between monadic functions in Haskell, with the type \verb|a -> JS b|,
and Javascript functions, which we will notate using \verb|JSFunction|.
This relationship is more interesting than a Haskell synonym;
it reflects the reification options for capturing functions.
\end{itemize}

 
Compiling
\begin{itemize}
\item If we transliterate, we have straight line code, can not pause.
    (Wait for Mvar, for example)
\item If we CPS translate, we can use continuations to capture the
   notion of a paused thread. Works well.
   Problems:
  \begin{itemize}
   \item Can not translate functions, how do they get there return value
   \item (Assumes straight line code.)
   \item Also, the code becomes unreadable to anyone except a die-hard 
       compiler freak.
  \end{itemize}
\end{itemize}
 
Choice:
\begin{itemize}
\item We support both!
\item Phantom argument to JS
\item A = Atomic, B = Blockable.
\end{itemize}

Highlights
\begin{itemize}
\item function / continuation
\item implementing wait \& fork
\end{itemize}

\begin{table}
\caption{Major instances of the Sunroof class}
\begin{center}
\begin{tabular}{r@{\quad}l@{\quad}l@{\quad}c}
\hline\rule{0pt}{12pt}%
  Constraint
  & Sunroof Type $\tau$
  & Haskell Analog \HaskellAnalog{$\tau$}
  & \Src{js} \\ \hline\rule{0pt}{12pt}%
  
  & \Src{()}       & \Src{()}     & $\checkmark$ \\
  & \Src{JSBool}   & \Src{Bool}   & $\checkmark$ \\
  & \Src{JSNumber} & \Src{Double} & $\checkmark$ \\
  & \Src{JSString} & \Src{String} & $\checkmark$ \\
  
  \Src{Sunroof $\alpha$}
  & \Src{JSArray $\alpha$} 
  & \Src{[$\HaskellAnalog{\alpha}$]}
  & $\checkmark$ \\
  
  & \Src{JSMap JSString $\beta$}
  & \Src{Map String $\HaskellAnalog{\beta}$}
  & \\
  
  \Src{SunroofArgument $\alpha$}
  & \Src{JSFunction $\alpha$ $\beta$ }
  & \Src{$\HaskellAnalog{\alpha}$ $\rightarrow$ JS$_\Src{A}$ $\HaskellAnalog{\beta}$} 
  & $\checkmark$ \\
  \Src{Sunroof $\beta$} \\
  
  \Src{SunroofArgument $\alpha$}
  & \Src{JSMVar $\alpha$}
  & \Src{MVar $\HaskellAnalog{\alpha}$}
  & \\
  
  \Src{SunroofArgument $\alpha$}
  & \Src{JSChan $\alpha$}
  & \Src{Chan $\HaskellAnalog{\alpha}$}
  & \\[2pt]
\hline
\end{tabular}
\end{center}
\end{table} 

\begin{table}
\caption{Reifying and calling Javascript functions}
\begin{center}
\begin{tabular}{r@{\quad}c@{\quad}c@{\quad}c@{\quad}c}
\hline\rule{0pt}{12pt}%

                & Monadic Function      & Reification   & Object in     & Invocation\\
                & in Haskell            & Function      & Javascript    & Function\\
\hline\rule{0pt}{12pt}%
  Functions
  & $\alpha\rightarrow\ $\Src{JS}$_\Src{A}~\beta$
  & \Src{function}
  & \Src{JSFunction}~$\alpha~\beta$
  & \Src{apply} \\
  Continuations
  & $\alpha\rightarrow\ $\Src{JS}$_\Src{B}~\Src{()}$
  & \Src{continuation}
  & \Src{JSContinuation}~$\alpha$
  & \Src{goto}\\
\hline
\end{tabular}
\end{center}
\end{table} 

Up to this point, Sunroof is academically interesting, but in a real sense
we are writing Javascript using Haskell syntax, so why not just write
Javascript? There are three things we have bought by using our DSL.
\begin{itemize}
\item Typing -- We have a simple monomorphic type system that helps development of Sunroof code.
\item Haskell Abstractions --
\item Dynamic Generation -- 
\end{itemize}

\section{Case Study - hp2ps}

hp2ps is a tools for graphically displaying GHC heap profiles.
The tools history is that is was written by Malcolm Wallace while working
for the York ??? group, in 1992. The tool works as a simple filter,
reading hp files, and outputting postscript, with the postscript
doing the heavy lifting of the graphical rendering.

\section{Conclusion / Discussion}

\section{Related Work}

-----------

Talk about Active, but not in any details.


 It was written
in C by 

To Javascript in Javascript (CoffeeScript).

Haskell to Javascript

Other Deep embeddings.

\section{Acknowledgment}

Conal Elliott - For his support with Data.Boolean / Number abstraction

%
% ---- Bibliography ----
%


\bibliographystyle{splncs03}
\bibliography{sunroof}

\end{document}
